\section{Teorema de Seifert-Van Kampen}
\label{teorema-de-seifert-van-kampen}

\begin{titlemize}{Lista de Dependências}
	\item \hyperref[grupo-fundamental]{Grupo fundamental};\\ %assunto1 é o label onde o Assunto 1 aparece
\end{titlemize}

%O teorema de Seifert-Van Kampen diz que 
\begin{thm}[Teorema de Seifert-Van Kampen (S-VK)]
    Seja $X=U\cup V$, onde $U,V,U\cap V$ são conjuntos abertos e conexos por caminhos, com $x\in U\cap V$. Considere as inclusões 
    \[i_U:U\cap V\hookrightarrow U,\;i_V:U\cap V\hookrightarrow V,\;j_U:U\hookrightarrow X,\;j_V:V\hookrightarrow X.\] 
    Então o seguinte diagrama 
    % https://q.uiver.app/#q=WzAsNCxbMCwxLCJcXHBpXzEoVVxcY2FwIFYseCkiXSxbMSwwLCJcXHBpXzEoVSx4KSJdLFsxLDIsIlxccGlfMShWLHgpIl0sWzIsMSwiXFxwaV8xKFgseCkiXSxbMCwxLCJpX3tVKn0iXSxbMCwyLCJpX3tWKn0iLDJdLFsxLDMsImpfe1UqfSJdLFsyLDMsImpfe1UqfSIsMl1d
\[\begin{tikzcd}
	& {\pi_1(U,x)} \\
	{\pi_1(U\cap V,x)} && {\pi_1(X,x)} \\
	& {\pi_1(V,x)}
	\arrow["{j_{U_*}}", from=1-2, to=2-3]
	\arrow["{i_{U_*}}", from=2-1, to=1-2]
	\arrow["{i_{V_*}}"', from=2-1, to=3-2]
	\arrow["{j_{V_*}}"', from=3-2, to=2-3]
\end{tikzcd}\]
é um \emph{pushout}.
\end{thm}
Como o \emph{pushout} é único a menos de isomorfismo, podemos afirmar que $\pi_1(X,x)$ é totalmente determinado pelos homomorfismos $i_{U_*}$ e $i_{V_*}$.

Nesta seção, discutimos alguns casos particulares e aplicações do teorema de S-VK. Manteremos as notações acima em todas as subseções.
\subsection{Caso A de Teorema de Seifert-Van Kampen} %afirmação aqui significa teorema/proposição/colorário/lema
\label{teorema-s-vk-caso-a-prop}
\begin{titlemize}{Lista de dependências}
	\item \hyperref[grupo-fundamental]{Grupo fundamental};\\
% quantas dependências forem necessárias.
\end{titlemize}
\begin{prop}
    Se $\pi_1(U,x)=\pi_1(V,x)=\{e\}$, então $\pi_1(X,x)=\{e\}$
\end{prop}
\begin{dem}
    É fácil verificar que $(\{e\},id_{\{e\}},id_{\{e\}})$ é o \emph{pushout} de $(\pi_1(U\cap V,x),i_{U_*},i_{V_*})$. Pela unicidade do \emph{pushout}, obtemos $\pi_1(X,x)=\{e\}$.
\end{dem}
\begin{titlemize}{Lista de consequências}
	\item \hyperref[grupo-fundamental-de-esferas-prop]{Grupo fundamental de esferas};\\ %'consequencia1' é o label onde o conceito Consequência 1 aparece
	%\item \hyperref[]{}
\end{titlemize}

\subsection{Grupo fundamental de esferas} %afirmação aqui significa teorema/proposição/colorário/lema
\label{grupo-fundamental-de-esferas-prop}
\begin{titlemize}{Lista de dependências}
	\item \hyperref[grupo-fundamental]{Grupo fundamental};\\
    \item \hyperref[teorema-s-vk-caso-a-prop]{Caso A de Teorema de Seifert-Van Kampen}.
% quantas dependências forem necessárias.
\end{titlemize}

\begin{corol}
    O grupo fundamental $\pi_1(\mathbb{S}^n,p)=\{e\}$ para todo $n\ge 2$.
\end{corol}
\begin{dem}
    Considere $U=\mathbb{S}^n\setminus\{(0,...,0,1)\}$ e $V=\mathbb{S}^n\setminus\{(0,...,0,-1)\}$. Nesse caso, $U\cap V$ é um aberto conexo por caminhos para todo $n\ge 2$. Como $U$ e $V$ são homeomorfos a $\mathbb{R}^n$, que é contrátil por ser convexo, temos que $\pi_1(U)=\pi_1(V)=\{e\}$. Pela proposição \ref{teorema-s-vk-caso-a-prop}, obtemos $\pi_1(\mathbb{S}^2)=\{e\}$ para todo $n\ge 2$. Esse argumento não se aplica para $n=1$, pois $U\cap V$ não é conexo por caminhos. 
\end{dem}
\subsection{Caso B de Teorema de Seifert-Van Kampen} %afirmação aqui significa teorema/proposição/colorário/lema
\label{teorema-s-vk-caso-b-prop}
\begin{titlemize}{Lista de dependências}
    \item \hyperref[colagem-de-n-celula-def]{Colagem de n-célula};\\
    \item \hyperref[grupo-fundamental]{Grupo fundamental};\\
    \item \hyperref[variedade-def]{Variedade topológica};\\
    \item \hyperref[grupo-fundamental-de-esferas-prop]{Grupo fundamental de esferas}
% quantas dependências forem necessárias.
\end{titlemize}
\begin{prop}
    Se $\pi_1(U\cap V,x)=\pi_1(V,x)=\{e\}$, então $j_{U_*}:\pi_1(U,x)\rightarrow \pi_1(X,x)$ é um isomorfismo.
\end{prop}
\begin{dem}
    É fácil verificar que $(\pi_1(U,x),\{e\}\hookrightarrow \pi_1(U,x), id_{\pi_1(U,x)})$ é o \emph{pushout} de $(\pi_1(U\cap V,x),i_{U_*},i_{V_*})$. Pela unicidade do \emph{pushout}, $\pi_1(U,x)$ é único a menos de isomorfismo, o que implica que $j_{U_*}$ é um isomorfismo.
\end{dem}

\begin{corol}
    Se $M$ uma variedade conexa de dimensão maior ou igual $3$ com $x\in M$, então $\pi_1(M-\{x\},p)\cong\pi_1(M,p)$ para todo $p\in M\setminus\{x\}$.
\end{corol}
\begin{dem}
Pela definição de variedade topológico, existe uma vizinhança aberta de $x$ em $M$, tal que $U$ é homeomorfo a $\text{int} (D^n)$. Considere $V=M\setminus\{x\}$, assim $U\cap V$ é homeomorfo a $\text{int}(D^n)\setminus \{0\}$ que é homotopicamente equivalente a $\mathbb{S}^{n-1}$. Como $U$, $V$ e $U\cap V$ são abertos conexo por caminhos e $\pi_1(\mathbb{S}^{n-1},p)=\pi_1(\text{int}(D^n),0)=\{e\}$ (o grupo fundamental da esfera pode ser encontrado em \ref{grupo-fundamental-de-esferas-prop} e \ref{grupo-fundamental-de-S1-prop}) para todo $n\ge 3$, pela proposição anterior, temos $\pi_1(M-\{x\},p)\cong\pi_1(M,p)$.
\end{dem}

\begin{corol}
    Seja $X$ um espaço Hausdorff conexo por caminhos. Seja $f:\mathbb{S}^{n-1}\rightarrow X$ uma função contínua e $i:\mathbb{S}^{n-1}\hookrightarrow D^n$ uma inclusão, onde $n\ge 3$. Denotamos o espaço obtido de $X$ pela colagem de uma $n$-célula por meio da função $f$ por $X_f$. Então, temos que $\pi_1(X,h^{-1}(p))\cong \pi_1(X_f, p)$ para todo ponto $p\in h(\text{int}(D^n))\cap X_f\setminus\{h(0)\}$, onde $\pi:X\rightarrow X_f$, $h:D^n\rightarrow X_f$ são as funções associadas ao \emph{pushout}.
\end{corol}
\begin{dem}
     Consideramos $V=h(\text{int}(D^n))$ e $U=X_f\setminus \{h(0)\}$. Como discutido em \ref{sequencia-exata-da-colagem-prop}, temos que $V$ é homeomorfo a $\text{int}(D^n)$, $U$ é homotopicamente equivalente a $X$ e $U\cap V$ é homotopicamente equivalente a $\mathbb{S}^{n-1}$. Dessa forma, temos que $\pi_1(U,p)=\pi_1(U\cap V,p)=\{e\}$ para todo $n\ge 3$ e $p\in U\cap V$. Pela proposição anterior, temos $\pi_1(X,h^{-1}(p))\cong\pi_1(U,p)\cong \pi_1(X_f, p)$, para todo $p\in U\cap V$
\end{dem}
Aqui, o ponto base $h^{-1}(p)$ não é relevante, pois $X$ é um espaço conexo por caminhos. Usamos $h^{-1}(p)$, pois $h$ é um homeomorfismo entre $V$ e $\text{int}(D^n)$ o que garante que $h^{-1}(p)$ é um ponto em $X$.
\subsection{Caso C de Teorema de Seifert-Van Kampen} %afirmação aqui significa teorema/proposição/colorário/lema
\label{teorema-s-vk-caso-c-prop}
\begin{titlemize}{Lista de dependências}
    \item \hyperref[pushout-de-grupos-prop]{\emph{Pushout} de grupos};\\
    \item \hyperref[grupo-fundamental]{Grupo fundamental};\\
    \item \hyperref[teorema-s-vk-caso-b-prop]{Caso B de Teorema de Seifert-Van Kampen}.
% quantas dependências forem necessárias.
\end{titlemize}
Note que, se $\pi_1(V,x)=\{e\}$, então a condição $j_{U_*}\circ i_{U_*}=j_{V_*}\circ i_{V_*}= e$ é equivalente a $\text{Im}(i_{U_*})\subseteq \text{Ker}(j_{U_*})$. A propriedade universal do \emph{pushout} então reduz a condição: para todo homomorfismo de grupos $\phi:\pi_1(U,x)\rightarrow H$ satisfazendo $\text{Im}(i_{U_*})\subseteq \text{Ker}(\phi)$, existe um único homomorfismo $\psi:\pi_1(X,x)\rightarrow H$ tal que $\psi\circ j_{U_*}=\phi$.

Essa condição decorre do teorema do homomorfismo de grupos, quando $\text{Im}(i_{U_*})$ é um subgrupo normal de $\pi_1(U,x)$. No entanto, em geral, $\text{Im}(i_{U_*})$ não é um subgrupo normal. Para podermos aplicar o teorema do homomorfismo de grupos, podemos "normalizar" $\text{Im}(i_{U_*})$. 

\begin{prop}
    Se $\pi_1(V,x)=\{e\}$, então $\pi_1(X,x)\cong \pi_1(U,x)/\ \overline{\textnormal{Im}(i_{U_*})}$.
\end{prop}

\begin{dem}
    Essa proposição é uma consequência direta de um corolário apresentado na subseção \ref{pushout-de-grupos-prop}.
    %Basta provar que $\pi_1(U,x)/\overline{\text{Im}(i_{U_*})},\pi,\{e\}\hookrightarrow \pi_1(U,x)/\overline{\text{Im}(i_{U_*})} )$ é o pushout de $(\pi_1(U\cap V,x),i_{U_*},i_{V_*})$, onde $\pi:\pi_1(U,x)\rightarrow \pi_1(U,x)/\overline{\text{Im}(i_{U_*})}$ é a projeção canônica do quociente. Note que, todo homomorfismo de grupos $\phi:\pi_1(U,x)\rightarrow H$ satisfazendo $\text{Im}(i_{U_*})\subseteq \text{Ker}(\phi)$ também satisfaz $\overline{\text{Im}(i_{U_*})}\subseteq \text{Ker}(\phi)$, pela proposição \ref{fecho normal-def} e pela normalidade de $\text{Ker}(\phi)$. Assim, pelo teorema do homomorfismo, existe um único homomorfismo $\psi:\pi_1(U,x)/\overline{\text{Im}(i_{U_*})}\rightarrow H$ tal que $\psi\circ j_{U_*}=\phi$.
\end{dem}

Agora, podemos analisar o grupo fundamental do espaço obtido pela colagem de uma 2-célula.

\begin{corol}
    Seja $X$ um espaço Hausdorff conexo por caminhos com $x\in X$. Seja $f:\mathbb{S}^{1}\rightarrow X$ uma função contínua e $i:\mathbb{S}^{1}\hookrightarrow D^2$ uma inclusão. Denotamos o espaço obtido de $X$ pela colagem de uma $n$-célula por meio da função $f$ por $X_f$. Então, temos que $\pi_1(X_f, p)\cong \pi_1(X,h^{-1}(p))/\overline{\text{Im}(f_*)}$ para todo ponto $p\in h(\text{int}(D^2))\cap X_f\setminus\{h(0)\}$, onde $\pi:X\rightarrow X_f$, $h:D^2\rightarrow X_f$ são as funções associadas ao \emph{pushout}.
\end{corol}
\begin{dem}
     Consideramos $V=h(\text{int}(D^2))$ e $U=X_f\setminus \{h(0)\}$. Como discutido em \ref{sequencia-exata-da-colagem-prop}, temos que $V$ é homeomorfo a $\text{int}(D^2)$, $U$ é homotopicamente equivalente a $X$ e $U\cap V$ é homotopicamente equivalente a $\mathbb{S}^{1}$. Dessa forma, temos $\pi_1(U,p)=\{e\}$ para todo $p\in U\cap V$. Pela proposição anterior, temos que $\pi_1(X_f,p)\cong \pi_1(U, p)/\overline{\text{Im}(i_{U_*})}$, para todo $p\in U\cap V$. Pelo diagrama comutativo seguinte 
     % https://q.uiver.app/#q=WzAsNixbMCwwLCJcXHBpXzEoVVxcY2FwIFYscCkiXSxbMCwxLCJcXHBpXzEoXFxtYXRoYmJ7U31eMSkiXSxbMSwwLCJcXHBpXzEoVSxwKSJdLFsxLDEsIlxccGlfMShYLGheey0xfShwKSkiXSxbMiwwLCJcXHBpXzEoWF9mLHApIl0sWzIsMSwiXFxwaV8xKFhfZixwKSJdLFswLDIsImlfe1VfKn0iXSxbMiw0LCJqX3tVXyp9Il0sWzEsMywiZl8qIiwyXSxbMyw1LCJcXHBpXyoiLDJdLFs0LDUsIj0iLDFdLFswLDEsIlxcY29uZyIsMV0sWzIsMywiXFxjb25nIiwxXV0=
\[\begin{tikzcd}
	{\pi_1(U\cap V,p)} & {\pi_1(U,p)} & {\pi_1(X_f,p)} \\
	{\pi_1(\mathbb{S}^1)} & {\pi_1(X,h^{-1}(p))} & {\pi_1(X_f,p)},
	\arrow["{i_{U_*}}", from=1-1, to=1-2]
	\arrow["\cong"{description}, from=1-1, to=2-1]
	\arrow["{j_{U_*}}", from=1-2, to=1-3]
	\arrow["\cong"{description}, from=1-2, to=2-2]
	\arrow["{=}"{description}, from=1-3, to=2-3]
	\arrow["{f_*}"', from=2-1, to=2-2]
	\arrow["{\pi_*}"', from=2-2, to=2-3]
\end{tikzcd}\]
     temos que $\pi_1(X_f, p)\cong \pi_1(X,h^{-1}(p))/\overline{\text{Im}(f_*)}$, para todo $p\in U\cap V$.
\end{dem}

Como mencionado no final de \ref{teorema-s-vk-caso-b-prop}, o ponto base $h^{-1}(p)$ não é relevante.

\subsection{Caso D de Teorema de Seifert-Van Kampen} %afirmação aqui significa teorema/proposição/colorário/lema
\label{teorema-s-vk-caso-d-prop}
\begin{titlemize}{Lista de dependências}
	\item \hyperref[grupo-fundamental]{Grupo fundamental};\\
    \item \hyperref[pushout-de-grupos-prop]{\emph{Pushout} de grupos}.\\
% quantas dependências forem necessárias.
\end{titlemize}
\begin{prop}
    Se $\pi_1(U\cap V,x)=\{e\}$, então $\pi_1(X,x)\cong\pi_1(U,x)*\pi_1(V,x).$
\end{prop}
\begin{dem}
    A prova segue da proposição apresentada em subseção \ref{pushout-de-grupos-prop}.
\end{dem}


\subsection{Caso geral de Teorema de Seifert-Van Kampen} %afirmação aqui significa teorema/proposição/colorário/lema
\label{teorema-s-vk-caso-geral-prop}
\begin{titlemize}{Lista de dependências}
    \item \hyperref[geradores-relacoes-def]{Geradores e Relações};\\
	\item \hyperref[grupo-fundamental]{Grupo fundamental};\\
    \item \hyperref[pushout-de-grupos-prop]{\emph{Pushout} de grupos}.\\
% quantas dependências forem necessárias.
\end{titlemize}

Como todo grupo pode ser apresentado em termos de geradores e relações, pelo teorema apresentado na subseção \ref{pushout-de-grupos-prop}, o teorema de Seifert-Van Kampen pode ser formulado como 
\begin{thm}
    Seja $X=U\cup V$, onde $U,V,U\cap V$ são conjuntos abertos e conexos por caminhos, com $x\in U\cap V$. Considere as inclusões 
    \[i_U:U\cap V\hookrightarrow U,\;i_V:U\cap V\hookrightarrow V,\;j_U:U\hookrightarrow X,\;j_V:V\hookrightarrow X.\]
    Sejam também
    \begin{itemize}
        \item $\pi_1(U,x)=F(S_1)/\langle R_1\rangle,$
        \item $\pi_1(V,x)=F(S_2)/\langle R_2\rangle,$
        \item $\pi_1(U\cap V,x)=F(S)/\langle R\rangle,$
    \end{itemize}
    onde $F(S_1)$ e $F(S_2)$ são os grupos livres gerados por $S_1$ e $S_2$, e $\langle R_1\rangle$, $\langle R_2\rangle$, $\langle R\rangle$ são os subgrupos normais correspondentes.

    Para cada $s\in S$, tomamos $f_s\in F(S_1)$ e $g_s\in F(S_2)$, de modo que
    \[i_{U_*}(s\langle R\rangle)=f_s\langle R_1\rangle\;\;\text{ e }\;\;i_{V_*}(s\langle R\rangle)=g_s\langle R_2\rangle.\]
    Defina o conjunto
    \[R'=\{f_sg^{-1}_s:s\in S\}\subset F(S_1)*F(S_2)=F(S_1\cup S_2).\]
    Então, Então, o grupo fundamental de $X$ em $x$ é dado por
    \[\pi_1(X,x)=\frac{F(S_1\cup S_2)}{\langle R_1\cup R_2\cup R' \rangle}.\]
\end{thm}