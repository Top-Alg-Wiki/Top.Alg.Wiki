%---------------------------------------------------------------------------------------------------------------------!Draft!-----------------------------------------------------------------------------------------------------------------
\subsection{Espaço localmente conexo por caminhos}
\label{localmente-conexo-por-caminhos-def}
%\begin{titlemize}{Lista de dependências}
	%\item \hyperref[dependecia1]{Dependência 1};\\ %'dependencia1' é o label onde o conceito Dependência 1 aparece (--à arrumar um padrão para referencias e labels--) 
	%\item \hyperref[]{};\\
% quantas dependências forem necessárias.
%\end{titlemize}
\begin{defi}[Espaço localmente conexo por caminhos]
	Um espaço topológico $X$ é \textit{\textbf{localmente conexo por caminhos}} se para toda vizinhança $U$ de $x \in X$ existe vizinhança aberta $V \subseteq U$ tal que $V$ é conexo por caminhos.
\end{defi}

Apesar do nome sugestivo, temos que
\begin{center}
Conexo por caminhos $\nRightarrow$ Localmente conexo por caminhos
\end{center}

\begin{titlemize}{Lista de consequências}
	\item \hyperref[localmente-conexo-por-caminhos-ex]{Espaços conexos por caminhos mas não localmente conexos por caminhos};\\ %'consequencia1' é o label onde o conceito Consequência 1 aparece
	\item \hyperref[levantamento-de-funções-prop]{Levantamento de funções}
\end{titlemize}

%[Bianca]: é mais fácil criar a lista de dependências do que a de consequências.