\subsection{Automorfismo de um recobrimento}
\label{automorfismo-de-recobrimento-def}
\begin{titlemize}{Lista de dependências}
	\item \hyperref[ações-de-grupo-def]{Ações de grupo};
\end{titlemize}
\begin{defi}[Automorfismo de um recobrimento]
    Um automorfismo de um recobrimento $q:E \longrightarrow X$ é um homeomorfismo $\phi:E \longrightarrow E$ tal que:
    \[\begin{tikzcd}
	E && E \\
	\\
	& X
	\arrow["\phi", from=1-1, to=1-3]
	\arrow["q"', from=1-1, to=3-2]
	\arrow["q", from=1-3, to=3-2]
    \end{tikzcd}\]
\end{defi}

Denotamos o grupo formado por todos os automorfismos de um recobrimento, também chamado de grupo de transformações de Deck, por $Aut(E, q, X)$. Por definição, para todo $x \in X$, a ação $Aut(E, q, X) \circlearrowright q^{-1}(x)$ é dada por: $$\phi \cdot e = \phi(e)$$

\begin{titlemize}{Lista de consequências}
	\item \hyperref[acao-de-automorfismos-e-livre-prop]{A ação do grupo de automorfismos é livre sobre as fibras};\\
    \item \hyperref[acao-de-automorfismo-transitiva-prop]{Quando a ação do grupo de automorfismos é transitiva sobre as fibras?};\\
    \item \hyperref[g-recobrimento-regular-def]{G-recobrimento regular};
\end{titlemize}
