%-------------------------------------------------------------------------------------------------------------!Draft!-------------------------------------------------------------------------------------------------------------------------
\section{Topologia quociente}
\label{topologia-quociente}
Um assunto que aparece de forma recorrente na topologia algébrica é o conceito de topologia quociente, que exploraremos a seguir. 

\subsection{Topologia Quociente}
\label{topologia-quociente-def}
\begin{titlemize}{Lista de dependências}
	\item \hyperref[topologia-final]{Topologia Final}; 
\end{titlemize}
\begin{defi}[Topologia Quociente]
	Seja \(X\) um espaço topológico e \(\sim\) uma relação de equivalência em \(X\).
	Podemos conferir ao espaço \(X/\sim\) uma estrutura de espaço topologíco da seguinte maneira. Considere a função projeção
	\begin{align*}
		\pi:X&\to X/\sim;\\
		x&\mapsto [x].
	\end{align*}
	Podemos fazer com que \(\pi\) seja uma função contínua dando, para \(X/\sim\) a topologia final com relação à \(\pi\). Isto é, os abertos de \(X/\sim\) são exatamente imagens de abertos em \(X\) por \(\pi\).  
\end{defi}
Varios exemplos importantes de espaços topológicos com os quais trabalharemos no estudo de topologia algébrica podem ser construídos como espaços quocientes. Em particular uma construção muito útil é a de quocientar um espaço por um subespaço, como explicado na seguinte definição.
\begin{defi}[Quociente por um subespaço]
	Seja \(X\) um espaço topológico e \(A \subseteq X\) um subespaço. Definimos a seguinte relação binária, \(\sim_{A}\):\\
	\(a\sim b\) se e somente se \(a=b\) ou \(a,b\in A\). Essa relação é uma relação de equivalência, e assim define um espaço \(X/\sim_A\). Esse espaço será denotado por \(X/A\). 
\end{defi}
Um exemplo desse tipo de construção é a esfera \(S^1\) que pode ser construida como \(I/\{0,1\}\) onde \(I=[0,1]\). 
\begin{titlemize}{Lista de consequências}
	\item \hyperref[pinched-torus-ex]{Torus Pinçado};
\end{titlemize}


% onde conteudos.tex é o nome do arquivo tex que voce quer incluir nessa secção.
\subsection{Função contínua em topologia quociente}
\label{funcao-continua-em-topologia-quociente-prop}
\begin{titlemize}{Lista de dependências}
	\item \hyperref[topologia-quociente-def]{Topologia quociente}; 
\end{titlemize}

\begin{prop}
    Sejam \(X,Y\) espaços topológicos, e seja \(\sim\) uma relação de equivalência em \(X\). Uma função $f:(X/\sim) \longrightarrow Y$ é contínua se e somente se $f\circ \pi$ é contínua, onde $\pi$ é a função projeção associada ao quociente.
\end{prop}
\begin{dem}
    Por um lado, suponhamos que $f$ seja contínua. Como a composição de funções contínuas é contínua, a função $f\circ \pi$ também seré contínua.

    Por outro lado, suponhamos que $f\circ \pi$ seja contínua. Seja $V\subseteq Y$ um aberto, pela hipóteses, $\pi^{-1}(f^{-1}(V))$ é um aberto. Agora, pela definição de topologia quociente, $f^{-1}(V)$ é um aberto em $X/\sim$, o que implica que $f$ é contínua. 
\end{dem}

Dois exemplos importantes de espaços quociente são os seguintes.
%-------------------------------------------------------------------------------------------------------------!Draft!-------------------------------------------------------------------------------------------------------------------------
\section{Cone e Suspensão sobre Espaços Topológicos}
\label{cone-suspensao}
\subimport{}{cone-def}
\subimport{}{suspensao-def}
Seguem alguns resultados relevantes sobre cones e suspensões:
\subimport{}{suspensao-cone-duplo-prop}
\subimport{}{cone-euclidiano-prop}
\subimport{}{suspensao-euclidiano-prop}
\subimport{}{cone-esfera-prop}
\subimport{}{suspensao-esfera-prop}
%---------------------------------------------------------------------------------------------------------------------!Draft!-----------------------------------------------------------------------------------------------------------------
\subsection{Espaços Quociente e a propriedade Hausdorff} %afirmação aqui significa teorema/proposição/colorário/lema
\label{topologia-quociente-hausdorff-thm}
\begin{titlemize}{Lista de dependências}
	\item \hyperref[topologia-quociente-def]{Espaços Quociente};\\ %'dependencia1' é o label onde o conceito Dependência 1 aparece (--à arrumar um padrão para referencias e labels--) 
% quantas dependências forem necessárias.
\end{titlemize}
Comentário sobre os objetos envolvidos na afirmação.
\begin{thm}[Espaços quocientes Hausdorff]% ou af(afirmação)/prop(proposição)/corol(corolário)/lemma(lema)/outros ambientes devem ser definidos no preambulo de Alg.Top-Wiki.tex 
Sejam $X$ um espaço Hausdorff e $\sim$ uma relação de equivalência em $X$ para a qual a projeção $\pi: X \rightarrow X/\sim$ é uma aplicação aberta. Defina o conjunto $R=\{(x,x')\in X\times X| x\sim x'\}$.

Então $X/\sim$ é Hausdorff se, e somente se, $R\subset X\times X$ é fechado.

\end{thm}
\begin{dem}
    $(\Longrightarrow)$ Se $X/\sim$ é de Hausdorff, gostaríamos de mostrar que $X\times X\backslash R$ é aberto. Para qualquer ponto $(x,x')\in (X\times X)\backslash R$, $x$ e $x'$ são tais que $\pi(x)\neq \pi(x')$. Como $X/\sim$ é de Hausdorff, existem abertos $U_x$ e $U_{x'}$ disjuntos em $X/\sim$ que são vizinhanças abertas de $\pi(x)$ e de $\pi(x')$, respectivamente.% e tais que $U_x\cap U_x' = \emptyset$.

    Temos ainda que $\pi^{-1}(U_x)$ e $\pi^{-1}(U_{x'})$ são abertos, pois a topologia de $X/\sim$ é a topologia quociente, e o produto $U=\pi^{-1}(U_x)\times \pi^{-1}(U_{x'})$ é aberto de $X\times X$ na topologia produto. Além disso, $(x,x')\in U$. Afirmamos que $U\subset X\times X\backslash R$. De fato, se $U\cap R\neq \emptyset$, teríamos $(v_1,v_2)\in U\cap R$ tal que $\pi(v_1)=\pi(v_2)$, mas $v_1 \in \pi^{-1}(U_x)$ e $v_2\in \pi^{-1}(U_{x'})$, e desse modo $\pi(v_1) = \pi(v_2) \in U_x \cap U_{x'} = \varnothing$, absurdo. Portanto, para todo $(x,x')\in X\times X\backslash R$, é possível encontrar uma vizinhança aberta $U$ de $(x,x')$ contida em $X\times X\backslash R$; $R$ é fechado, como queríamos.\newline

    $(\Longleftarrow)$ Dado que $R$ é fechado, gostaríamos de encontrar vizinhanças disjuntas de $a,~b\in X/\sim$ quaisquer para concluir que $X/\sim$ é Hausdorff. Sabemos que existem $x,~y\in X$ tais que $\pi(x)=a$ e $\pi(y)=b$ pois a projeção $\pi$ é uma aplicação sobrejetora. Como $X$ é de Hausdorff, existem abertos disjuntos $U_x$ e $U_y$, vizinhanças de $x$ e de $y$, respectivamente. Além disso, uma vez que $R$ é fechado, $X\times X\backslash R$ é aberto e, portanto, $(U_x\times U_y)\cap((X\times X)\backslash R)$ é aberto na topologia produto.

    Sejam $p_1:X\times X\rightarrow X$ e $p_2:X\times X\rightarrow X$ definidos por $$p_1(x_1,x_2)=x_1,\qquad p_2(x_1,x_2)=x_2 \qquad\forall (x_1,x_2)\in X\times X.$$ Como a topologia produto em $X\times Y$ é gerada pela base dada pelos produtos de abertos $X$ e de $Y$, é possível concluir que $p_1$ e $p_2$ são aplicações abertas. Desse modo, $U_1=p_1((U_x\times U_y)\cap((X\times X)\backslash R))$ e $U_2=p_2((U_x\times U_y)\cap((X\times X)\backslash R))$ são abertos em $X$. Por fim, basta observar que os abertos $\pi(U_1)$ e $\pi(U_2)$ são tais que $\pi(U_1)\cap \pi(U_2)=\emptyset$ uma vez que se $v\in \pi(U_1)\cap\pi(U_2)$, teríamos $v=\pi(v_1)$ para algum $v_1\in U_1$ e $v=\pi(v_2)$ para algum $v_2\in U_2$, o que implicaria $v_1\sim v_2$, um absurdo pois, pela construção de $U_1$ e $U_2$, $(v_1,v_2)\not\in R$.  Também temos $a\in U_1$ e $b\in U_2$ pois, como $a\neq b$, $x\not\sim y$. Encontramos assim os dois abertos que separam $a$ e $b$, mostrando que $X/\sim$ é Hausdorff.
\end{dem}

Comentários sobre a afirmação.

\begin{titlemize}{Lista de consequências}
	\item \hyperref[consequencia1]{Consequência 1};\\ %'consequencia1' é o label onde o conceito Consequência 1 aparece
\end{titlemize}


O espaço quociente também é essencial para realizar a colagem dos espaços.
\subsection{Pushout de espaços topológicos} %afirmação aqui significa teorema/proposição/colorário/lema
\label{pushout-de-espacos-topologicos-def}
\begin{titlemize}{Lista de dependências}
	\item \hyperref[topologia-quociente-def]{Espaços Quociente};\\ %'dependencia1' é o label onde o conceito Dependência 1 aparece (--à arrumar um padrão para referencias e labels--) 
% quantas dependências forem necessárias.
\end{titlemize}

\begin{defi}
    Sejam $X,Y,Z$ espaços topológicos. E Sejam $f:Z\rightarrow X$ e $g:Z\rightarrow Y$ funções contínuas. O \textbf{Pushout} de $f$ e $g$ é o espaço quociente $X\sqcup Y/\sim$, onde $\sim$ é a menor relação de equivalência que contém $\{(f(z),g(z))\in X\times Y:z\in Z\}$.
    % https://q.uiver.app/#q=WzAsNCxbMCwwLCJaIl0sWzAsMSwiWCJdLFsxLDAsIlkiXSxbMSwxLCJYXFxzcWN1cCBZL1xcc2ltIl0sWzAsMSwiZiIsMl0sWzAsMiwiZyJdLFsxLDNdLFsyLDNdXQ==
\[\begin{tikzcd}
	Z & Y \\
	X & {X\sqcup Y/\sim}
	\arrow["g", from=1-1, to=1-2]
	\arrow["f"', from=1-1, to=2-1]
	\arrow[from=1-2, to=2-2]
	\arrow[from=2-1, to=2-2]
\end{tikzcd}\]
\end{defi}

\subsection{Colagem de n-célula} %afirmação aqui significa teorema/proposição/colorário/lema
\label{colagem-de-n-celula-def}
\begin{titlemize}{Lista de dependências}
	\item \hyperref[topologia-quociente-def]{Espaços Quociente};\\
    \item \hyperref[pushout-de-espacos-topologicos-def]{\emph{Pushout} de espaços topológicos}.%'dependencia1' é o label onde o conceito Dependência 1 aparece (--à arrumar um padrão para referencias e labels--) 
% quantas dependências forem necessárias.
\end{titlemize}

\begin{defi}
    Seja $X$ um espaço topológico, e sejam $f:\mathbb{S}^{n-1}\rightarrow X$ uma função contínua e $i:\mathbb{S}^{n-1}\hookrightarrow D^n$ uma inclusão, onde $n\ge 2$. O \textbf{espaço obtido de} $X$ \textbf{pela colagem de uma $n$-célula por meio da função} $f$ é o \emph{pushout} de $f$ e $i$, denotado por $X_f$ ou $D^n\cup_f X$.
\end{defi}

%\begin{titlemize}{Lista de consequências}
	%\item %\hyperref[consequencia1]{Consequência 1};\\ %'consequencia1' é o label onde o conceito Consequência 1 aparece
%\end{titlemize}

\subsection{Colagem de um disco com um ponto} %afirmação aqui significa teorema/proposição/colorário/lema
\label{colagem-de-um-disco-com-um-ponto-ex}
\begin{titlemize}{Lista de dependências}
	\item \hyperref[topologia-quociente-def]{Espaços Quociente};\\
    \item \hyperref[pushout-de-espacos-topologicos-def]{Pushout de espaços topológicos};\\
    \item \hyperref[colagem-de-n-celula-def]{Colagem de n-célula}%'dependencia1' é o label onde o conceito Dependência 1 aparece (--à arrumar um padrão para referencias e labels--) 
% quantas dependências forem necessárias.
\end{titlemize}

\begin{ex}
    Para $n\ge 2$, a colagem $\{x\}_f=D^n\cup_f \{x\}$, em que $f:\mathbb{S}^{n-1}\rightarrow \{x\}$ é a função constante, é a esfera $\mathbb{S}^n$. Visto que $int(D^n)$ é homeomorfo a $\mathbb{R}^n$, e $\mathbb{R}^n$ é homeomorfo a $\mathbb{S}^n\setminus\{\text{polo norte}\}$, e que a colagem garante que $\{x\}_f\setminus\{\text{origem do disco }D^n \}$ é homeomorfo a $\mathbb{S}^n\setminus\{\text{polo sul}\}$, conclui-se que $\{x\}_f$ é homeomorfo à esfera $\mathbb{S}^n$.
\end{ex}

%\begin{titlemize}{Lista de consequências}
	%\item %\hyperref[consequencia1]{Consequência 1};\\ %'consequencia1' é o label onde o conceito Consequência 1 aparece
%\end{titlemize}

%%% Local Variables:
%%% mode: LaTeX
%%% TeX-master: "../Alg.Top-Wiki"
%%% End:

