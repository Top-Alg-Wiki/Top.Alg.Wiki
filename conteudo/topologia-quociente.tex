%-------------------------------------------------------------------------------------------------------------!Draft!-------------------------------------------------------------------------------------------------------------------------
\section{Topologia quociente}
\label{topologia-quociente}
Um assunto que aparece de forma recorrente na topologia algébrica é o conceito de topologia quociente, que exploraremos a seguir. 

\subsection{Topologia Quociente}
\label{topologia-quociente-def}
% \begin{titlemize}{Lista de dependências}
% 	\item \hyperref[topologia-final]{Topologia Final}; 
% \end{titlemize}
\begin{defi}[Topologia Quociente]
	Seja \(X\) um espaço topológico e \(\sim\) uma relação de equivalência em \(X\).
	Podemos conferir ao espaço \(X/\sim\) uma estrutura de espaço topológico da seguinte maneira. Considere a função projeção
	\begin{align*}
		\pi:X&\to X/\sim;\\
		x&\mapsto [x].
	\end{align*}
	Podemos fazer com que \(\pi\) seja uma função contínua munindo \(X/\sim\) com a \emph{topologia final} com relação à \(\pi\). Isto é, um subconjunto de \(X/\sim\) é aberto se, e somente se, sua pré-imagem por $\pi$ é aberto de \(X\).
\end{defi}

Varios exemplos importantes de espaços topológicos com os quais trabalharemos no estudo de topologia algébrica podem ser construídos como espaços quocientes. Em particular, uma construção muito útil é a de tomar o quociente de um espaço por um subespaço, como explicado na seguinte definição.
\begin{defi}[Quociente por um subespaço]
	Seja \(X\) um espaço topológico e \(A \subseteq X\) um subespaço. Definimos a seguinte relação binária, \(\sim_A\):\\
    \centerline{
	\(a\sim_A b\) se e somente se \(a=b\) ou \(a,b\in A\).}\\ Essa relação é de equivalência, e assim definimos \(X/A = X/\sim_A\). 
\end{defi}

Vejamos alguns exemplos simples.

\begin{ex}
    \begin{itemize}
        \item O círculo \(\mathbb{S}^1 = \mathbb{T}^1\) pode ser construído como \(I/\{0,1\}\), onde \(I=[0,1]\).
        \item Mais geralmente, o $n$-toro $\mathbb{T}^n$ pode ser construído como $[0,1]^n/\sim$, onde $\sim$ é a relação de equivalência que identifica $x = (x_1,\ldots,x_n), y = (y_1,\ldots,y_n) \in [0,1]^n$ se existe $1\leq i\leq n$ tal que $x_j = y_j$ para todo $j \neq i$ e $\{x_i,x_j\} = \{0,1\}$, ou então se $x=y$.
    \end{itemize}
\end{ex}

\begin{titlemize}{Lista de consequências}
    \item \hyperref[funcao-continua-em-topologia-quociente-prop]{Função contínua em topologia quociente}
	\item \hyperref[topologia-quociente-hausdorff-thm]{Espaços quocientes Hausdorff}
\end{titlemize}


% onde conteudos.tex é o nome do arquivo tex que voce quer incluir nessa secção.
\subsection{Função contínua em topologia quociente}
\label{funcao-continua-em-topologia-quociente-prop}
\begin{titlemize}{Lista de dependências}
	\item \hyperref[topologia-quociente-def]{Topologia quociente}; 
\end{titlemize}

\begin{prop}
    Sejam $X,Y$ espaços topológicos. E seja $\sim$ uma relação de equivalência em $X$. Uma função $f:(X/\sim) \longrightarrow Y$ é contínua se e somente se $f\circ \pi$ é contínua, onde $\pi$ é a função projeção associada ao quociente.
\end{prop}
\begin{dem}
    Por um lado, suponhamos que $f$ seja contínua. Como a composição de funções contínuas é contínua, a função $f\circ \pi$ também seré contínua.

    Por outro lado, suponhamos que $f\circ \pi$ seja contínua. Seja $V\subseteq Y$ um aberto, pela hipóteses, $\pi^{-1}(f^{-1}(V))$ é um aberto. Agora, pela definição de topologia quociente, $f^{-1}(V)$ é um aberto em $X/\sim$, o que implica que $f$ é contínua. 
\end{dem}


Alguns exemplos importantes de espaços quociente são os seguintes.
\subsection{Espaço Projetivo}
\label{espaco-projetivo-def}
\begin{defi}
     Sejam $n\geq 0$ e $V$ um espaço vetorial sobre o corpo $\mathbb{K}$. Definimos o \textbf{espaço projetivo sobre $V$} como o espaço topológico quociente $\mathbb{P}(V) = V/\sim$, onde $x\sim y$ se, e somente, existe $\alpha \in \mathbb{K} \setminus\{0\}$ tal que $x = \alpha y$.

     O \textbf{espaço projetivo $n$-dimensional sobre $\mathbb{K}$} é definido como $\mathbb{KP}^n = \mathbb{P}(\mathbb{K}^n)$.
\end{defi}

\input{conteudo/cone-suspensao}
%---------------------------------------------------------------------------------------------------------------------!Draft!-----------------------------------------------------------------------------------------------------------------
\subsection{Espaços Quociente e a propriedade Hausdorff} %afirmação aqui significa teorema/proposição/colorário/lema
\label{topologia-quociente-hausdorff-thm}
\begin{titlemize}{Lista de dependências}
	\item \hyperref[topologia-quociente-def]{Espaços Quociente};\\ %'dependencia1' é o label onde o conceito Dependência 1 aparece (--à arrumar um padrão para referencias e labels--) 
% quantas dependências forem necessárias.
\end{titlemize}
%Comentário sobre os objetos envolvidos na afirmação.
\begin{thm}[Espaços quocientes Hausdorff]% ou af(afirmação)/prop(proposição)/corol(corolário)/lemma(lema)/outros ambientes devem ser definidos no preambulo de Alg.Top-Wiki.tex 
Sejam $X$ um espaço Hausdorff e $\sim$ uma relação de equivalência em $X$ para a qual a projeção $\pi: X \rightarrow X/\sim$ é uma aplicação aberta. Defina o conjunto $R=\{(x,x')\in X\times X| x\sim x'\}$.

Então $X/\sim$ é Hausdorff se, e somente se, $R\subset X\times X$ é fechado.

\end{thm}
\begin{dem}
    $(\Longrightarrow)$ Se $X/\sim$ é de Hausdorff, gostaríamos de mostrar que $X\times X\backslash R$ é aberto. Para qualquer ponto $(x,x')\in (X\times X)\backslash R$, $x$ e $x'$ são tais que $\pi(x)\neq \pi(x')$. Como $X/\sim$ é de Hausdorff, existem abertos $U_x$ e $U_{x'}$ disjuntos em $X/\sim$ que são vizinhanças abertas de $\pi(x)$ e de $\pi(x')$, respectivamente.% e tais que $U_x\cap U_x' = \emptyset$.

    Temos ainda que $\pi^{-1}(U_x)$ e $\pi^{-1}(U_{x'})$ são abertos, pois a topologia de $X/\sim$ é a topologia quociente, e o produto $U=\pi^{-1}(U_x)\times \pi^{-1}(U_{x'})$ é aberto de $X\times X$ na topologia produto. Além disso, $(x,x')\in U$. Afirmamos que $U\subset X\times X\backslash R$. De fato, se $U\cap R\neq \emptyset$, teríamos $(v_1,v_2)\in U\cap R$ tal que $\pi(v_1)=\pi(v_2)$, mas $v_1 \in \pi^{-1}(U_x)$ e $v_2\in \pi^{-1}(U_{x'})$, e desse modo $\pi(v_1) = \pi(v_2) \in U_x \cap U_{x'} = \varnothing$, absurdo. Portanto, para todo $(x,x')\in X\times X\backslash R$, é possível encontrar uma vizinhança aberta $U$ de $(x,x')$ contida em $X\times X\backslash R$; $R$ é fechado, como queríamos.\newline

    $(\Longleftarrow)$ Dado que $R$ é fechado, gostaríamos de encontrar vizinhanças disjuntas de $a,~b\in X/\sim$ quaisquer para concluir que $X/\sim$ é Hausdorff. Sabemos que existem $x,~y\in X$ tais que $\pi(x)=a$ e $\pi(y)=b$ pois a projeção $\pi$ é uma aplicação sobrejetora. Como $X$ é de Hausdorff, existem abertos disjuntos $U_x$ e $U_y$, vizinhanças de $x$ e de $y$, respectivamente. Além disso, uma vez que $R$ é fechado, $X\times X\backslash R$ é aberto e, portanto, $(U_x\times U_y)\cap((X\times X)\backslash R)$ é aberto na topologia produto.

    Sejam $p_1:X\times X\rightarrow X$ e $p_2:X\times X\rightarrow X$ definidos por $$p_1(x_1,x_2)=x_1,\qquad p_2(x_1,x_2)=x_2 \qquad\forall (x_1,x_2)\in X\times X.$$ Como a topologia produto em $X\times Y$ é gerada pela base dada pelos produtos de abertos $X$ e de $Y$, é possível concluir que $p_1$ e $p_2$ são aplicações abertas. Desse modo, $U_1=p_1((U_x\times U_y)\cap((X\times X)\backslash R))$ e $U_2=p_2((U_x\times U_y)\cap((X\times X)\backslash R))$ são abertos em $X$. Por fim, basta observar que os abertos $\pi(U_1)$ e $\pi(U_2)$ são tais que $\pi(U_1)\cap \pi(U_2)=\emptyset$ uma vez que se $v\in \pi(U_1)\cap\pi(U_2)$, teríamos $v=\pi(v_1)$ para algum $v_1\in U_1$ e $v=\pi(v_2)$ para algum $v_2\in U_2$, o que implicaria $v_1\sim v_2$, um absurdo pois, pela construção de $U_1$ e $U_2$, $(v_1,v_2)\not\in R$.  Também temos $a\in U_1$ e $b\in U_2$ pois, como $a\neq b$, $x\not\sim y$. Encontramos assim os dois abertos que separam $a$ e $b$, mostrando que $X/\sim$ é Hausdorff.
\end{dem}

% Comentários sobre a afirmação.
% \begin{titlemize}{Lista de consequências}
% 	\item \hyperref[consequencia1]{Consequência 1};\\ %'consequencia1' é o label onde o conceito Consequência 1 aparece
% \end{titlemize}
O espaço quociente também é essencial para realizar a colagem de espaços topológicos.
\subsection{\emph{Pushout} de espaços topológicos} %afirmação aqui significa teorema/proposição/colorário/lema

\label{pushout-de-espacos-topologicos-def}
\begin{titlemize}{Lista de dependências}
	\item \hyperref[topologia-quociente-def]{Espaços Quociente};\\ %'dependencia1' é o label onde o conceito Dependência 1 aparece (--à arrumar um padrão para referencias e labels--) 
    \item \hyperref[funcao-continua-em-topologia-quociente-prop]{Função contínua em topologia quociente}.
% quantas dependências forem necessárias.
\end{titlemize}

\begin{defi}
    Sejam $X,Y,Z$ espaços topológicos, e sejam $f:Z\rightarrow X$ e $g:Z\rightarrow Y$ funções contínuas. O \textbf{\emph{pushout} de $f$ e $g$} é o espaço quociente $X\sqcup_Z Y=X\sqcup Y/\sim$, onde $\sim$ é a menor relação de equivalência que contém $\{(f(z),g(z))\in X\times Y:z\in Z\}$. 

\end{defi}

\begin{prop}
    Sejam $X,Y,Z$ espaços topológicos. Além disso, sejam $f:Z\rightarrow X$ e $g:Z\rightarrow Y$ funções contínuas. Seja também $\pi:X\sqcup Y\rightarrow X\sqcup_Z Y$ a função projeção associada ao quociente. Então, o espaço $X\sqcup_Z Y$, juntamente com as funções contínuas definidas por:
    \begin{align*}
        i_X:X &\longrightarrow X\sqcup Y/\sim & i_Y:Y&\longrightarrow X\sqcup Y/\sim\\
        x&\longmapsto \pi(x) & y &\longmapsto \pi(y)
    \end{align*}
    
    forma um diagrama de \emph{pushout}. Ou seja, dadas quaisquer duas funções contínuas $h_X:X\rightarrow W$ e $h_Y:Y\rightarrow W$ que satisfaçam $h_X\circ f=h_Y\circ g$, existe uma única função contínua 
    $\phi:X\sqcup_Z Y\rightarrow W$ tal que 
    $$h_X=\phi\circ i_X \;\;\;\text{ e }\;\;\; h_Y=\phi\circ i_Y.$$ 
    Isso é ilustrado no diagrama seguinte:

% https://q.uiver.app/#q=WzAsNSxbMCwwLCJaIl0sWzAsMiwiWCJdLFsyLDAsIlkiXSxbMiwyLCJYXFxzcWN1cF9aIFkiXSxbMywzLCJXIl0sWzAsMSwiZiIsMl0sWzAsMiwiZyJdLFsxLDMsImlfWCJdLFsyLDMsImlfWSIsMl0sWzEsNCwiaF9YIiwyXSxbMiw0LCJoX1kiXSxbMyw0LCJcXGV4aXN0cyEgXFxwaGkiLDEseyJzdHlsZSI6eyJib2R5Ijp7Im5hbWUiOiJkYXNoZWQifX19XV0=
\[\begin{tikzcd}
	Z && Y \\
	\\
	X && {X\sqcup_Z Y} \\
	&&& W.
	\arrow["g", from=1-1, to=1-3]
	\arrow["f"', from=1-1, to=3-1]
	\arrow["{i_Y}"', from=1-3, to=3-3]
	\arrow["{h_Y}", from=1-3, to=4-4]
	\arrow["{i_X}", from=3-1, to=3-3]
	\arrow["{h_X}"', from=3-1, to=4-4]
	\arrow["{\exists! \phi}"{description}, dashed, from=3-3, to=4-4]
\end{tikzcd}\]
\end{prop}

\begin{dem}
    De acordo com a construção da topologia quociente e da topologia de união disjunta, as funções $i_X,i_Y$ são contínuas. Além disso, a função dada por 
    \begin{align*}
        h_X\sqcup h_Y:X\sqcup Y&\longrightarrow W\\
        a&\longmapsto h_X\sqcup h_Y(a)=\begin{cases}
         h_X(a) & \text{ if }a\in X\\
         h_Y(a) & \text{ if }a\in Y.
        \end{cases}
    \end{align*}

    também é contínua. Como $h_X\circ f=h_Y\circ g$, pela definição de \emph{pushout} de $f$ e $g$, a função $\phi:=(h_X\sqcup h_Y)\circ \pi^{-1}$ é bem-definida. Além disso, a função $\phi$ é contínua, pois a função $\phi\circ\pi=h_X\sqcup h_Y$ é contínua (pela proposição \ref{funcao-continua-em-topologia-quociente-prop}). Pela construção de $\phi$, temos $h_X=\phi\circ i_X$ e $h_Y=\phi\circ i_Y$, o que prova a existência de tal função.


    Finalmente, provamos que esta função é única: suponha que $\phi'$ seja outra função contínua que satisfaça $h_X=\phi'\circ i_X$ e $h_Y=\phi'\circ i_Y$. Então, temos $\phi'|_{i_X(X)}=\phi|_{i_X(X)}$ e $\phi'|_{i_Y(Y)}=\phi|_{i_Y(Y)}$. Como $i_X(X)\cup i_Y(Y)=X\sqcup_ZY$, concluímos que $\phi=\phi'$.
\end{dem}

%\begin{titlemize}{Lista de consequências}
	%\item %\hyperref[consequencia1]{Consequência 1};\\ %'consequencia1' é o label onde o conceito Consequência 1 aparece
%\end{titlemize}

\subsection{Colagem de n-célula} %afirmação aqui significa teorema/proposição/colorário/lema
\label{colagem-de-n-celula-def}
\begin{titlemize}{Lista de dependências}
	\item \hyperref[topologia-quociente-def]{Espaços Quociente};\\

    \item \hyperref[pushout-de-espacos-topologicos-def]{\emph{Pushout} de espaços topológicos}.%'dependencia1' é o label onde o conceito Dependência 1 aparece (--à arrumar um padrão para referencias e labels--) 

% quantas dependências forem necessárias.
\end{titlemize}

\begin{defi}
    Seja $X$ um espaço topológico, e sejam $f:\mathbb{S}^{n-1}\rightarrow X$ uma função contínua e $i:\mathbb{S}^{n-1}\hookrightarrow D^n$ uma inclusão, onde $n\ge 2$. O \textbf{espaço obtido de} $X$ \textbf{pela colagem de uma $n$-célula por meio da função} $f$ é o \emph{pushout} de $f$ e $i$, denotado por $X_f$ ou $D^n\cup_f X$.
\end{defi}

%\begin{titlemize}{Lista de consequências}
	%\item %\hyperref[consequencia1]{Consequência 1};\\ %'consequencia1' é o label onde o conceito Consequência 1 aparece
%\end{titlemize}

\subsection{Colagem de um disco com um ponto} %afirmação aqui significa teorema/proposição/colorário/lema
\label{colagem-de-um-disco-com-um-ponto-ex}
\begin{titlemize}{Lista de dependências}
	\item \hyperref[topologia-quociente-def]{Espaços Quociente};\\
    \item \hyperref[pushout-de-espacos-topologicos-def]{Pushout de espaços topológicos};\\
    \item \hyperref[colagem-de-n-celula-def]{Colagem de n-célula}%'dependencia1' é o label onde o conceito Dependência 1 aparece (--à arrumar um padrão para referencias e labels--) 
% quantas dependências forem necessárias.
\end{titlemize}

\begin{ex}

    Dado $n\ge 2$, sejam $N = (0,\ldots,0,1) \in \mathbb{S}^n$ e $S = -N$. A colagem $\{x\}_f=D^n\cup_f \{x\}$, em que $f:\mathbb{S}^{n-1}\rightarrow \{x\}$ é a função constante, é homeomorfa à esfera $\mathbb{S}^n$. 
\end{ex}

\begin{dem}
    Note que $\text{int}(D^n)\cong \mathbb{R}^n\cong \mathbb{S}^n\setminus\{N\}$. Seja $g_0:\text{int}(D^n)\rightarrow \mathbb{R}^n$ um homeomorfismo. Podemos estender $g_0$ na seguinte forma 
    \begin{align*}
        g:\{x\}_f&\longrightarrow \mathbb{S}^n\\
        p&\longmapsto g(p)=\begin{cases}
            g_0(p) &\text{ se }p\ne [x]\\
            N &\text{ se }p=[x].
        \end{cases}
    \end{align*}
    Essa função é bem-definida e bijetora. Agora, para mostrar que $g$ é um homeomorfismo, basta mostrar que $g$ é contínua e aberta.\\
    A função $g$ é contínua: A função $g$ é contínua se, e somente se, $g\circ \pi$ é contínua, onde $\pi:D^n\bigsqcup \{x\}\rightarrow \{x\}_f$ é a projeção associada ao quociente. A função $g\circ \pi$ é contínua, pois dado um aberto $U$ de $\mathbb{S}^n$, temos:
    \begin{itemize}
        \item se $N\notin U$, então $(g\circ\pi)^{-1}(U)=g_0^{-1}(U)$, que é aberto;
        \item se $N\in U$, então $(g\circ \pi)^{-1}(U)=g_0^{-1}(U\setminus\{N\})\cup \{x\}$, que é um aberto, pois $x$ é um ponto isolado.
    \end{itemize}
    Portanto, $g$ é contínua.\\
    A função $g$ é aberta: Seja $U$ um aberto de $\{x\}_f$. Teremos dois casos: 
    \begin{itemize}
        \item Se $x\notin U$, então $g(U)=g_0(U)$, que é um aberto em  $\mathbb{S}^n\setminus\{N\}$. Assim, ou $g_0(U)$ é aberto em $\mathbb{S}^n$, ou $g_0(U)\cup\{N\}$ é aberto em $\mathbb{S}^n$. Se $g_0(U)$ for aberto, então $g(U)$ será um aberto em $\mathbb{S}^n$. Se $g_0(U)\cup \{N\}$ for aberto, então $g_0(U)=(g_0(U)\cup\{N\})\setminus\{N\}$ é um aberto em $\mathbb{S}^n$. Em ambos os casos, $g(U)$ será um aberto em $\mathbb{S}^n$;
        \item Se $x\in U$, então $\pi^{-1} (U)$ é um aberto em $D^n\sqcup \{x\}$. Pela construção do quociente, temos $\partial D^n\subseteq\pi^{-1}(U)$, logo, para todo ponto $y\in \partial D^n$, existe um $r_y>0$ tal que 
        $$B_y:=\{z\in D^n: ||z-y||<r_y\}\subseteq \pi_1^{-1}(U).$$
        A coleção $\{B_y\}_{y\in \partial D^n}$ é uma cobertura aberta de $\partial D^n$. Como o bordo $\partial D^n$ é compacto, existem $y_1,...,y_k$ tal que 
        \[\partial D^n\subseteq B_{y_1}\cup...\cup B_{y_k}.\]
        Considere $r=\text{inf}\{r_{y_1},...,r_{y_k}\}$. Assim, o conjunto 
        \[B=\pi(\{y\in D^n: ||y||>(1-r)\}\cup\{x\})\subseteq U\]
        é um aberto em $\{x\}_f$, e $g(B)\subseteq g(U)$ corresponde a uma bola centrada em $N$ em $\mathbb{S}^n$. Note que $U\setminus \{x\}$ é um aberto, pois $x$ é um ponto fechado. Pelo item anterior, temos que o aberto
        \[g(U)=g(B)\cup g(U\setminus\{x\})\]
        é uma união de abertos, o que implica que $g(U)$ é um aberto em $\mathbb{S}^n$.
    \end{itemize} 
    Portanto, $g$ é aberta.
\end{dem}

    Para $n\ge 2$, a colagem $\{x\}_f=D^n\cup_f \{x\}$, em que $f:\mathbb{S}^{n-1}\rightarrow \{x\}$ é a função constante, é a esfera $\mathbb{S}^n$. Visto que $int(D^n)$ é homeomorfo a $\mathbb{R}^n$, e $\mathbb{R}^n$ é homeomorfo a $\mathbb{S}^n\setminus\{\text{polo norte}\}$, e que a colagem garante que $\{x\}_f\setminus\{\text{origem do disco }D^n \}$ é homeomorfo a $\mathbb{S}^n\setminus\{\text{polo sul}\}$, conclui-se que $\{x\}_f$ é homeomorfo à esfera $\mathbb{S}^n$.
\end{ex}

%\begin{titlemize}{Lista de consequências}
	%\item %\hyperref[consequencia1]{Consequência 1};\\ %'consequencia1' é o label onde o conceito Consequência 1 aparece
%\end{titlemize}

\subsection{Produto \emph{wedge} de espaços topológicos} %afirmação aqui significa teorema/proposição/colorário/lema
\label{produto-wedge-def}
\begin{titlemize}{Lista de dependências}
	\item \hyperref[topologia-quociente-def]{Espaços Quociente};\\ %'dependencia1' é o label onde o conceito Dependência 1 aparece (--à arrumar um padrão para referencias e labels--) 
\end{titlemize}

\begin{defi}
    Sejam $(X,x_0),(Y,y_0)$ espaços topológicos pontuados. O \textbf{produto \emph{wedge}} de $(X,x_0)$ e $(Y,y_0)$, denotado por $(X,x_0)\vee (Y,y_0)$, é o espaço quociente obtido da união disjunta $X\sqcup Y$ por meio da identificação de $x_0$ e $y_0$ a um único ponto. 
\end{defi}



%\begin{titlemize}{Lista de consequências}
	%\item %\hyperref[consequencia1]{Consequência 1};\\ %'consequencia1' é o label onde o conceito Consequência 1 aparece
%\end{titlemize}

%%% Local Variables:
%%% mode: LaTeX
%%% TeX-master: "../Alg.Top-Wiki"
%%% End:

