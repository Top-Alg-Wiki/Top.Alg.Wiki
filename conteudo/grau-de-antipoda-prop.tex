\subsection{Grau de antípoda} %afirmação aqui significa teorema/proposição/colorário/lema
\label{grau-de-antipoda-prop}
\begin{titlemize}{Lista de dependências}
    \item \hyperref[homologia-singular-def]{Homologia singular};\\
    \item \hyperref[homomorfismo-de-homologias-singulares-induzido-prop]{Homomorfismo de homologias singulares induzido};\\
    \item \hyperref[homologia-singular-de-S1-prop]{Homologia singular da circunferência};\\
    \item \hyperref[grupo-de-homologia-singular-de-n-esfera-prop]{Grupo de homologia singular de n-esfera};\\
    \item \hyperref[grau-de-funcoes-em-esferas-def]{Grau de funções em esferas};\\
    \item \hyperref[propriedades-de-grau-de-funções-prop]{Propriedade de grau de funções em esferas};\\
    \item \hyperref[grau-da-reflexao-prop]{Grau da reflexão}
\end{titlemize}

\begin{defi}
    Uma \textbf{antípoda} é uma função de $a:\mathbb{S}^n\rightarrow \mathbb{S}^n$ dada por $a(x)=-x$.
\end{defi}

\begin{lemma}
    A antípoda da esfera $\mathbb{S}^n$ tem grau $(-1)^{n+1}$
\end{lemma}

\begin{dem}
    Para cada $1\le i\le n+1$, seja $r_i: \mathbb{S}^n\rightarrow \mathbb{S}^n$ a reflexão da i-ésima coordenada. A antípoda fatora-se como a composição $a=r_1\circ ...\circ r_{n+1}$. Segue das propriedades de grau de funções em esferas que 
    \[deg(a)=deg(r_1)\cdot...\cdot deg(r_{n+1})=(-1)^{n+1}\]
\end{dem}

\begin{titlemize}{Lista de consequências}
    \item \hyperref[homomorfismo-de-homologias-singulares-induzido-prop]{Homomorfismo de homologias singulares induzido}.\\
	%\item \hyperref[]{}
\end{titlemize}
