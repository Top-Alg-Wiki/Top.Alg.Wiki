%---------------------------------------------------------------------------------------------------------------------!Draft!-----------------------------------------------------------------------------------------------------------------
\subsection{Soma conexa de variedades}
\label{soma-conexa-def}
\begin{titlemize}{Lista de dependências}
	\item \hyperref[variedade-def]{Variedades}
\end{titlemize}
\begin{defi}[Soma conexa]
	Fixemos $m\geq 0$, e sejam $(M,p_0)$ e $(N,q_0)$ duas $m$-variedades topológicas pontuadas. A \textbf{soma conexa de $(M,p_0)$ e $(N,q_0)$}, denotada por $(M,p_0)\#(N,q_0)$, é definida como a variedade $(M\setminus B) \sqcup (N\setminus B')/\sim$, onde $B$ e $B'$ são vizinhanças abertas de $p_0$ e $q_0$, respectivamente, ambas homeomorfas a uma bola de $\mathbb{R}^m$, e $\sim$ identifica $\partial B$ e $\partial B'$ via um homeomorfismo de $\overline{B}$ sobre $\overline{B}'$.
\end{defi}

Em diversos exemplos comuns, o espaço resultante independe (a menos de homeomorfismo) da escolha dos pontos, e nesse caso escrevemos simplesmente $M\# N$.

\begin{titlemize}{Lista de consequências}
	\item \hyperref[forma-normal-caso-a-thm]{Caso A do Teorema de Forma Normal};\\
    \item \hyperref[forma-normal-caso-b-thm]{Caso B do Teorema de Forma Normal}
\end{titlemize}

%[Bianca]: é mais fácil criar a lista de dependências do que a de consequências.