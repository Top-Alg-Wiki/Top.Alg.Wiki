%---------------------------------------------------------------------------------------------------------------------!Draft!-----------------------------------------------------------------------------------------------------------------
\subsection{Suspensão e cone duplo}
\label{suspensao-cone-duplo-prop}
\begin{titlemize}{Lista de dependências}
	\item \hyperref[cone-def]{Cone};\\
	\item \hyperref[suspensao-def]{Suspensão}.
\end{titlemize}
Provaremos que a suspensão sobre um espaço topológico é homeomorfa a um cone duplo, confirmando as intuições sobre os objetos.
\begin{proposition}[A construção de suspensão coincide com a de cone duplo]
	Seja $X$ um espaço topológico, e sobre $C(X) \amalg C(X)$ considere a relação de equivalência\[
    ([(x,s)],i) \sim ([(y,t)],j) \Leftrightarrow ([(x,s)],i) = ([(y,t)],j)\text{ ou }(x=y\text{ e }s=t=0), \forall x,y \in X, \forall s,t \in I, \forall i,j \in \{-1,1\}
    .\]%usamos \sim para todas as relações de equivalência (da suspensão, as dos cones e a que definimos aqui)?
    Defina
    \begin{align*}
        \psi: S(X) &\rightarrow (C(X) \amalg C(X))/ \sim,\\
                [(x,t)] &\mapsto [((x,|t|),\text{sgn}(t))]
    \end{align*}
    onde $\text{sgn}(t) = -1$ caso $t < 0$ e $\text{sgn}(t) = 1$ caso contrário. Então $\psi$ é homeomorfismo.

	Ou seja, a construção de suspensão sobre um espaço topológico coincide com a colagem de dois cones sobre o mesmo espaço topológico, quando identificamos as suas respectivas bases.
\end{thm}


\begin{titlemize}{Lista de consequências}
	%\item \hyperref[consequencia1]{Consequência 1};\\ %'consequencia1' é o label onde o conceito Consequência 1 aparece
	%\item \hyperref[]{}
\end{titlemize}

