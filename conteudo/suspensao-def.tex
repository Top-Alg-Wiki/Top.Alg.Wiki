%---------------------------------------------------------------------------------------------------------------------!Draft!-----------------------------------------------------------------------------------------------------------------
\subsection{Suspensão sobre um Espaço Topológico}
\label{suspensao-def}
\begin{titlemize}{Lista de dependências}
	\item \hyperref[topologia-quociente-def]{Topologia Quociente}.
\end{titlemize}
\begin{def}[Suspensão]
	Dado um espaço topológico $X$, definimos a \bf{suspensão sobre X} como o espaço quociente $X\times [-1,1]/\sim$, onde $[-1,1]$ é munido da topologia usual de subespaço de $\mathbb{R}$, e para todos $(x,s),(y,t) \in X\times [-1,1]$,\[
    (x,s)\sim(y,t) \\Leftrightarrow (x,s)=(y,t)\text{ ou }s=t=1\text{ ou }s=t=-1.
    \]
\end{def}

É simples ver que a relação definida acima é de equivalência.

Tal construção é semelhante a de \item \hyperref[cone-def]{cone sobre um espaço topológico}.

\begin{titlemize}{Lista de consequências}
	%\item \hyperref[consequencia1]{Consequência 1};\\ %'consequencia1' é o label onde o conceito Consequência 1 aparece
	%\item \hyperref[]{}
\end{titlemize}