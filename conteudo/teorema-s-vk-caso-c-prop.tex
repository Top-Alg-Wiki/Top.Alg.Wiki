\subsection{Caso C de Teorema de Seifert-Van Kampen} %afirmação aqui significa teorema/proposição/colorário/lema
\label{teorema-s-vk-caso-c-prop}
\begin{titlemize}{Lista de dependências}
    \item \hyperref[pushout-de-grupos-prop]{\emph{Pushout} de grupos};\\
    \item \hyperref[grupo-fundamental]{Grupo fundamental};\\
    \item \hyperref[teorema-s-vk-caso-b-prop]{Caso B de Teorema de Seifert-Van Kampen}.
% quantas dependências forem necessárias.
\end{titlemize}
Note que, se $\pi_1(V,x)=\{e\}$, então a condição $j_{U_*}\circ i_{U_*}=j_{V_*}\circ i_{V_*}= e$ é equivalente a $\text{Im}(i_{U_*})\subseteq \text{Ker}(j_{U_*})$. A propriedade universal do \emph{pushout} então reduz a condição: para todo homomorfismo de grupos $\phi:\pi_1(U,x)\rightarrow H$ satisfazendo $\text{Im}(i_{U_*})\subseteq \text{Ker}(\phi)$, existe um único homomorfismo $\psi:\pi_1(X,x)\rightarrow H$ tal que $\psi\circ j_{U_*}=\phi$.

Essa condição decorre do teorema do homomorfismo de grupos, quando $\text{Im}(i_{U_*})$ é um subgrupo normal de $\pi_1(U,x)$. No entanto, em geral, $\text{Im}(i_{U_*})$ não é um subgrupo normal. Para podermos aplicar o teorema do homomorfismo de grupos, podemos "normalizar" $\text{Im}(i_{U_*})$. 

\begin{prop}
    Se $\pi_1(V,x)=\{e\}$, então $\pi_1(X,x)\cong \pi_1(U,x)/\ \overline{\textnormal{Im}(i_{U_*})}$.
\end{prop}

\begin{dem}
    Essa proposição é uma consequência direta de um corolário apresentado na subseção \ref{pushout-de-grupos-prop}.
    %Basta provar que $\pi_1(U,x)/\overline{\text{Im}(i_{U_*})},\pi,\{e\}\hookrightarrow \pi_1(U,x)/\overline{\text{Im}(i_{U_*})} )$ é o pushout de $(\pi_1(U\cap V,x),i_{U_*},i_{V_*})$, onde $\pi:\pi_1(U,x)\rightarrow \pi_1(U,x)/\overline{\text{Im}(i_{U_*})}$ é a projeção canônica do quociente. Note que, todo homomorfismo de grupos $\phi:\pi_1(U,x)\rightarrow H$ satisfazendo $\text{Im}(i_{U_*})\subseteq \text{Ker}(\phi)$ também satisfaz $\overline{\text{Im}(i_{U_*})}\subseteq \text{Ker}(\phi)$, pela proposição \ref{fecho normal-def} e pela normalidade de $\text{Ker}(\phi)$. Assim, pelo teorema do homomorfismo, existe um único homomorfismo $\psi:\pi_1(U,x)/\overline{\text{Im}(i_{U_*})}\rightarrow H$ tal que $\psi\circ j_{U_*}=\phi$.
\end{dem}

Agora, podemos analisar o grupo fundamental do espaço obtido pela colagem de uma 2-célula.

\begin{corol}
    Seja $X$ um espaço Hausdorff conexo por caminhos com $x\in X$. Seja $f:\mathbb{S}^{1}\rightarrow X$ uma função contínua e $i:\mathbb{S}^{1}\hookrightarrow D^2$ uma inclusão. Denotamos o espaço obtido de $X$ pela colagem de uma $n$-célula por meio da função $f$ por $X_f$. Então, temos que $\pi_1(X_f, p)\cong \pi_1(X,h^{-1}(p))/\overline{\text{Im}(f_*)}$ para todo ponto $p\in h(\text{int}(D^2))\cap X_f\setminus\{h(0)\}$, onde $\pi:X\rightarrow X_f$, $h:D^2\rightarrow X_f$ são as funções associadas ao \emph{pushout}.
\end{corol}
\begin{dem}
     Consideramos $V=h(\text{int}(D^2))$ e $U=X_f\setminus \{h(0)\}$. Como discutido em \ref{sequencia-exata-da-colagem-prop}, temos que $V$ é homeomorfo a $\text{int}(D^2)$, $U$ é homotopicamente equivalente a $X$ e $U\cap V$ é homotopicamente equivalente a $\mathbb{S}^{1}$. Dessa forma, temos $\pi_1(U,p)=\{e\}$ para todo $p\in U\cap V$. Pela proposição anterior, temos que $\pi_1(X_f,p)\cong \pi_1(U, p)/\overline{\text{Im}(i_{U_*})}$, para todo $p\in U\cap V$. Pelo diagrama comutativo seguinte 
     % https://q.uiver.app/#q=WzAsNixbMCwwLCJcXHBpXzEoVVxcY2FwIFYscCkiXSxbMCwxLCJcXHBpXzEoXFxtYXRoYmJ7U31eMSkiXSxbMSwwLCJcXHBpXzEoVSxwKSJdLFsxLDEsIlxccGlfMShYLGheey0xfShwKSkiXSxbMiwwLCJcXHBpXzEoWF9mLHApIl0sWzIsMSwiXFxwaV8xKFhfZixwKSJdLFswLDIsImlfe1VfKn0iXSxbMiw0LCJqX3tVXyp9Il0sWzEsMywiZl8qIiwyXSxbMyw1LCJcXHBpXyoiLDJdLFs0LDUsIj0iLDFdLFswLDEsIlxcY29uZyIsMV0sWzIsMywiXFxjb25nIiwxXV0=
\[\begin{tikzcd}
	{\pi_1(U\cap V,p)} & {\pi_1(U,p)} & {\pi_1(X_f,p)} \\
	{\pi_1(\mathbb{S}^1)} & {\pi_1(X,h^{-1}(p))} & {\pi_1(X_f,p)},
	\arrow["{i_{U_*}}", from=1-1, to=1-2]
	\arrow["\cong"{description}, from=1-1, to=2-1]
	\arrow["{j_{U_*}}", from=1-2, to=1-3]
	\arrow["\cong"{description}, from=1-2, to=2-2]
	\arrow["{=}"{description}, from=1-3, to=2-3]
	\arrow["{f_*}"', from=2-1, to=2-2]
	\arrow["{\pi_*}"', from=2-2, to=2-3]
\end{tikzcd}\]
     temos que $\pi_1(X_f, p)\cong \pi_1(X,h^{-1}(p))/\overline{\text{Im}(f_*)}$, para todo $p\in U\cap V$.
\end{dem}

Como mencionado no final de \ref{teorema-s-vk-caso-b-prop}, o ponto base $h^{-1}(p)$ não é relevante.
