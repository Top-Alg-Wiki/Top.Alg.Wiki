\section{Homologia}
\label{homologia}

\begin{titlemize}{Lista de Dependências}
    \item \hyperref[homotopia]{Homotopia};\\ %assunto1 é o label onde o Assunto 1 aparece
    \item \hyperref[grupos-livres]{Grupos Livres};\\
    \item \hyperref[simplexo-def]{Simplexos};\\
\end{titlemize}

Em topologia algébrica, a homologia é a sequência de grupos de homologia associada a um espaço topológico. Esses grupos capturam de forma algébrica a ideia dos "buracos" em diferentes dimensões no espaço. Assim, a homologia é uma ferramenta fundamental para distinguir e classificar espaços topológicos.

\subsection{Complexo de cadeias}
\label{complexo-de-cadeias-def}
\begin{titlemize}{Lista de dependências}
	\item %\hyperref[homologia-simplicial-def]{Homologia Simplicial};\\ %'dependencia1' é o label onde o conceito Dependência 1 aparece (--à arrumar um padrão para referencias e labels--) 
% quantas dependências forem necessárias.
\end{titlemize}
\begin{defi}[Complexo de cadeias]
	Um \textbf{Complexo de cadeias} é uma sequência $C_{-1}=0,C_0,C_1, C_2,...$ de grupos abelianos acompanhada de homomorfismos $d_n:C_n\rightarrow C_{n-1}$ para cada $n\ge 0$, tais que $d_{n}\circ d_{n+1}=0$. Denotamos esse complexo de cadeias por $C_{\bullet}$, e os homomorfismos $d_n$ são chamados de \textbf{diferenciais} de $C_\bullet$. A n-ésima \textbf{homologia} de $C_\bullet$ é definida por
    \[H_n(C_\bullet):=\frac{\text{Ker}(d_n)}{\text{Im}(d_{n+1})}\]
    Usualmente, $\text{Ker}(d_n)$ é chamado de grupo abeliano de $n$-ciclo em $C_\bullet$ e é denotado por $Z_n(C_\bullet)$. Por outro lado, $\text{Im}(d_{n+1})$ é chamada de grupo abeliano de $n$-bordos em $C_\bullet$ e é denotada por $B_n(C_\bullet)$.
\end{defi}

A homologia é a medida não numérica de quão diferentes $Z_n(C_\bullet)$ e $B_n(C_\bullet)$ são.

\begin{titlemize}{Lista de consequências}
    \item \hyperref[aplicacao-de-cadeias-def]{Aplicação de cadeias};\\
    \item \hyperref[homotopia-de-cadeias-def]{Homotopia de cadeia};\\
    \item \hyperref[homomorfismo-induzido-de-cadeias-prop]{Homomorfismo induzido de cadeias};\\
    \item \hyperref[equivalencia-de-homotopia-de-cadeias-def]{Equivalência de homotopia de cadeias};\\
    \item \hyperlink{homologia-simplicial-def}{Homologia simplicial};\\
    \item \hyperref[homologia-singular-def]{Homologia singular};\\
    \item \hyperref[homomorfismo-de-homologias-singulares-induzido-prop]{Homomorfismo de homologias singulares induzido}.
\end{titlemize}

\subsection{Aplicação de cadeias}
\label{aplicacao-de-cadeias-def}
\begin{titlemize}{Lista de dependências}
	\item \hyperref[complexo-de-cadeias-def]{Complexo de cadeias}.\\ %'dependencia1' é o label onde o conceito Dependência 1 aparece (--à arrumar um padrão para referencias e labels--) 
% quantas dependências forem necessárias.
\end{titlemize}

\begin{defi}
    Uma \textbf{aplicação de cadeias} $f_\bullet:C_\bullet\rightarrow D_\bullet$ é uma sequência de homomorfismos $f_n:C_n\rightarrow D_n$ tal que $f_n\circ d_{n+1}^C=d_{n+1}^D\circ f_{n+1}$.
\end{defi}

\begin{titlemize}{Lista de consequências}
    \item \hyperref[homotopia-de-cadeias-def]{Homotopia de cadeia;}\\
    \item \hyperref[homomorfismo-induzido-de-cadeias-prop]{Homomorfismo induzido de cadeias};\\
    \item \hyperref[equivalencia-de-homotopia-de-cadeias-def]{Equivalência de homotopia de cadeias};\\
    \item \hyperref[homomorfismo-de-homologias-singulares-induzido-prop]{Homomorfismo de homologias singulares induzido}.
\end{titlemize}

\subsection{Homotopia de cadeias}
\label{homotopia-de-cadeias-def}
\begin{titlemize}{Lista de dependências}
	\item \hyperref[complexo-de-cadeias-def]{Complexo de cadeias};\\ %'dependencia1' é o label onde o conceito Dependência 1 aparece (--à arrumar um padrão para referencias e labels--) 
% quantas dependências forem necessárias.
    \item \hyperref[aplicacao-de-cadeias-def]{Aplicação de cadeias}.
\end{titlemize}

\begin{defi}
    Sejam $f_\bullet, g_\bullet:C_\bullet\rightarrow D_\bullet$ duas aplicações de cadeias. Uma \textbf{homotopia de cadeias} $h:f_\bullet\Rightarrow g_\bullet$ é uma sequência de homomorfismos $h_n:C_n\rightarrow D_{n+1}$, indexada por $n\ge -1$, tal que 
    \[g_n-f_n=d_{n+1}^D\circ h_n+ h_{n-1}\circ d_n^C:C_n\rightarrow D_n.\]
    Nessa caso, diremos que $f_\bullet$ e $g_\bullet$ são \textbf{homotópica de cadeias}, e denotaremos por $f_\bullet\simeq g_\bullet$.
\end{defi}

\begin{titlemize}{Lista de consequências}
    \item \hyperref[homomorfismo-induzido-de-cadeias-prop]{Homomorfismo induzido de cadeias};\\
    \item \hyperref[equivalencia-de-homotopia-de-cadeias-def]{Equivalência de homotopia de cadeias};\\
    \item \hyperref[homomorfismo-de-homologias-singulares-induzido-prop]{Homomorfismo de homologias singulares induzido}.\\
\end{titlemize}

\subsection{Homomorfismo induzido de cadeias} %afirmação aqui significa teorema/proposição/colorário/lema
\label{homomorfismo-induzido-de-cadeias-prop}
\begin{titlemize}{Lista de dependências}
	\item \hyperref[complexo-de-cadeias-def]{Complexo de cadeias};\\ 
    \item \hyperref[aplicacao-de-cadeias-def]{Aplicação de cadeias};\\
    \item \hyperref[homotopia-de-cadeias-def]{Homotopia de cadeia}.
\end{titlemize}
Assim como uma função contínua entre espaços topológicos induz um homomorfismo entre os grupos fundamentais associados, uma aplicação de cadeias induz um homomorfismo entre os grupos de homologias correspondentes.
\begin{lemma}%af(afirmação)/prop(proposição)/corol(corolário)/lemma(lema)/outros ambientes devem ser definidos no preambulo de Alg.Top-Wiki.tex 
	Uma aplicação de cadeias $f_\bullet: C_\bullet\rightarrow D_\bullet$ induz um homomorfismo 
    \begin{align*}
        f_*:H_n(C_\bullet)&\longrightarrow H_n(D_\bullet)\\
        [x]&\longmapsto [f_n(x)].
    \end{align*}
    Para todo $n\ge 0$. Além disso, se $f_\bullet$ e $g_\bullet$ são homotópicas de cadeias, então $f_*=g_*$.
\end{lemma}

\begin{proof}
    Vamos checar que $f_*$ é bem definido.
    \begin{itemize}
        \item Primeiramente, mostramos que $[f_n(x)]$ está dentro do codomínio. Seja $[x]\in H_n(C_\bullet)=\frac{Z_n(C_\bullet)}{B_n(C_\bullet)}$ representado por um $x\in Z_n(C_\bullet)$. Consideramos dois casos:\\
        Caso 1: $n=0$. Nesse caso, temos $Z_0(C_\bullet)=C_0$ e $Z_0(D_\bullet)=D_0$. Como $f_0(Z_0(C_\bullet))\subseteq Z_0(D_\bullet)$, temos que $f_0(x)$ é um ciclo, ou seja, $f_0(x)$ representa uma classe de homologia em $H_0(D_\bullet)$.\\
        Caso 2: $n\ge 1$. Como $x$ é um ciclo em $C_n$, temos:
        \[d_n^D\circ f_n(x)=f_{n-1}\circ d_n^C(x)=f_{n-1}(0)=0,\]
        ou seja, o elemento $f_n(x)\in D_n$ é um ciclo. Portanto $f_n(x)$ representa uma classe de homologia em $H_n(D_\bullet)$.
        \item Agora, suponha que $[x]=[y]$, ou seja $x-y\in B_n(C_\bullet)$. Logo, existe um $z\in C_{n+1}$ tal que $x-y=d_{n+1}^D(z)$. Então, temos: 
        \[f_n(x)-f_n(y)=f_n(d_{n+1}^C(z))=d_{n+1}^D (f_{n+1}(z))\]
        é um bordo. Portanto $[f_n(x)]=[f_n(y)]\in H_n(D_\bullet).$
    \end{itemize}
    Como $f_n$ são homomorfismos, o mapa $f_*$ também é um homomorfismo. 

    Agora, suponha que $h_\bullet: f_\bullet \Rightarrow g_\bullet$ é uma homotopia de cadeias. Seja $x\in Z_n(C_\bullet)$. Então, temos 
    \[g_n(x)-f_n(x)=d_{n+1}^D(h_n(x))+h_{n-1}(d_n^C(x)).\]
    Mas $x$ é um ciclo, logo $g_n(x)-f_n(x)=d_n(x)=0$. Isso mostra que $g_n(x)-f_n(x)$ é um bordo, portanto, $[g_n(x)]=[f_n(x)]$.
\end{proof}

De acordo com a construção do homomorfismo induzido, é fácil observar as seguintes propriedades.

\begin{corol}
    \begin{enumerate}
        \item Se $f_\bullet: C_\bullet\rightarrow D_\bullet$ e $g_\bullet:D_\bullet\rightarrow E_\bullet$ são aplicações de cadeias, então 
        \[(g_\bullet\circ f_\bullet)_*=g_*\circ f_*.\]
        \item $(id_{C_\bullet})_*=id_{H_n(C_\bullet)}$.
    \end{enumerate}

\end{corol}

\begin{titlemize}{Lista de consequências}
    \item \hyperref[equivalencia-de-homotopia-de-cadeias-def]{Equivalência de homotopia de cadeias};\\
    \item \hyperref[homomorfismo-de-homologias-singulares-induzido-prop]{Homomorfismo de homologias singulares induzido}.
	%\item \hyperref[consequencia1]{Consequência 1};\\ %'consequencia1' é o label onde o conceito Consequência 1 aparece
\end{titlemize}

\subsection{Equivalência de homotopia de cadeias} %afirmação aqui significa teorema/proposição/colorário/lema
\label{equivalencia-de-homotopia-de-cadeias-def}
\begin{titlemize}{Lista de dependências}
	\item \hyperref[complexo-de-cadeias-def]{Complexo de cadeias};\\ 
    \item \hyperref[aplicacao-de-cadeias-def]{Aplicação de cadeias};\\
    \item \hyperref[homotopia-de-cadeias-def]{Homotopia de cadeia};\\
    \item \hyperref[homomorfismo-induzido-de-cadeias-prop]{Homomorfismo induzido de cadeias}.
\end{titlemize}

\begin{defi}
    Uma aplicação de cadeias $f_\bullet:C\bullet\rightarrow D_\bullet$ é uma \textbf{equivalência de homotopia de cadeias} se existem uma aplicação de cadeias $g_\bullet:D_\bullet\rightarrow C_\bullet$ e homotopias de cadeias $f_\bullet\circ g_\bullet\simeq id_{D_\bullet}$ e $g_\bullet\circ f_\bullet \simeq id_{C_\bullet}$.
\end{defi}

Uma consequência imediata de \ref{homomorfismo-induzido-de-cadeias-prop} é: 

\begin{lemma}
    Se $f_\bullet:C_\bullet\rightarrow D_\bullet$ é uma equivalência de homotopia de cadeia, então $f_*:H_n(C_\bullet)\rightarrow H_n(D_\bullet)$ é um isomorfismo para todo $n\ge 0$.
\end{lemma}

\begin{proof}
    Pelo Lema \ref{homomorfismo-induzido-de-cadeias-prop}, temos
    \[f_*\circ g_*=(f_\bullet\circ g_\bullet)_*=(id_{D_\bullet})_*=id_{H_n(D_\bullet)},\]
    e vice-versa. 
\end{proof}

\begin{titlemize}{Lista de consequências}
    \item \hyperref[homomorfismo-de-homologias-singulares-induzido-prop]{Homomorfismo de homologias singulares induzido}.\\
	%\item \hyperref[]{}
\end{titlemize}

\subsection{Sequência Exata} %afirmação aqui significa teorema/proposição/colorário/lema
\label{sequencia-exata-def}
\begin{titlemize}{Lista de dependências}
	\item \hyperref[complexo-de-cadeias-def]{Complexo de cadeias};\\ 
    \item \hyperref[aplicacao-de-cadeias-def]{Aplicação de cadeias}.
\end{titlemize}

\begin{defi}
    Seja 
    \[...\rightarrow A_0\xrightarrow{f_0}A_1\xrightarrow{f_1} A_2\rightarrow ...\]
    uma sequência de homomorfismos de grupos. Diremos que essa sequência é \textbf{exata} se para cada $i$, $\text{Im}(f_i)=\text{Ker}(f_{i+1})$
\end{defi}

\begin{ex}
    A sequência de homomorfismos de grupos da forma 
    \[0\rightarrow A\xrightarrow{f}B\xrightarrow{g}C\rightarrow0\]
    é exata se, e somente se, as seguintes condições são satisfeitas
    \begin{enumerate}
        \item $f$ é um injetor (isto é, $\text{Ker}(f)=\{0\}$),
        \item $g$ é um sobrejetor (isto é, $\text{Im}(g)=\text{Ker}(C\rightarrow 0)$=$C$),
        \item $\text{Ker}(g)=\text{Im}(f)$. 
    \end{enumerate}
    Essa sequência é chamada de \textbf{sequência exata curta}.
\end{ex}

\begin{defi}
    Sejam $\mathcal{A},\mathcal{B},\mathcal{C}$ complexos de cadeias, e sejam $f= (f_n):\mathcal{A}\rightarrow \mathcal{B}$ e $g=(g_n):\mathcal{B}\rightarrow \mathcal{C}$ Aplicações de cadeias. A sequência 
    \[0\rightarrow \mathcal{A}\xrightarrow{f} \mathcal{B}\xrightarrow{g} \mathcal{C}\rightarrow 0\]
    é \textbf{exata} se, e somente se, para cada $n$, a sequência
    \[0\rightarrow A_n\xrightarrow{f_n} B_n\xrightarrow{g_n}C_n\rightarrow 0\]
    é exata.
\end{defi}

%\begin{titlemize}{Lista de consequências}
    %\item %\hyperref[homomorfismo-de-homologias-singulares-induzido-prop]{Homomorfismo de homologias singulares induzido}.\\
	%\item \hyperref[]{}
%\end{titlemize}

\subsection{Homomorfismo Conectante} %afirmação aqui significa teorema/proposição/colorário/lema
\label{homomorfismo-conectante-def}
\begin{titlemize}{Lista de dependências}
	\item \hyperref[complexo-de-cadeias-def]{Complexo de cadeias};\\ 
    \item \hyperref[aplicacao-de-cadeias-def]{Aplicação de cadeias};\\
    \item \hyperref[sequencia-exata-def]{Sequência exata}.
\end{titlemize}

    Sejam $\mathcal{A},\mathcal{B},\mathcal{C}$ complexos de cadeias, e sejam $\phi= (\phi_n):\mathcal{A}\rightarrow \mathcal{B}$ e $\psi=(\psi_n):\mathcal{B}\rightarrow \mathcal{C}$ Aplicações de cadeias. Supõe que a sequência 
    \[0\rightarrow \mathcal{A}\xrightarrow{\phi} \mathcal{B}\xrightarrow{\psi} \mathcal{C}\rightarrow 0\]
    é exata. 

    Construiremos uma função $\tilde{\delta}:Z_n(\mathcal{C})\rightarrow Z_{n-1}(\mathcal{A})$. Seja $z\in Z_n(\mathcal{C})$. Como $\psi_n$ é sobrejetora, existe $b\in B_n$ tal que $\psi_n(b)=z$. Como $\psi$ é uma aplicação de cadeias, temos
    \[\psi_{n-1}(\partial b)=\partial\psi_n(b)=\partial z=0.\]
    Como a sequência é exata, existe $a\in A_{n-2}$ tal que $\phi_{n-1}(a)=\partial b$. Como $\phi$ é uma aplicação de cadeias, temos
    \[\phi_{n-2}(\partial a)=\partial \phi_{n-1}(a)=\partial\partial b=0.\]
    Como $\phi_{n-2}$ é injetor, $\partial a=0$, e portanto $a\in Z_{n-1}(\mathcal{A})$. Dessa forma, podemos definir uma função 
    \begin{align*}
        \tilde{\delta_n}:Z_n(\mathcal{C})&\longrightarrow Z_{n-1}(\mathcal{A})\\
        z&\longmapsto a.
    \end{align*}

    Agora, mostramos que $\tilde{\delta_n}$ induz um homomorfismo $\delta_n:H_n(\mathcal{C})\rightarrow H_{n-1}(\mathcal{A})$, ou seja, provamos que: se $z_1,z_2$ em $Z_n(\mathcal{C})$ são ciclos tais que $z_1-z_2=\partial c$ para algum $c\in C_{n+1}$, então $\tilde{\delta_n}(z_1-z_2)= \partial a$, para algum $a\in A_n$. 

    Denotamos $a_1:=\tilde{\delta_n} (z_1)$ e $a_2:=\tilde{\delta_n}(z_2)$. Por construção $a_1$ e $a_2$ são tais que $\phi_{n-1}(a_1)=\partial b_1$ e $\phi_{n-1}(a_2)=\partial b_2$, onde $b_1$ e $b_2$ são elementos de $B_n$ que verificam $\psi(b_1)=z_1$ e $\psi(b_2)=z_2$. Como $\psi_{n+1}$ é sobrejetor, existe $b\in B_{n+1}$ tal que $\psi_{n+1}(b)=c$. Então, 
    \[\psi_n(\partial b)=\partial \psi_{n+1}(b)=\partial c=z_1-z_2.\]
    Logo, $b_1-b_2-\partial b\in \text{Ker}(\psi_n)=\text{Im}(\phi_n)$. Por conseguinte, existe $a\in A_n$ tal que $\phi_n(a)=b_1-b_2-\partial b$. Pela definição de aplicação de cadeias 
    \begin{align*}
        \phi_{n-1}(\partial a)&=\partial\phi_n(a)=\partial(b_1-b_2-\partial b)=\partial b_1-\partial b_2-\partial\partial b\\
        &=\phi_{n-1}(a_1)-\phi_{n-1}(a_2)=\phi_{n-1}(a_1-a_2).
    \end{align*}
    Como $\phi_{n-1}$ é injetor, $a_1-a_2=\partial a$. como queríamos.

    Portanto, a função $\tilde{\delta_n}:Z_n(\mathcal{C})\rightarrow Z_{n-1}(\mathcal{A})$ induz, por passagem ao quociente, um homomorfismos $\delta_n:H_n(\mathcal{C})\rightarrow H_{n-1}(\mathcal{A})$ para cada $n\ge 0$, dado por 
    \[\delta_n (z+B_n(\mathcal{C}))=\tilde{\delta_n}(z)+B_{n-1}(\mathcal{A}).\]
    \begin{defi}
        O homomorfismo $\delta_n:H_n(\mathcal{C})\rightarrow H_{n-1}(\mathcal{A})$ é chamado \textbf{homomorfismo conectante}. O índice $n$ será omitido quando não houve risco de confusão.
    \end{defi}
\begin{titlemize}{Lista de consequências}
    \item \hyperref[sequencia-exata-longa-induzida-prop]{Sequência exata longa induzida}.\\
	%\item \hyperref[]{}
\end{titlemize}

\subsection{Sequência exata longa induzida} %afirmação aqui significa teorema/proposição/colorário/lema
\label{sequencia-exata-longa-induzida-prop}
\begin{titlemize}{Lista de dependências}
	\item \hyperref[complexo-de-cadeias-def]{Complexo de cadeias};\\ 
    \item \hyperref[aplicacao-de-cadeias-def]{Aplicação de cadeias};\\
    \item \hyperref[homomorfismo-induzido-de-cadeias-prop]{Homomorfismo induzido de cadeias};\\
    \item \hyperref[sequencia-exata-def]{Sequência exata};\\
    \item \hyperref[homomorfismo-conectante-def]{Homomorfismo conectante}.
\end{titlemize}

\begin{thm}
    Seja 
    \[0\rightarrow \mathcal{A}\xrightarrow{\phi} \mathcal{B}\xrightarrow{\psi} \mathcal{C}\rightarrow 0\]
    uma sequência exata de complexos. Então, a sequência 
    \[...\rightarrow H_n(\mathcal{A})\xrightarrow{\phi_*}H_n(\mathcal{B})\xrightarrow{\psi_*} H_n (\mathcal{C})\xrightarrow{\delta}(H_{n-1}(\mathcal{A}))\rightarrow...\]
    é exata.
\end{thm}

\begin{dem}
    Verificamos a exatidão em 3 passos:

    Passo 1: $\text{Im}(\phi_*)=\text{Ker}(\psi_*)$

    Por hipótese, $\psi\circ \phi=0$. Logo $\psi_*\circ \phi_*=0$, ou seja $\text{Im}(\phi_*)\subseteq \text{Ker}(\psi_*).$ Para provar a outra inclusão, seja $b+B_n(\mathcal{B})\in \text{Ker} (\psi_*)\subseteq H_n (\mathcal{B})$. Então, $\psi(b)\in B_n(\mathcal{C})$, ou seja, existe $c\in C_{n+1}$ tal que $\partial c=\psi(b)$. Como $\psi$ é sobrejetora, existe $b^+\in B_{n+1}$ tal que $\psi(b^+)=c$. Então, $b-\partial b^+\in B_n$ e temos: 
    \[\psi(b-\partial b^+)=\psi(b)-\psi(\partial b^+)=\partial c-\partial \psi(b^+)=\partial(c-\psi(b^+))=0.\]
    Isso mostra que $b-\partial b^+ \in \text{Ker}(\psi)=\text{Im}(\phi)$, ou seja, existe $a\in A_n$ tal que $\phi(a)=b-\partial b^+$, e temos: 
    \[\phi(\partial a)=\partial\phi (a)=\partial (b-\partial b^+)=\partial b=0.\]
    Como $\phi$ é injetora, $\partial a=0$, ou seja, $a\in Z_n(\mathcal{C})$. Além disso, 
    \[\phi_*(a+B_n(\mathcal{A}))=\phi(a)+B_n(\mathcal{B})=b-\partial b^++B_n(\mathcal{B})=b+B_n(\mathcal{B}).\]
    Isso prova a inclusão $\text{Ker}(\psi_*)\subseteq \text{Im}(\phi_*)$.

    Passo 2: $\text{Im}(\psi_*)=\text{Ker}(\delta).$

    Seja $\overline{z}=z+B_n(\mathcal{C})\in \text{Im}(\psi_*)$, onde $z=\psi(b)$ para $b\in Z_n(\mathcal{B})$. Pela construção do homomorfismo conectante, $\delta(\overline{z})=a+B_{n-1}(\mathcal{A})$, em que $a\in A_{n-1}$ é tal que $\phi(a)=\partial b$. Como $\partial b=0$ e $\phi$ é injetora, temos que $a=0$, portanto, $\overline{z}\in \text{Ker}(\delta)$. Isso prova que $\text{Im}(\psi_*)\subseteq \text{Ker}(\delta)$.
    
    Por outro lado, seja $\overline{z}=z+B_n(\mathcal{C})\in \text{Ker}(\delta)$. Pela construção de $\delta$, definimos $a:=\tilde{\delta_n}(z)\in A_{n-1}$ e seja $b\in B_n$ tal que $\psi(b)=z$ e, ainda, $\phi(a)=\partial b$. Como $\overline{z}\in \text{Ker}(\delta)$, temos $a\in B_{n-1}(\mathcal{A})$, ou seja, existe $a^+\in A_n$ tal que $a=\partial a^+$. Além disso, temos
    \[\partial(b-\phi(a^+))=\partial b-\partial \phi(a^+)=\partial b- \phi(\partial a^+)=\partial b-\phi(a)=\partial b-\partial b=0.\]
    Isso mostra que $b-\phi(a^+)\in Z_n(\mathcal{B})$ e, portanto, está bem definida a classe $(b-\phi(a^+))+B_n(\mathcal{B})\in H_n(\mathcal{B})$. Além disso, 
    \[\psi(b-\phi(a^+))=\psi (b)-\psi\circ\phi(a^+)=\psi(b)=z.\]
    Portanto, $\psi_*((b-\phi(a^+))+B_n(\mathcal{B}))=\overline{z}$. Isso prova que $\text{Ker}(\delta)\subseteq \text{Im}(\psi_*)$.

    Passo 3: $\text{Im}(\delta)=\text{Ker}(\phi_*)$.

    Por construção, se $a+B_{n-1}(\mathcal{A})\in \text{Im}(\delta)$, então $\phi(a)=\partial b$ para algum $b\in B_n$ e, assim, $\phi_*(a+B_{n-1}(\mathcal{A}))=B_n(\mathcal{B})$, ou seja $a+B_{n-1}(\mathcal{A})\in \text{Ker}(\phi_*)$. Isso prova que $\text{Im}(\delta)\subseteq \text{Ker}(\phi_*)$. 

    Por outro lado, seja $a+B_{n-1}(\mathcal{A})\in \text{Ker}(\phi_*)$. Então, $\phi(a)\in B_{n-1}(\mathcal{B})$, ou seja, existe $b\in B_n$ tal que $\phi(a)=\partial b$. Para o elemento $\psi(b)\in C_n$, temos:
    \[\partial \psi(b)=\psi(\partial b)=\psi\circ\phi(a)=0.\]
    Logo, $\psi(b)\in Z_n(\mathcal{C})$ e, pela construção de $\delta$, temos $\delta
    (\psi(b)+B_n(\mathcal{C}))=a+B_{n-1}(\mathcal{A}).$ isso mostra que $\text{Ker}(\phi_*)\subseteq \text{Im}(\delta)$.
\end{dem}
    
\begin{titlemize}{Lista de consequências}
    \item \hyperref[sequencia-de-mayer-vietoris-prop]{Sequência de Mayer-Vietoris}.
	%\item \hyperref[]{}
\end{titlemize}

\subsection{Homologia Simplicial}
\label{homologia-simplicial-def}
\begin{titlemize}{Lista de dependências}
	\item \hyperref[complexo-de-cadeias-def]{Complexo de cadeias};\\ 
    \item \hyperref[aplicacao-de-cadeias-def]{Aplicação de cadeias};\\
    \item \hyperref[homotopia-de-cadeias-def]{Homotopia de cadeia}.
% quantas dependências forem necessárias.
\end{titlemize}
\begin{defi}
	Seja $K$ um complexo simplicial, e seja $O_n (K)$ um grupo abeliano livre com base dada por símbolos 
    \[\{[v_0,...,v_n]:v_0,v_1,...,v_n \text{ geram um simplexo em }K\}.\]
    Aqui, $v_i$ são considerado ordenado, e eles podem gerar um simplexo de dimensão menor que $n$ (i.e. a lista pode repetir).

    Seja $T_n(K)\le O_n(K)$ um subgrupo gerado por seguintes elementos 
    \begin{itemize}
        \item a sequência $[v_0,...,v_n]$ tem vertices repetidos,
        \item $[v_0,v_1,...,v_n]-sign(\sigma)\cdot[v_{\sigma(0)},v_{\sigma(1)},...,v_{\sigma(n)}]$, onde $\sigma$ é uma permutação em $\{0,1,...,n\}$. 
    \end{itemize}
    Definoms $C_n(K):=O_n(K)/T_n(K)$ como grupo quociente.
\end{defi}

\begin{defi}
    O \textbf{operador bordo} é um homomorfismo de grupo dado por 
    \begin{align*}
        d_n:C_n(K)&\longrightarrow C_{n-1}(K)\\
        [v_0,v_1,...,v_n]&\longmapsto \sum_{i=0}^n (-1)^i \cdot[v_0,v_1,...,\widehat{v_i},...,v_n],
    \end{align*}
    onde $[v_0,v_1,...,\widehat{v_i},...,v_n]$ denota a sequência obtida pela remoção de $v_i$.
\end{defi}

\begin{lemma}
    O homomorfismo $d_{n-1}\circ d_n:C_n(K)\rightarrow C_{n-2}(K)$ é nulo.
\end{lemma}

\begin{dem}
    Seja $[v_0,...,v_n]\in C_n(K)$, então 
    \begin{align*}
        d_{n-1}\circ d_n ( [v_0,...,v_n])&=d_{n-1}\Bigl(\sum_{i=0}^n (-1)^i \cdot[v_0,v_1,...,\widehat{v_i},...,v_n] \Bigr) \\
        &=\sum_{i=0}^n (-1)^i  \Bigl( \sum_{k=0}^{i-1}(-1)^k[v_0,...,\widehat{v_k},...,\widehat{v_i},...,v_n])\\
        &+\sum_{k=i}^{n-1} (-1)^k[v_0,...,\widehat{v_i},...,\widehat{v_{k+1}},...,v_n]  \Bigr).
    \end{align*}
    O coeficiente de $[v_0,...,\widehat{v_a},...,\widehat{v_b},...,v_n]$ é $(-1)^a(-1)^b$ de $k=a$ e $i=b$ mais $(-1)^a(-1)^{b-1}$ de $i=a$ e $k+1=b$. Assim, cada termo se cancela, o que implica que $d_{n-1}\circ d_n([v_0,...,v_n])=0$. Como $C_n(K)$ é gerado pelos simplexos $[v_0,...,v_n]$, concluímos que $d_{n-1}\circ d_n=0$.
\end{dem}

Como a consequência, esse lema garante que $\text{Im}(d_n)\subseteq \text{Ker}(d_{n-1})$. Ou seja, a sequência 
\[...\rightarrow C_{n+1}(K)\xrightarrow{d_{n+1}}C_n(K)\xrightarrow{d_n} C_{n-1}(K)\rightarrow...\rightarrow 0\]
é um complexo de cadeias.

\begin{defi}
    O n-ésima \textbf{grupo de homologia simplicial} de um complexo simplicial $K$ é 
    \[H_n(K):=\frac{\text{Ker}(d_n)}{\text{Im}(d_{n+1})}.\]
\end{defi}


\begin{titlemize}{Lista de consequências}
	\item %\hyperref[consequencia1]{Consequência 1};\\ %'consequencia1' é o label onde o conceito Consequência 1 aparece
	%\item \hyperref[]{}
\end{titlemize}

\subsection{Homologia-singular} %afirmação aqui significa teorema/proposição/colorário/lema
\label{homologia-singular-def}
\begin{titlemize}{Lista de dependências}
	\item \hyperref[complexo-de-cadeias-def]{Complexo de cadeias};\\ 
    \item \hyperref[aplicacao-de-cadeias-def]{Aplicação de cadeias};\\
    \item \hyperref[homotopia-de-cadeias-def]{Homotopia de cadeia}.\\
\end{titlemize}

\begin{defi}
    Seja $X$ um espaço topológico. Um p-\textbf{simplexo singular} em $X$ é uma função contínua 
    \[\phi:\Delta^p\longrightarrow X.\]
\end{defi}

\begin{defi}
    Se $\phi$ é um $p$-simplexo singular em um espaço topológico $X$, e $i$ é um inteiro tal que $0\le i\le p$, definimos $\partial_i (\phi)$, um (p-1)-simplexo singular em $X$, por 
    \[\partial_i \phi(t_0,...,t_{p-1})=\phi(t_0,...,t_{i-1},0,t_{i+1},...,t_{p-1}).\]
    Ou seja, $\partial_i \phi=\phi|_{[v_0,...,\widehat{v_i},...,v_{p}]}$ é a $i$-ésima face de $\phi$, obtida pela substituição do parâmetro $t_i$ por zero, onde $[v_0,...,v_p]=\Delta^p$
\end{defi}

\begin{defi}
    Seja $X$ um espaço topológico, definimos $S_n(X)$ como grupo abeliano livre cujo base é o conjunto de todos $n$-simplexos singulares de $X$. Um elemento de $S_n(X)$ é dito $n$-\textbf{cadeia singular} de $X$ e tem a forma 
    \[\sum_\phi n_\phi \phi\]
    onde $n_\phi$ é um inteiro, igual a zero para todos, exceto um número finito de $\phi$.
\end{defi}

Podemos estender o operador de $i$-ésima face para um homomorfismo de $S_n(X)$ em $S_{n-1} (X)$. 

\begin{defi}
    Seja $X$ um espaço topológico, definimos o operador $\partial_i$ como
    \begin{align*}
        \partial_i: S_n(X)&\longrightarrow S_{n-1}(X)\\
        \sum_\phi n_\phi \phi&\longmapsto \sum_\phi n_\phi \partial_i\phi.
    \end{align*}
    O \textbf{operador bordo} é então um homomorfismo definido por
    \begin{align*}
        \partial_{(n)}=\sum_{i=0}^n (-1)^i \partial_i:S_n(X)\longrightarrow S_{n-1}(X).
    \end{align*}
    Para simplificar a notação, omitiremos o índice do operador $\partial_{(n)}$.
\end{defi}

\begin{lemma}
    O homomorfismo $\partial\circ \partial:S_n(X)\rightarrow S_{n-2}(X)$ é nulo.
\end{lemma}

\begin{dem}
    Seja $\phi\in S_n(X)$, então 
    \begin{align*}
        \partial\circ \partial(\phi)&=\partial\Bigl(\sum_{i=0}^n (-1)^i \partial_i\phi \Bigr) \\
        &=\sum_{i=0}^n (-1)^i  \Bigl( \sum_{k=0}^{i-1}(-1)^k \partial_k\circ\partial_i \phi)+\sum_{k=i}^{n-1} (-1)^k\partial_{k}\circ\partial_i \phi  \Bigr).
    \end{align*}
    Note que $\partial_k\circ\partial_i\phi=\partial_{i-1}\circ\partial_k \phi$ se $k<i$. Logo, o coeficiente de $\partial_a\circ\partial_b \phi$ é $(-1)^a(-1)^b$ de $k=a$ e $i=b$ mais $(-1)^a(-1)^{b-1}$ de $i=a$ e $k=b-1$. Assim, cada termo se cancela, o que implica que $\partial\circ\partial\phi=0$. Como $S_n(K)$ é gerado pelos n-simplexos singulares, concluímos que $\partial\circ\partial=0$.
\end{dem}

Como a consequência, esse lema garante que $\text{Im}(\partial_{(n+1)})\subseteq \text{Ker}(\partial_{(n)})$. Ou seja, a sequência 
\[...\rightarrow S_{n+1}(X)\xrightarrow{\partial}S_n(X)\xrightarrow{\partial} S_{n-1}(X)\rightarrow...\rightarrow 0\]
é um complexo de cadeias. Denotamos esse complexo por $S(X)_\bullet$.

Assim como no complexo de cadeias, denotamos $\text{Im}(\partial_{(n+1)})$ por $B_n(X)$ e $\text{Ker}(\partial_{(n)})$ por $Z_n(X)$.

\begin{defi}
    O n-ésima \textbf{grupo de homologia singular} de um espaço topológico $X$ é 
    \[H_n(X):=\frac{Z_n(X)}{B_n(X)}.\]
\end{defi}

\begin{titlemize}{Lista de consequências}
    \item \hyperref[homomorfismo-de-homologias-singulares-induzido-prop]{Homomorfismo de homologias singulares induzido}.\\ %'consequencia1' é o label onde o conceito Consequência 1 aparece
	%\item \hyperref[]{}
\end{titlemize}

\subsection{Homomorfismo de homologias singulares induzido} %afirmação aqui significa teorema/proposição/colorário/lema
\label{homomorfismo-de-homologias-singulares-induzido-prop}
\begin{titlemize}{Lista de dependências}
	\item \hyperref[complexo-de-cadeias-def]{Complexo de cadeias};\\ 
    \item \hyperref[aplicacao-de-cadeias-def]{Aplicação de cadeias};\\
    \item \hyperref[homotopia-de-cadeias-def]{Homotopia de cadeia};\\
    \item \hyperref[homomorfismo-induzido-de-cadeias-prop]{Homomorfismo induzido de cadeias};\\
    \item \hyperref[equivalencia-de-homotopia-de-cadeias-def]{Equivalência de homotopia de cadeias};\\
    \item \hyperref[homologia-singular-def]{Homologia singular}.
\end{titlemize}

\begin{lemma}
    Uma função contínua $f:X\rightarrow Y$ entre espaços topológicos induz um homomorfismo de cadeia 
    \begin{align*}
    f_n:S_n(X)&\longrightarrow S_n(Y)\\
    \sum_{\phi}n_\phi\phi&\longmapsto \sum_\phi n_\phi (f\circ \phi).
    \end{align*}
\end{lemma}

\begin{dem}
    É fácil ver que $f_n$ é um homomorfismo de grupo, basta mostrar que $f_{n-1}\circ\partial=\partial\circ f_n$. Seja $\phi\in S_n(X)$. Como 
    \begin{align*}
        f_{n-1}\partial (\phi)=f_{n}(\sum_{i=0}^n (-1)^i \partial_i \phi)=\sum_{i=0}^n(-1)^i f\circ\partial_i\phi=\sum_{i=0}^n (-1)^i \partial_i(f\circ\phi)=\partial\circ f_n (\phi),
    \end{align*}
    podemos concluir que $f_\bullet:=(f_n)_{n\ge 0}$ é um homomorfismo de cadeias.
\end{dem}

Por lemma \ref{homomorfismo-induzido-de-cadeias-prop}, temos 

\begin{corol}
    Uma função contínua $f:X\rightarrow Y$ entre espaços topológicos induz um homomorfismo 
    \begin{align*}
        f_*: H_n(X)&\longrightarrow H_n(Y)\\
        [\sum_\phi n_\phi \phi]&\longmapsto [\sum_\phi n_\phi (f\circ \phi)]
    \end{align*} 
    entre homologias singulares.

    Além disso, temos 
    \begin{enumerate}
        \item $(f\circ g)_*=f_*\circ g_*$,
        \item $(id_X)_*=id_{H_n(X)}$ para todo $n\ge 0$.
    \end{enumerate}
\end{corol}
Isso mostra que a homologia singular é um invariante topológico.
\begin{corol}
    Se $f:X\rightarrow Y$ é um homeomorfismo, então $f_*:H_n(X)\rightarrow H_n(X)$ é um isomorfismo para todo $n\ge 0$. 
\end{corol}

\begin{proof}
    Seja $g:Y\rightarrow X$ a função inversa de $f$, pelo Corolário anterior, temos 
    \[f_*\circ g_*=(f\circ g)_*=(id_Y)_*=id_{H_n(Y)},\]
    e vice-versa.
\end{proof}

\begin{thm}
    Sejam $f,g:X\rightarrow Y$ funções contínuas entre espaços topológicos. Se $F:X\times I\rightarrow Y$ é uma homotopia de $f$ em $g$. Então, $f$ e $g$ induzem um mesmo homomorfismo $f_*=g_*:H_n(X)\rightarrow H_n(Y).$
\end{thm}

\begin{dem}
    A demonstração é baseada no Algebraic Topology do Allen Hatcher; o leitor pode encontrar uma interpretação geométrica dessa prova no livro.

    O ponto crucial é um procedimento para subdividir $\Delta^n\times I$ em simplexos. Em $\Delta^n\times I$, seja $\Delta^n\times \{0\}=[v_0,...,v_n]$ e $\Delta^n\times\{1\}=[w_0,...,w_n]$, onde $v_i$ e $w_i$ possuem a mesma imagem sob a projeção $\Delta^n\times I\rightarrow \Delta^n$. Podemos passar de $[v_0,...,v_n]$ para $[w_0,...,w_n]$ interpolando uma sequência de $n$-simplexos, cada um obtido do anterior movendo um vértice $v_i$ até $w_i$, começando com $v_n$ e trabalhando para trás até $v_0$. Portanto, o primeiro passo é mover $[v_0,...,v_n]$ para cima até $[v_0,...,v_{n-1},w_n],$ então o segundo passo é mover isso para $[v_0,...,v_{n-2}, w_{n-1},w_n]$ e assim por diante. Na etapa típica $[v_0,...,v_{i},w_{i+1},...,w_n]$ move-se para cima até $[v_0,...,v_{i-1},w_i,...,w_n]$. A região entre esses dois simplexos é exatamente o (n+1)-simplexo $[v_0,...,v_i,w_i,...,w_n]$ que tem $[v_0,...,v_i,w_{i+1},...,w_n]$ como face inferior e $[v_0,...,v_{i-1},w_i,...,w_n]$ como face superior. Em conjunto, $\Delta^n\times I$ é a união de (n+1)-simplexos $[v_0,...,v_i,w_i,...,w_n]$, cada um intersectando o próximo em uma face.

    Agora, definimos \textbf{operador prisma} $P:S_n(X)\rightarrow S_{n+1}(Y)$ pela seguinte fórmula 
    \[P(\phi)=\sum_{i=0}^n (-1)^i F\circ (\phi\times id_I)|_{[v_0,...,v_i,w_i,...,w_n]},\]
    onde $\phi$ é um $n$-simplexo singular. Vamos mostrar que esses operadores prisma satisfazem a seguinte relação 
    \[\partial P=g_\bullet-f_\bullet-P\partial.\]
    Geometricamente, o lado esquerdo da equação representa o bordo da prisma, e os três termos do lado direito representam a base superior $\Delta^n\times \{1\}$, a base inferior $\Delta^n\times\{0\}$, e os lados $\partial \Delta^n\times I$ da prisma. Para provar a relação, calculamos 
    \begin{align*}
        \partial P(\phi)=&\sum_{i=0}^{n} \Bigl(\sum_{j=0}^{i} (-1)^i(-1)^j F\circ (\phi\times id_I)|_{[v_0,...,\widehat{v_j},...,v_i, w_i,...,w_n]} \\
        &+ \sum_{j=i}^{n} (-1)^i(-1)^{j+1} F\circ (\phi\times id_I)|_{[v_0,...,v_i,w_i,...,\widehat{w_j},...,w_n]}\Bigr)
    \end{align*}
    Ou seja, 
    \begin{align*}
        \partial P(\phi)=&\sum_{j\le i\le n} (-1)^i(-1)^j F\circ (\phi\times id_I)|_{[v_0,...,\widehat{v_j},...,v_i, w_i,...,w_n]} \\
        &+ \sum_{i\le j\le n} (-1)^i(-1)^{j+1} F\circ (\phi\times id_I)|_{[v_0,...,v_i,w_i,...,\widehat{w_j},...,w_n]}
    \end{align*}
    Os termos com $i=j$ nas duas somas se cancelam exceto para $F\circ(\phi\times id_I)|_{[\widehat{v_0},w_0,...,w_n]}$, que é $g\circ\phi=g_\bullet (\phi)$, e $-F\circ (\phi\times id_I)|_{[v_0,...,v_n,\widehat{w_n}]},$ que é $-f\circ\phi=-f_\bullet(\phi)$. Os termos com $i\ne j$ são exatamente $-P\partial (\phi)$, pois 
    \begin{align*}
        P\partial(\phi)=&\sum_{j=0}^{n} \Bigl( \sum_{i=j+1}^{n} (-1)^{i-1}(-1)^j F\circ (\phi\times id_I)|_{[v_0,...,\widehat{v_j},...,v_i,w_i,...,w_n]}\\
        &+\sum_{i=0}^{j-1} (-1)^i(-1)^j F\circ(\phi\times id_I)|_{[v_0,...,v_i,w_i,...,\widehat{w_j},...,w_n]}\Bigr)
    \end{align*}
    ou seja,
    \begin{align*}
        P\partial(\phi)=&\sum_{j<i\le n} (-1)^{i-1}(-1)^j F\circ (\phi\times id_I)|_{[v_0,...,\widehat{v_j},...,v_i,w_i,...,w_n]}\\
        &+\sum_{i<j\le n} (-1)^i(-1)^j F\circ(\phi\times id_I)|_{[v_0,...,v_i,w_i,...,\widehat{w_j},...,w_n]}\Bigr).
    \end{align*}
    Portanto, $P$ é uma homotopia de cadeias de $f$ e $g$. Pelo Lema \ref{homomorfismo-induzido-de-cadeias-prop}, temos que $f_*=g_*$.
\end{dem}

\begin{corol}
    Se dois espaços topológicos $X,Y$ são equivalentes homotópicos, então $H_n(X)\cong H_n (Y)$ para todo $n\ge 0$. 
\end{corol}

\begin{titlemize}{Lista de consequências}
	\item \hyperref[consequencia1]{Consequência 1};\\ %'consequencia1' é o label onde o conceito Consequência 1 aparece
\end{titlemize}

\subsection{Homologia singular de um ponto} %afirmação aqui significa teorema/proposição/colorário/lema
\label{homologia-singular-de-um-ponto-prop}
\begin{titlemize}{Lista de dependências}
	\item \hyperref[complexo-de-cadeias-def]{Complexo de cadeias};\\ 
    \item \hyperref[homologia-singular-def]{Homologia singular}.
\end{titlemize}

\begin{prop}
    O n-ésimo grupo de homologia de um ponto é igual a
    \begin{align*}
        H_n(\{x\})\cong\begin{cases}
            \mathbb{Z}&\text{se }n=0\\
            0&\text{se }n>0.
        \end{cases}
    \end{align*}
\end{prop}

\begin{dem}
    Note que só existe um único n-simplexo singular $\sigma_n:\Delta^n\rightarrow \{x\}$ que é uma função constante. Logo, $S_n(\{x\})$ é o grupo cíclico infinito gerado por $\sigma_n$. Além disso, para $n\ge 1$, $\partial_n \sigma_n=\sigma_{n-1}$. Dessa forma, obtemos 
    \begin{align*}
        \partial\sigma_n=\sum_{i=0}^n (-1)^i \partial_n\sigma_n=\begin{cases}
            \sigma_{n-1}&\text{se n é par}\\
            0&\text{se n é ímpar}.
        \end{cases}
    \end{align*}
    Portanto, o complexo de cadeias $S(\{x\})_*$ é a sequência 
    \[...\xrightarrow{0} S_{4}(\{x\})\xrightarrow{\partial} S_3(\{x\})\xrightarrow{0} S_2(\{x\})\xrightarrow{\partial}...\rightarrow 0,\]
    em que $\partial_{(2k)}$ é o isomorfismo dado por $\sigma_{2k}\rightarrow \sigma_{2k-1}$ para todo $k\ge 1$. Logo $Z_n(\{x\})=B_n(\{x\})=\{0\}$ para todo $n\ge 1$, enquanto que $Z_0(\{x\})=S_0 (\{x\})\cong \mathbb{Z}$ e $B_0(\{x\})=0$.
\end{dem}

Como $\Delta^n$ é conexo, qualquer função contínua de $\Delta^n$ para um conjunto finito $\{x_1,...,x_m\}$, onde o conjunto é equipado com a topologia discreta, deve ser constante. Consequentemente, o mesmo argumento utilizado anteriormente implica que o n-ésimo grupo de homologia de $\{x_1,...,x_n\}$ é igual a
\begin{align*}
        H_n(\{x_1,...,x_m\})\cong\begin{cases}
            \mathbb{Z}^m&\text{se }n=0\\
            0&\text{se }n>0.
        \end{cases}
    \end{align*}
\begin{titlemize}{Lista de consequências}
    \item \hyperref[homologia-singular-de-um-espaco-contratil-prop]{Homologia singular de um espaço contrátil}.
	%\item \hyperref[]{}
\end{titlemize}

\subsection{0-ésimo grupo de homologia singular de um espaço 0-conexo} %afirmação aqui significa teorema/proposição/colorário/lema
\label{0-esimo-grupo-de-homologia-de-espaco-zero-conexo-prop}
\begin{titlemize}{Lista de dependências}
	\item \hyperref[complexo-de-cadeias-def]{Complexo de cadeias};\\ 
    \item \hyperref[homologia-singular-def]{Homologia singular}.
\end{titlemize}

\begin{prop}
    Se um espaço topológico não vazio $X$ é 0-conexo ou conexo por caminho, então $H_0(X)\cong \mathbb{Z}$.
\end{prop}
\begin{dem}
    Note que $Z_0(X)=S_0(X)$. Logo, cada elemento de $Z_0(X)$ tem a forma $z=a_1x_1+...+a_kx_k$, onde $a_i\in \mathbb{Z}$ e $x_i\in X$.

    Consideramos o homomorfismo $\alpha:Z_0(X)\rightarrow \mathbb{Z}$ definido por 
    \[\alpha (a_1x_1+...+a_kx_k)=a_1+...+a_k.\]
    Como $X$ não é vazio, $\alpha$ é sobrejetor. Vamos provar que $B_0(X)=\text{Ker}(\alpha)$.

    Para cada 1-simplexo singular $\sigma\in S_1(X)$ temos 
    \[\alpha(\partial \sigma)=\alpha(\sigma(0,1)-\sigma(1,0))=1-1=0,\]
    o que implica que $B_0(X)\subseteq \text{Ker}(\alpha)$. 
    
    Por outro lado, seja dada uma 0-cadeia $c_0=a_1x_1+...+a_kx_k$ em $\text{Ker}(\alpha)$. Fixemos um ponto $x_0\in X$. Pela definição de 0-conexo, para cada índice $i$ existe um 1-simplexo singular (uma curva) $\sigma_i:\Delta^1\rightarrow X$ tal que $\sigma_i (1,0)=x_0$ e $\sigma_i(0,1)=x_i$. Assim, segue que a 1-cadeia singular $c_1=a_1\sigma_1+...+a_k\sigma_k$ tem bordo 
    \begin{align*}
        \partial(c_1)=&a_1\partial \sigma_1+...+a_k\partial\sigma_i=a_1 (x_1-x_0)+...+a_k (x_k-x_0)\\
        &=a_1x_1+...+a_kx_k -(\sum_{i=0}^k a_i)x_0.
    \end{align*}
    Como $c_0\in \text{Ker}(\alpha)$, $\sum_{i=0}^k a_i=0$. Logo, segue que 
    \[\partial c_1=c_0.\]
    Isso mostra que $c_0\in B_0(X)$, ou seja, $\text{Ker}(\alpha)\subseteq B_0(X).$

    Com isso provamos que $B_0(X)=\text{Ker}(\alpha)$. Finalmente, pelo Teorema do Isomorfismo, obtemos
    \[H_0(X)=Z_0(X)/B_0(X)=Z_0(X)/\text{Ker}(\alpha)\cong \mathbb{Z}.\]
\end{dem}

Se $X$ não é 0-conexo, assumamos que seja $X=\bigsqcup_\lambda X_\lambda$ a separação de $X$ em componentes por caminho. Por continuidade, a imagem de cada n-simplexo singular $\sigma:\Delta^n\rightarrow X$ está contida em um e somente um componente por caminho $X_\lambda$. Isso resulta que $S_n(X)=\bigoplus_\lambda S_n(X_\lambda)$. Nesse contexto, o operador bordo opera componente a componente. Logo, 
\[Z_n(X)=\bigoplus_\lambda Z_n(X_\lambda)\;\;\;\text{ e }\;\;\;B_n(X)=\bigoplus_\lambda B_n(X_\lambda).\] 
Portanto, para cada $n\ge 0,$
\[H_n(X)=\frac{\bigoplus_\lambda Z_n(X_\lambda)}{\bigoplus_\lambda B_n(X_\lambda)}\cong\bigoplus\frac{Z_n(X_\lambda)}{B_n(X_\lambda)}=\bigoplus_\lambda H_n(X_\lambda
).\]
Em particular, 
\begin{corol}
    O grupo $H_0(X)$ é abeliano livre com tantos geradores quanto as componentes por caminho do espaço $X$.
\end{corol}

\begin{titlemize}{Lista de consequências}
    \item \hyperref[0-conexo-e-homomorfismo-de-homologia-induzido-prop]{0-conexo e homomorfismo de homologia induzido}.\\
	%\item \hyperref[]{}
\end{titlemize}

\subsection{0-conexo e homomorfismo de homologias induzido} %afirmação aqui significa teorema/proposição/colorário/lema
\label{0-conexo-e-homomorfismo-de-homologia-induzido-prop}
\begin{titlemize}{Lista de dependências}
    \item \hyperref[homologia-singular-def]{Homologia singular};\\
    \item \hyperref[homomorfismo-de-homologias-singulares-induzido-prop]{Homomorfismo de homologias singulares induzido};\\
    \item \hyperref[0-esimo-grupo-de-homologia-de-espaco-zero-conexo-prop]{0-ésimo grupo de homologia singular de um espaço 0-conexo}.
\end{titlemize}

\begin{prop}
    Sejam $X,Y$ espaços topológicos não vazios. Seja $f:X\rightarrow Y$ uma função contínua e considere o homomorfismo induzido $f_*:H_0(X)\rightarrow H_0(Y)$. Então:
    \begin{enumerate}
        \item Se $X$ é 0-conexo, então $f_*$ é um monomorfismo.
        \item Se $Y$ é 0-conexo, então $f_*$ é um epimorfismo.
        \item Se $X$ e $Y$ são 0-conexos, então $f_*$ é um isomorfismo.
    \end{enumerate}
\end{prop}

\begin{dem}
    Consideramos os epimorfismos $\overline{\alpha}:H_0(X)\rightarrow\mathbb{Z}$ e $\overline{\beta}:H_0(Y)\rightarrow\mathbb{Z}$ definidos por 
    \[\overline{\alpha}(a_1x_1+...+a_kx_k+B_0(X))=a_1+...+a_k;\]
    \[\overline{\beta}(b_1y_1+...+b_ly_l+B_0(Y))=b_1+...+b_l.\]
    Temos que $\overline{\alpha}$ (resp. $\overline{\beta}$) é um isomorfismo se $X$ (resp. $Y$) é 0-conexo. Além disso, para uma classe de homologia $[z]:=a_1x_1+...+a_kx_k+B_0(X)$ em $H_0(X)$, temos: 
    \begin{align*}
        \overline{\beta}(f_*([z]))&=\overline{\beta}(f\circ (a_1x_1+...+a_kx_k)+B_0(Y))\\
        &=\overline{\beta}(a_1(f\circ x_1)+...+a_k (f\circ x_k)+B_0(Y))\\
        &=a_1+...+a_k=\overline{\alpha}([z])
    \end{align*}
    Portanto, $\overline{\beta}\circ f_*=\overline{\alpha}$. Assim:
    \begin{enumerate}
        \item Se $X$ é 0-conexo, então $\overline{\alpha}$ é um isomorfismo e, consequentemente, $f_*$ é um monomorfismo.
        \item Se $Y$ é 0-conexo, então $\overline{\beta}$ é um isomorfismo e, consequentemente, $f_*$ é um epimorfismo.
        \item Se $X$ e $Y$ são 0-conexos, então, com base nos dois itens anteriores, $f_*$ é um isomorfismo.
    \end{enumerate}
\end{dem}

%\begin{titlemize}{Lista de consequências}
    %\item %\hyperref[homomorfismo-de-homologias-singulares-induzido-prop]{Homomorfismo de homologias singulares induzido}.\\
	%\item \hyperref[]{}
%\end{titlemize}

\subsection{Homologia singular de um espaço contrátil} %afirmação aqui significa teorema/proposição/colorário/lema
\label{homologia-singular-de-um-espaco-contratil-prop}
\begin{titlemize}{Lista de dependências}
	\item \hyperref[complexo-de-cadeias-def]{Complexo de cadeias};\\ 
    \item \hyperref[homologia-singular-def]{Homologia singular};\\
    \item \hyperref[homomorfismo-de-homologias-singulares-induzido-prop]{Homomorfismo de homologias singulares induzido};\\
    \item \hyperref[homologia-singular-de-um-ponto-prop]{Homologia singular de um ponto}.
\end{titlemize}

\begin{prop}
    Se $X$ é um espaço contrátil, então o n-ésimo grupo de homologia de $X$ é igual a
    \begin{align*}
        H_n(X)\cong\begin{cases}
            \mathbb{Z}&\text{se }n=0\\
            0&\text{se }n>0.
        \end{cases}
    \end{align*}
\end{prop}

\begin{dem}
    Como grupo de homologia é uma invariante homotópica, para um ponto $x\in X$, a inclusão $\{x\}\hookrightarrow X$ induz isomorfismo em homologia, o que mostra que $X$ tem o mesmo grupo de homologia de um ponto.
\end{dem}
    
%\begin{titlemize}{Lista de consequências}
    %\item %\hyperref[homomorfismo-de-homologias-singulares-induzido-prop]{Homomorfismo de homologias singulares induzido}.\\
	%\item \hyperref[]{}
%\end{titlemize}

\subsection{Simplexos singulares subordinados a uma cobertura} %afirmação aqui significa teorema/proposição/colorário/lema
\label{simplexos-singulares-subordinados-a-uma-cobertura-def}
\begin{titlemize}{Lista de dependências}
	\item \hyperref[complexo-de-cadeias-def]{Complexo de cadeias};\\ 
    \item \hyperref[homologia-singular-def]{Homologia singular};\\
    \item \hyperref[homomorfismo-de-homologias-singulares-induzido-prop]{Homomorfismo de homologias singulares induzido};\\
    \item \hyperref[simplexos-singulares-subordinados-a-uma-cobertura-def]{Simplexos singulares subordinados a uma cobertura};\\
    
\end{titlemize}

\begin{defi}
    Seja $X$ um espaço topológico e seja $\mathcal{U}$ uma cobertura de $X$, não necessariamente aberto. Denotamos por $S^{\mathcal{U}}_n (X)$ o subgrupo (abeliano livre) de $S_n (X)$ gerado pelos n-simplexos singulares $\sigma:\Delta^n\rightarrow X$ cuja imagem $\sigma(\Delta^n)$ está contido em algum elemento de $\mathcal{U}$, os quais chamamos \textbf{n-simplexos singulares subordinados} à $\mathcal{U}$.
\end{defi}
    Como para cada $0\le i\le n$, $\text{Im}(\partial_i \sigma )\subseteq \text{Im}(\sigma)$, para cada $\sigma\in S_n^{\mathcal{U}}(X)$, temos que $\partial\sigma\in S^{\mathcal{U}}_{n-1}(X)$. Desse modo, o operador bordo $\partial$ do complexo de cadeias singulares de $X$ induz, por restirção de domínio e contradomínio um operador bordo $\partial^{\mathcal{U}}$ tal que 
    \[...\rightarrow S^{\mathcal{U}}_{n+1}(X)\xrightarrow{\partial^{\mathcal{U}}}S_n^{\mathcal{U}}(X)\xrightarrow{\partial^{\mathcal{U}}}S_{n-1}^{\mathcal{U}} (X)\rightarrow...\]
    é um complexo de cadeias, denotado por $S^{\mathcal{U}}(X)_*$.

\begin{defi}
O n-ésimo grupo de homologia $H_n^{\mathcal{U}}(X)$ do complexo de cadeias $S^{\mathcal{U}}(X)_*$ é chamado o \textbf{n-ésimo grupo de homologia de} $X$ \textbf{subordinada} à $\mathcal{U}$.
\end{defi}

A função identidade $id:X\rightarrow X$ induz uma aplicação de cadeias $i_n:S_n^{\mathcal{U}}(X)\rightarrow S_n (X)$ e, por conseguinte, um homomorfismo em homologia.

%Suponhamos que a coleção $int(\mathcal{U}):=\{int(U):U\in \mathcal{U}\}$ seja uma cobertura de $X$. Então, dado um n-simplexo singular $\sigma:\Delta^n\rightarrow X$, a coleção $\mathcal{V}=\{\sigma^{-1}(int(U)):U\in \mathcal{U}\}$ é uma cobertura aberta do compacto $\Delta^n$. Tomando um número de Lebesgue $\delta>0$ para tal cobertura para cada $K\subseteq \Delta^n$ com diâmetro menor que $\delta$, existe um $U\in\mathcal{U}$ tal que $\sigma(K)\subseteq int (U)$. Por meio de iterações do processo de subdivisões baricêntricas de $\Delta^n$, o n-simplexo $\sigma$ pode ser expressado como uma n-cadeia $\sigma=a_1\sigma_1+...+a_k\sigma_k$ em que $\sigma_i \in C^{\mathcal{U}}_n(X)$ para cada $i$

\begin{prop}
    Suponhamos que a coleção $int(\mathcal{U}):=\{int(U):U\in \mathcal{U}\}$ seja uma cobertura de $X$. Então, $i_*:H^{\mathcal{U}}_n (X)\rightarrow H_n (X)$ é um isomorfismo para todo $n\ge 0$
\end{prop}

\begin{dem}
    A prova é feita por iterações do processo de subdivisões baricêntricas. Esta prova é bastante longa e trabalhosa, por isso omitimos a demonstração. O leitor consegue achar a demonstração em Proposição 2.21 em 
    \textit{Hatcher, Allen. Algebraic Topology. Cambridge, Cambridge University Press, 2001.}
\end{dem}

\begin{titlemize}{Lista de consequências}
    \item \hyperref[sequencia-de-mayer-vietoris-prop]{Sequência de Mayer-Vietoris}.\\
	%\item \hyperref[]{}
\end{titlemize}

\subsection{Sequência de Mayer-Vietoris} %afirmação aqui significa teorema/proposição/colorário/lema
\label{sequencia-de-mayer-vietoris-prop}
\begin{titlemize}{Lista de dependências}
	\item \hyperref[complexo-de-cadeias-def]{Complexo de cadeias};\\ 
    \item \hyperref[aplicacao-de-cadeias-def]{Aplicação de cadeias};\\
    \item \hyperref[homomorfismo-induzido-de-cadeias-prop]{Homomorfismo induzido de cadeias};\\
    \item \hyperref[sequencia-exata-def]{Sequência exata};\\
    \item \hyperref[homomorfismo-conectante-def]{Homomorfismo conectante};\\
    \item \hyperref[sequencia-exata-longa-induzida-prop]{Sequência exata longa induzida};\\
    \item \hyperref[homologia-singular-def]{Homologia singular};\\
    \item \hyperref[homomorfismo-de-homologias-singulares-induzido-prop]{Homomorfismo de homologias singulares induzido};\\
    \item \hyperref[simplexos-singulares-subordinados-a-uma-cobertura-def]{Simplexos singulares subordinados a uma cobertura};\\
    
\end{titlemize}

A sequência de Mayer-Vietoris é uma das mais poderosas ferramentas para o cálculo da homologia de um espaço topológico.
\begin{thm}
    Sejam $X$ um espaço topológico e $U$ e $V$ subconjuntos de $X$ tais que $int(U)\cup int(V)=X$. Considere as inclusões 
    \[i:U\cap V\hookrightarrow U,\;j:U\cap V\hookrightarrow V,\;k:U\hookrightarrow X,\;l:V\hookrightarrow X.\]
    Então, a sequência 
    \[...H_{n+1}(X)\xrightarrow{\delta} H_n(U\cap V)\xrightarrow{\Phi}H_n(U)\oplus H_n(V)\xrightarrow{\Psi} H_n(X)\rightarrow ...\]
    é exata, onde $\Phi:=i_*\oplus-j_*$ e $\Psi:=k_*+l_*$ 
\end{thm}

\begin{dem}
    Tomamos a cobertura $\mathcal{U}=\{U,V\}$ e consideramos, para cada $n\ge 0$, os homomorfismos $\phi:S_n(U\cap V)\rightarrow S_n(U)\oplus S_n(V)$ e $\psi:S_n(U)\oplus S_n(V)\rightarrow S_n^{\mathcal{U}}(X)$ dados por 
    \[\phi_n(c)=i_n(c)\oplus -j_n(c)\;\;\text{ e }\;\;\psi_n(c_1\oplus c_2)=k_n(c_1)+l_n(c_2).\]
    Para uma n-cadeia $c=a_1\sigma_1+...+a_r\sigma_r\in S_n(U\cap V)$, temos que 
    \[i_n(c)=a_1 (i\circ \sigma_1)+...+a_r(i\circ \sigma_r).\]
    Como $i\circ\sigma_1,...,i\circ \sigma_r$ são elementos da base do grupo abeliano livre $S_n(U)$, temos que $i_n(c)=0$ se, e somente se, $c=0$, o que mostra que $i_n$ é injetor. Analogamente, $j_n$ também é injetor. Portanto, $\phi_n$ é injetor.

    Pela definição, cada n-cadeia $c\in S_n^{\mathcal{U}}(X)$ é, da forma $c=c_1+c_2$ onde $c_1\in S_n (U)$ e $c_2\in S_n(V)$. Como $c=\psi_n(c_1\oplus c_2)$, provamos que $\psi_n$ é sobrejetor.

    Dada uma n-cadeia $c=a_1\sigma_1+...+a_r\sigma_r\in S_n(U\cap V)$, temos 
    \begin{align*}
        \psi_n\circ \phi_n(c)=& a_1(k\circ i\circ \sigma_1)+...+a_r(k\circ i\circ \sigma_r)\\
        &- (a_1(l\circ j\circ \sigma_1)+...+a_r (l\circ j\circ \sigma_r)).
    \end{align*}
    Como para cada n-simplexo singular $\sigma_i$, as composições $k\circ i\circ \sigma_i$ e $l\circ j \circ \sigma_i$ são iguais em $X$, obtemos $\psi_n\circ\phi_n(c)=0$. Portanto $\text{Im}(\phi_n)\subseteq \text{Ker}(\psi_n)$.

    Por outro lado, se as n-cadeias $c_1=a_1\sigma_1+...+ a_r\sigma_r \in S_n(U)$ e $c_2=a_1'\sigma_1'+...+a_s' \sigma_s'\in S_n(V)$ são não nulas tais que $c_1\oplus c_2\in \text{Ker}(\psi_n)$, então 
    \[a_1(k\circ \sigma_1)+...+a_r (k\circ \sigma_r)=-(a_1'(l\circ \sigma_1')+...+a_s'(l\circ \sigma_s')).\]
    Como só há uma única expressão de cada elemento do grupo abeliano livre $S_n(X)$ como combinação linear dos n-simplexos singulares, $r=s$ e, a menos de reordenação dos índices, $\sigma_t=\sigma_{t}'$ e $a_t'=-a_t$ para cada $1\le t\le r=s$. Dessa forma, $c_1,c_2\in S_n(U\cap V)$ e $c_2=-c_1$. Isso mostra que $c_1\oplus c_2=\phi_n (c_1)$, o que implica que $\text{Ker}(\psi_n)\subseteq \text{Im}(\phi_n)$.

    Pelos resultados obtidos acima, temos a sequência 
    \[0\rightarrow S_n (U\cap V)\xrightarrow{\phi_n} S_n (U)\oplus S_n(V)\xrightarrow{\psi_n} S_n^{\mathcal{U}}(X)\rightarrow 0\]
    é exata para cada $n\ge 0$. Logo a
    sequência dos complexos de cadeias 
    \[0\rightarrow S (U\cap V)_*\xrightarrow{\phi} S (U)_*\oplus S(V)_*\xrightarrow{\psi} S^{\mathcal{U}}(X)_*\rightarrow 0\]
    é exata, onde $\phi=(\phi_n)_{n\ge 0}$ e $\psi=(\psi_n)_{n\ge 0}$.

    Pelo Teorema \ref{sequencia-exata-longa-induzida-prop}, essa sequência induz a sequência 
    \[...\rightarrow H_{n+1}^\mathcal{U}(X)\xrightarrow{\delta} H_n(U\cap V)\xrightarrow{\phi_*}H_n(U)\oplus H_n(V)\xrightarrow{\psi_*} H^\mathcal{U}_n(X)\rightarrow ...\;.\]
    Como $\Phi=\phi_*$ e $\Psi=\psi_*$, a proposição \ref{simplexos-singulares-subordinados-a-uma-cobertura-def} garante que a sequência do enunciado do teorema 
    \[...H_{n+1}(X)\xrightarrow{\delta} H_n(U\cap V)\xrightarrow{\Phi}H_n(U)\oplus H_n(V)\xrightarrow{\Psi} H_n(X)\rightarrow ...\]
    é exata.
\end{dem}
Pela definição do homomorfismo conectante e pelo teorema \ref{sequencia-exata-longa-induzida-prop}, podemos caracterizar o homomorfismo $\delta$ na seguinte forma: uma classe de homologia $\overline{z}\in H_n (X)$ é da forma $\overline{z}=z+B_n(X)\in H_n(X)$, com o n-ciclo $z$ escrito como a soma $z=z_1+z_2$ de uma n-cadeia $z_1$ em $U$ e uma n-cadeia $z_2$ em $V$ (pelo proposição \ref{simplexos-singulares-subordinados-a-uma-cobertura-def}). Como $z$ é um n-ciclo, temos que $\partial z_1=-\partial z_2$ são (n-1)-ciclos em $U\cap V$. Então, 
\[\delta(\overline{z})=\partial z_1+B_{n-1}(U\cap V).\]

Sob certas condições, uma função contínua induz uma aplicação natural entre sequências de Mayer-Vietoris.

\begin{prop}
    Sejam $X, X'$ espaços topológicos decompostos como $X=int(U)\cup int(V)$ e $X'=int(U')\cup int(V')$. Seja $f:X\rightarrow X'$ uma função contínua tal que $f(U)\subseteq U'$ e $f(V)\subseteq V'$. Então, é comutativo o diagrama seguinte
    % https://q.uiver.app/#q=WzAsOCxbMCwwLCJIX3tuKzF9KFgpIl0sWzEsMCwiSF9uKFVcXGNhcCBWKSJdLFsyLDAsIkhfbihVKVxcb3BsdXMgSF9uKFYpIl0sWzMsMCwiSF9uKFgpIl0sWzAsMSwiSF97bisxfShYJykiXSxbMSwxLCJIX24oVSdcXGNhcCBWJykiXSxbMiwxLCJIX24oVScpXFxvcGx1cyBIX24oVicpIl0sWzMsMSwiSF9uKFgnKSJdLFswLDEsIlxcZGVsdGEiXSxbMSwyLCJcXFBoaSJdLFsyLDMsIlxcUHNpIl0sWzAsNCwiZl8qIiwyXSxbMSw1LCJmfF8qIiwyXSxbNCw1LCJcXGRlbHRhJyIsMl0sWzUsNiwiXFxQaGknIiwyXSxbNiw3LCJcXFBzaSciLDJdLFszLDcsImZfKiJdLFsyLDYsImZ8XypcXG9wbHVzIGZ8XyoiXV0=
\[\begin{tikzcd}
	{H_{n+1}(X)} & {H_n(U\cap V)} & {H_n(U)\oplus H_n(V)} & {H_n(X)} \\
	{H_{n+1}(X')} & {H_n(U'\cap V')} & {H_n(U')\oplus H_n(V')} & {H_n(X')}
	\arrow["\delta", from=1-1, to=1-2]
	\arrow["{f_*}"', from=1-1, to=2-1]
	\arrow["\Phi", from=1-2, to=1-3]
	\arrow["{f|_*}"', from=1-2, to=2-2]
	\arrow["\Psi", from=1-3, to=1-4]
	\arrow["{f|_*\oplus f|_*}", from=1-3, to=2-3]
	\arrow["{f_*}", from=1-4, to=2-4]
	\arrow["{\delta'}"', from=2-1, to=2-2]
	\arrow["{\Phi'}"', from=2-2, to=2-3]
	\arrow["{\Psi'}"', from=2-3, to=2-4]
\end{tikzcd}\]
    onde cada linha é um trecho da sequência de Mayer-Vietoris correspondente, e $f|$ denota a restrição de $f$ em domínio associado.
\end{prop}

\begin{dem}
    Como $i'\circ f|=f|\circ i$, $j'\circ f|=f|\circ j$, $f\circ k=k'\circ f|$ e $f\circ l=l'\circ f|$, temos que $i'_*\circ f|_*=f|_*\circ i_*$, $j'_*\circ f|_*=f|_*\circ j_*$, $f_*\circ k_*=k'_*\circ f|_*$ e $f_*\circ l_*=l'_*\circ f|_*$. Isso implica que os quadrados do centro e da direita são comutativos.

    Agora, vamos provar que o quadrado à esquerda também é comutativo. Seja $\overline{z}=z+B_{n+1}(X)\in H_{n+1}(X)$, pela observação acima, o $z$ pode ser escolhido como a soma $z=z_1+z_2$ de uma (n+1)-cadeia $z_1$ em $U$ e uma (n+1)-cadeia $z_2$ em V. Como $f:X\rightarrow X'$ induz uma aplicação de cadeias, temos 
    \begin{align*}
        f|_*\circ \delta(\overline{z}) & =f|_*(\partial z_1+B_{n}(U\cap V))=f(\partial z_1)+B_n(U'\cap V')\\
        &=\partial f(z_1)+B_n(U'\cap V')=\delta'(f(z_1)+f(z_2)+B_{n+1}(X'))\\
        &=\delta'(f(z)+B_{n+1}(X'))=\delta'(f_*(\overline{z})).
    \end{align*}
    Isso mostra que o quadrado à esquerda é comutativo.
\end{dem}
\begin{titlemize}{Lista de consequências}
    \item \hyperref[homologia-singular-de-S1-prop]{Homologia singular da circunferência};\\
    \item \hyperref[sequencia-exata-da-colagem-prop]{Sequência exata da colagem};\\
    \item \hyperref[grau-da-reflexao-prop]{Grau da reflexão}
	%\item \hyperref[]{}
\end{titlemize}

\subsection{Homologia singular da circunferência} %afirmação aqui significa teorema/proposição/colorário/lema
\label{homologia-singular-de-S1-prop}
\begin{titlemize}{Lista de dependências}
    \item \hyperref[sequencia-exata-def]{Sequência exata};\\
    \item \hyperref[homomorfismo-conectante-def]{Homomorfismo conectante};\\
    \item \hyperref[homologia-singular-def]{Homologia singular};\\
    \item \hyperref[homomorfismo-de-homologias-singulares-induzido-prop]{Homomorfismo de homologias singulares induzido};\\
    \item \hyperref[homologia-singular-de-um-ponto-prop]{Homologia singular de um ponto};\\
    \item \hyperref[0-esimo-grupo-de-homologia-de-espaco-zero-conexo-prop]{0-ésimo grupo de homologia singular de um espaço 0-conexo};\\
    \item \hyperref[homologia-singular-de-um-espaco-contratil-prop]{Homologia singular de um espaço contrátil};\\
    \item \hyperref[sequencia-de-mayer-vietoris-prop]{Sequência de Mayer-Vietoris}.

    
    
\end{titlemize}

\begin{prop}
    O n-ésimo grupo de homologia da circunferência é igual a 
    \begin{align*}
        H_n(\mathbb{S}^1)\cong\begin{cases}
            \mathbb{Z}&\text{se }n=0,1\\
            0&\text{se }n>1.
        \end{cases}
    \end{align*}
\end{prop}

\begin{proof}
    Denotamos os polos norte e sul de $\mathbb{S}^1\subseteq \mathbb{R}^2$ por $pn=(0,1)$ e $ps=(0,-1)$ respectivamente. Tomamos os abertos $U=\mathbb{S}^1\setminus \{ps\}$ e $V=\mathbb{S}^1\setminus \{pn\}$, cuja união $U\cup V=\mathbb{S}^1$. Pelo Teorema \ref{sequencia-de-mayer-vietoris-prop}, a sequência de Mayer-Vietoris
    \[...H_{n+1}(\mathbb{S}^1)\xrightarrow{\delta} H_n(U\cap V)\xrightarrow{\Phi}H_n(U)\oplus H_n(V)\xrightarrow{\Psi} H_n(\mathbb{S}^1)\rightarrow ...\]
    é exata.

    Os abertos $U$ e $V$ são ambos contráteis e, além disso, existe uma equivalência de homotopia sobrejetora $r:U\cap V\rightarrow\{q_1,q_2\}$, onde os pontos $q_1=(-1,0)$ e $q_2=(1,0)$. Dessa forma, os grupos de homologias $H_n(U),H_n(V)$ e $H_n(U\cup V)$ são todos triviais para $n\ge 1$, enquanto que $H_0(U)\cong \mathbb{Z}\cong H_0 (V)$ e $H_0 (U\cap V)\cong \mathbb{Z}\oplus \mathbb{Z}$.

    Como $\mathbb{S}^1$ é 0-conexo, segue que $H_0(\mathbb{S}^1)\cong \mathbb{Z}$.

    Para $n\ge 2$, pela sequência de Mayer-Vietoris, o trecho  
    \[0\rightarrow H_n (\mathbb{S}^1)\rightarrow 0\]
    é exata, consequentemente, $H_n(\mathbb{S}^1)=0.$

    O grupo $H_1(\mathbb{S}^1)$ aparece no trecho 
    \[0\rightarrow H_1(\mathbb{S}^1)\xrightarrow{\delta} H_0 (U\cap V)\xrightarrow{\Phi} H_0(U)\oplus H_0 (V),\]
    onde $\Phi=i_*\oplus -j_*$, sendo $i:U\cap V\hookrightarrow U$ e $j:U\cap V\hookrightarrow V$ as inclusões. Os pontos $q_1$ e $q_2$ podem vistos como 0-ciclos, representam geradores do grupo $H_0(U\cap V)\cong \mathbb{Z}\oplus \mathbb{Z}$. Por outro lado, esses elementos também são geradores tanto de $H_0 (U)$ quanto de $H_0(V)$, pois, em ambos os grupos, eles representam a mesma classe. Isso ocorre porque $q_1-q_2$ é o bordo de um arco hemisféricos da circunferência de $q_2$ para $q_1$ passando por cima (ou orientados no sentido anti-horário) em $U$ e de um arco hemisféricos passando por baixo (orientados no sentido horário) em $V$. Logo, existe um $q\in U\cap V$ tal que $q_1,q_2\in [q]_U=q+B_0 (U)$ e $q_1,q_2\in [q]_V=q+B_0 (V)$. Portanto, $\Phi$ é dado por 
    \begin{align*}
        \Phi(q_1+B_0(U\cap V))=[q]_U\oplus-[q]_V;\\
        \Phi(q_2+B_0(U\cap V))=[q]_U\oplus -[q]_V.
    \end{align*}
    Resulta que $\text{Ker}(\Phi)\cong \mathbb{Z}$, correspondendo ao subgrupo $\langle (1,-1) \rangle\subseteq \mathbb{Z}\oplus \mathbb{Z}\cong H_0(U\cap V)$. Por exatidão da última sequência, obtemos 
    \[H_1(\mathbb{S}^1)\cong \text{Im}(\delta)=\text{Ker}(\Phi)\cong \mathbb{Z}.\]
    Portanto, $H_n(\mathbb{S}^1)\cong \mathbb{Z}$ para $n=0$ ou $n=1$, e $H_n(\mathbb{S}^1)=0$ para todo $n\ge 2$.
\end{proof}
Para identificar um 1-ciclo $z_1\in Z_1(\mathbb{S}^1)$ cuja classe de homologia $\overline{z_1}=z_1+B_1 (\mathbb{S}^1)$ seja um gerador de $H_1(\mathbb{S}^1)$, podemos escolher $z_1$ como a soma de 1-cadeias, ou seja $z_1=c_1+c_2$, com $c_1\in S_1(U)$ e $c_2\in S_1(V)$. Como $\delta(\overline{z_1})=\partial c_1+B_0(U\cap V)$ (a observação no final de \ref{sequencia-de-mayer-vietoris-prop}), pelos isomorfismos $H_1(\mathbb{S}^1)\cong\text{Im}(\delta))\cong \text{Ker}(\Phi)$, temos que $\partial c_1=q_1-q_2=-\partial c_2$. Nessas condições, $c_1$ e $c_2$ são os 1-simplexos singulares correspondentes aos arcos hemisféricos da circunferência, orientados no sentido anti-horário.

\begin{titlemize}{Lista de consequências}
    \item \hyperref[grupo-de-homologia-singular-de-n-esfera-prop]{Grupo de homologia singular de n-esfera}.\\
	%\item \hyperref[]{}
\end{titlemize}

\subsection{Sequência exata da colagem} %afirmação aqui significa teorema/proposição/colorário/lema
\label{sequencia-exata-da-colagem-prop}
\begin{titlemize}{Lista de dependências}
    \item \hyperref[sequencia-exata-def]{Sequência exata};\\
    \item \hyperref[homomorfismo-conectante-def]{Homomorfismo conectante};\\
    \item \hyperref[homologia-singular-def]{Homologia singular};\\
    \item \hyperref[homomorfismo-de-homologias-singulares-induzido-prop]{Homomorfismo de homologias singulares induzido};\\
    \item \hyperref[homologia-singular-de-um-ponto-prop]{Homologia singular de um ponto};\\
    \item \hyperref[0-esimo-grupo-de-homologia-de-espaco-zero-conexo-prop]{0-ésimo grupo de homologia singular de um espaço 0-conexo};\\
    \item \hyperref[0-conexo-e-homomorfismo-de-homologia-induzido-prop]{0-conexo e homomorfismo de homologia induzido};\\
    \item \hyperref[homologia-singular-de-um-espaco-contratil-prop]{Homologia singular de um espaço contrátil};\\
    \item \hyperref[sequencia-de-mayer-vietoris-prop]{Sequência de Mayer-Vietoris};\\
    \item \hyperref[colagem-de-n-celula-def]{Colagem de n-célula}

    
    
\end{titlemize}

Seja $X$ um espaço Hausdorff. E Seja $f:\mathbb{S}^{n-1}\rightarrow X$ uma função contínua, onde $n\ge 2$. Nesse caso, o espaço $X_f$ também é Hausdorff. Sejam $i:X\hookrightarrow D^n\sqcup X$ e $j:D^n\hookrightarrow D^n$ as inclusões naturais e seja $\pi:D^n\sqcup X\rightarrow X_f$ a função quociente. Temos:
\begin{itemize}
    \item Como os pontos de $Y$ não se relacionam entre si, exceto consigo mesmos, a composição $l=\pi\circ i:X\rightarrow X_f$ induz um homeomorfismo entre $X$ e $l(X)\subset X_f$.
    \item Como a esfera $\mathbb{S}^{n-1}$ é conexa por caminho , a imagem de $f$ está contida em uma componente por caminho de $X$. Isso garante que as componentes por caminhos de $X$ e de $X_f$ estão em correspondência bijetiva. Pelo Corolário final do \ref{0-esimo-grupo-de-homologia-de-espaco-zero-conexo-prop}, $l_*:H_0 (X)\rightarrow H_0(X_f)$ é um isomorfismo.
    \item Como os pontos da bola aberta $B=int(D^n)$ não se relacionam entre si, exceto consigo mesmo, a composição $k=\pi\circ j|_B: B\rightarrow X_f$ induz um homeomorfismo entre $B$ e $k(B)\subseteq X_f$.
    \item $k(B)$ é aberto e $l(X)$ é fechado em $X_f$, pois $l(X)$ é compacto e $X_f$ é Hausdorff. Além disso, ambos são disjuntos, a fronteira de $k(B)$ é $l(f(\mathbb{S}^{n-1}))$, e o espaço de colagem $X_f=k(B)\sqcup l(X)$.
    \item Seja $\pi(0)\in X_f$ a imagem por $\pi$ do origem $0\in D^n$ e seja $\rho: D^n\setminus 0\rightarrow \mathbb{S}^{n-1}$ o retrato por deformação radial dado por $\rho(x)=x/||x||$. Então, está bem definida e é um retrato por deformação a função $\mathbf{r}:X_f\setminus \pi(0)\rightarrow l(X)$ dada por 
    \begin{align*}
        \textbf{r}(x)=\begin{cases}
            x&\text{ se }x\in l(X);\\
            \pi(\rho(k^{-1}(x))) &\text{ se }x\in k(B).
        \end{cases}
    \end{align*}
\end{itemize}

Agora, definimos $U=X_f\setminus \pi(0)$ e $V=X_f\setminus l(X)$. Como $\pi(0)$ e $l(Y)$ são fechados em $X_f$, $U$ e $V$ são abertos em $X_f$. Pelas definições, a união $U\sqcup V=X_f$. Além disso:
\begin{itemize}
    \item $U$ se deforma sobre $l(X)$ que é homeomorfo a $X$.
    \item O aberto $V=k(B)$ se deforma sobre o ponto $\pi(0)$, ou seja, $V$ é contrátil.
    \item $U\cap V$ é homeomorfo a $B\setminus \{0\}$ e, portanto, deforma sobre $k(S^{n-1}_{1/2})$, onde $S^{n-1}_{1/2}\subseteq B$ é a $(n-1)$-esfera de centro 0 e raio $1/2$.
\end{itemize}
Por Teorema \ref{sequencia-de-mayer-vietoris-prop}, a sequência de Mayer-Vietoris da decomposição $X_f=U\sqcup V$ 
\[...H_{m+1}(X_f)\xrightarrow{\delta} H_m(U\cap V)\xrightarrow{\Phi}H_m(U)\oplus H_m(V)\xrightarrow{\Psi} H_m(X_f)\rightarrow ...\]
é exata. Pelas observações acima, temos $H_m(U)\cong H_m(X)$ e $H_m (V)\cong H_m(\{0\})$ para todo $m\ge 0$. Por causa disso, para $m\ge 1$, o grupo $H_m(U)\oplus H_m(V)$ pode ser substituído por $H_m(X)$, e o homomorfismo $\Psi$ pode ser substituído por $l_*:H_m(X)\rightarrow H_m (X_f)$. Por outro lado, $H_0(U)\oplus H_0 (V)\cong H_0(X)\oplus \mathbb{Z}$ e $l_*:H_0(X)\rightarrow H_0 (X_f)$ é um isomorfismo. Logo, $\Psi: H_0(U)\oplus H_0 (V)\rightarrow H_0 (X_f)$ é sobrejetor e pode ser reescrito como $\Psi:H_0(X)\oplus \mathbb{Z}\rightarrow H_0(X_f)$.

Para $\Phi$, o contradomínio dele é isomorfo a $H_m(X)\oplus H_m (\pi(0))$, enquanto que o domínio dele $H_m(U\cap V)$ é isomorfo a $H_m(S^{n-1}_{1/2})$. Visto que $S^{n-1}_{1/2}$ é uma $(n-1)$-esfera e $V$ é contrátil, para todo $m>0$, o homomorfismo $\Phi:H_m(U\cap V)\rightarrow H_m(U)\oplus H_m(V)$ pode ser substituído na sequência de Mayer-Vietoris, a menos de isomorfismo, por $\Phi:H_m(\mathbb{S}^{n-1})\rightarrow H_m(X)$. Uma vez que $\mathbb{S}^{n-1}$ é 0-conexo, temos $\Phi:H_0(\mathbb{S}^{n-1})\rightarrow H_0(X)\oplus\mathbb{Z}$ é injetor (\ref{0-conexo-e-homomorfismo-de-homologia-induzido-prop}).
    
Chegamos à conclusão de que: 
\begin{prop}
    A sequência 
    \begin{align*}
        ...H_{m+1}(X_f)&\xrightarrow{\delta} H_m(\mathbb{S}^{n-1})\xrightarrow{\Phi}H_m(X)\xrightarrow{l_*} H_m(X_f)\rightarrow ...\\
        ...& \rightarrow H_0 (\mathbb{S}^{n-1})\xrightarrow{\Phi} H_0(X)\oplus \mathbb{Z}\xrightarrow{\Psi} H_0(X_f)\rightarrow 0
    \end{align*}
    é exata.
\end{prop}

Como $H_m(\mathbb{S}^{n-1})=0$ para todo $m\in\mathbb{N}-{0,n-1}$ (veremos daqui a pouco em \ref{grupo-de-homologia-singular-de-n-esfera-prop}), temos que $\Phi:H_m(\mathbb{S}^{n-1})\rightarrow H_m(X)$ é nulo para todo $m$ positivo tal que $m\ne n-1$.

\begin{titlemize}{Lista de consequências}
    \item \hyperref[grupo-de-homologia-singular-de-n-esfera-prop]{Grupo de homologia singular de n-esfera}.\\
	%\item \hyperref[]{}
\end{titlemize}

\subsection{Grupo de homologia singular de n-esfera} %afirmação aqui significa teorema/proposição/colorário/lema
\label{grupo-de-homologia-singular-de-n-esfera-prop}
\begin{titlemize}{Lista de dependências}
    \item \hyperref[sequencia-exata-def]{Sequência exata};\\
    \item \hyperref[homomorfismo-conectante-def]{Homomorfismo conectante};\\
    \item \hyperref[homologia-singular-def]{Homologia singular};\\
    \item \hyperref[homomorfismo-de-homologias-singulares-induzido-prop]{Homomorfismo de homologias singulares induzido};\\
    \item \hyperref[homologia-singular-de-um-ponto-prop]{Homologia singular de um ponto};\\
    \item \hyperref[0-esimo-grupo-de-homologia-de-espaco-zero-conexo-prop]{0-ésimo grupo de homologia singular de um espaço 0-conexo};\\
    \item \hyperref[0-conexo-e-homomorfismo-de-homologia-induzido-prop]{0-conexo e homomorfismo de homologia induzido};\\
    \item \hyperref[homologia-singular-de-um-espaco-contratil-prop]{Homologia singular de um espaço contrátil};\\
    \item \hyperref[homologia-singular-de-S1-prop]{Homologia singular da circunferência};\\
    \item \hyperref[sequencia-de-mayer-vietoris-prop]{Sequência de Mayer-Vietoris};\\
    \item \hyperref[colagem-de-n-celula-def]{Colagem de n-célula};\\
    \item \hyperref[colagem-de-um-disco-com-um-ponto-ex]{Colagem de um disco com um ponto};\\
    \item \hyperref[sequencia-exata-da-colagem-prop]{Sequência exata de colagem}
\end{titlemize}

\begin{prop}
    Seja $n$ um inteiro estritamente maior que $0$. O m-ésimo grupo de homologia singular de $\mathbb{S}^n$ é igual a  
    \begin{align*}
        H_m(\mathbb{S}^n)\cong\begin{cases}
            \mathbb{Z}&\text{ se }m=0\text{ ou }m=n\\
            0&\text{ caso contrário.}
        \end{cases}
    \end{align*}
\end{prop}

\begin{dem}
    Já vimos o caso $n=1$ em \ref{homologia-singular-de-S1-prop}. Para $n\ge 2$, consideramos a esfera $\mathbb{S}^n$ como o espaço de colagem $\mathbb{S}^n=D^n\cup_f\{x\}$, onde $f:\mathbb{S}^{n-1}\rightarrow \{x\}$ é a função constante (veja detalhe em \ref{colagem-de-um-disco-com-um-ponto-ex}). Como $\mathbb{S}^n$ é 0-conexo, $H_0(\mathbb{S}^n)\cong \mathbb{Z}$.
    
    Para $m\ge 2$, visto que $H_m(\{x\})$ e $H_{m-1}(\{x\})$ são ambos nulos, pela sequência exata da colagem, é exata a sequência 
    \[0\rightarrow H_m(\mathbb{S}^n)\xrightarrow{\delta} H_{m-1}(\mathbb{S}^{n-1})\rightarrow 0.\]
    Isso mostra que, para $m\ge 2$, $H_m(\mathbb{S}^n)\cong H_{m-1}(\mathbb{S}^{n-1}).$

    Novamente, pela sequência exata de colagem, é exata a sequência 
    \[0\rightarrow H_1(\mathbb{S}^n)\xrightarrow{\delta_1} H_0(\mathbb{S}^{n-1})\xrightarrow{\Phi} H_0(\{x\})\oplus \mathbb{Z}.\]
    Como a esfera $\mathbb{S}^{n-1}$ é 0-conexo, temos que $\Phi$ é um monomorfismo (\ref{0-conexo-e-homomorfismo-de-homologia-induzido-prop}) e, consequentemente, $\delta_1$ é a função nula, mas $\delta_1$ é um injetor. Isso mostra que $H_1(\mathbb{S}^n)=0$ para todo $n\ge 2$.

    Assim, como $H_1(\mathbb{S}^2)=0$ e $H_m(\mathbb{S}^2)=H_{m-1}(\mathbb{S}^1)$ para todo $m\ge 2$, obtemos 
    \begin{align*}
        H_m(\mathbb{S}^2)\cong\begin{cases}
            \mathbb{Z}&\text{ se }m=0,2\\
            0&\text{ caso contrário.}
        \end{cases}
    \end{align*}
    Processando indutivamente, obtemos 
    \begin{align*}
        H_m(\mathbb{S}^n)\cong\begin{cases}
            \mathbb{Z}&\text{ se }m=0,n\\
            0&\text{ caso contrário.}
        \end{cases}
    \end{align*}
\end{dem}

\begin{titlemize}{Lista de consequências}
    \item \hyperref[teorema-de-invariancia-de-dimensao-de-esfera-prop]{Teorema de invariância de dimensão de esfera};\\
	\item \hyperref[teorema-de-ponto-fixo-de-brouwer-geral-prop]{Teorema de ponto fixo de Brouwer (versão geral)};\\
    \item \hyperref[grau-de-funcoes-em-esferas-def]{Grau de funções}.
    
\end{titlemize}

\subsection{Teorema de invariância de dimensão de esfera} %afirmação aqui significa teorema/proposição/colorário/lema
\label{teorema-de-invariancia-de-dimensao-de-esfera-prop}
\begin{titlemize}{Lista de dependências}
    \item \hyperref[homologia-singular-def]{Homologia singular};\\
    \item \hyperref[homomorfismo-de-homologias-singulares-induzido-prop]{Homomorfismo de homologias singulares induzido};\\
    \item \hyperref[homologia-singular-de-S1-prop]{Homologia singular da circunferência};\\
    \item \hyperref[grupo-de-homologia-singular-de-n-esfera-prop]{Grupo de homologia singular de n-esfera}.
\end{titlemize}

\begin{prop}
    A n-esfera $\mathbb{S}^n$ é homeomorfa à m-esfera $\mathbb{S}^m$ se, e somente se, $n=m$.
\end{prop}

\begin{dem}
    Se $n=m$, então $\mathbb{S}^n=\mathbb{S}^m$.

    Para a outra direção, provamos por contra-positiva. Suponha que $0<n\ne m$. Como $H_n(\mathbb{S}^n)\cong \mathbb{Z}$ e $H_n(\mathbb{S}^m)=0$, concluímos que $\mathbb{S}^n$ não é homeomorfa a $\mathbb{S}^m$, como queríamos.
\end{dem}

%\begin{titlemize}{Lista de consequências}
    %\item %\hyperref[homomorfismo-de-homologias-singulares-induzido-prop]{Homomorfismo de homologias singulares induzido}.\\
	%\item \hyperref[]{}
%\end{titlemize}

\subsection{Grau de funções em esferas} %afirmação aqui significa teorema/proposição/colorário/lema
\label{grau-de-funcoes-em-esferas-def}
\begin{titlemize}{Lista de dependências}
    \item \hyperref[homologia-singular-def]{Homologia singular};\\
    \item \hyperref[homomorfismo-de-homologias-singulares-induzido-prop]{Homomorfismo de homologias singulares induzido};\\
    \item \hyperref[homologia-singular-de-S1-prop]{Homologia singular da circunferência};\\
    \item \hyperref[grupo-de-homologia-singular-de-n-esfera-prop]{Grupo de homologia singular de n-esfera}.
\end{titlemize}

O conceito de grau de funções contínuas de esferas, devido a Brouwer, é um ferramento poderoso, que produz resultados importantes.

\begin{defi}
    Seja $f:\mathbb{S}^n\rightarrow \mathbb{S}^n$ uma função contínua, com $n>0$. Consideramos o homomorfismo de homologias singulares induzido
    \[f_*:H_n(\mathbb{S}^n)\rightarrow H_n(\mathbb{S}^n).\]
    Como $H_n(\mathbb{S}^n)\cong \mathbb{Z}$, o homomorfismo $f_*$ é necessariamente da forma. $f_*(\alpha)=k\alpha$. O inteiro $k$ é chamado \textbf{grau da função} $f$ e é denotado por $deg(f)$.
\end{defi}

%\begin{titlemize}{Lista de consequências}
    %\item %\hyperref[homomorfismo-de-homologias-singulares-induzido-prop]{Homomorfismo de homologias singulares induzido}.\\
	%\item \hyperref[]{}
%\end{titlemize}

\subsection{Propriedades de grau de funções de esferas} %afirmação aqui significa teorema/proposição/colorário/lema
\label{propriedades-de-grau-de-funções-prop}
\begin{titlemize}{Lista de dependências}
    \item \hyperref[homologia-singular-def]{Homologia singular};\\
    \item \hyperref[homomorfismo-de-homologias-singulares-induzido-prop]{Homomorfismo de homologias singulares induzido};\\
    \item \hyperref[homologia-singular-de-um-espaco-contratil-prop]{Homologia singular de um espaço contrátil};\\
    \item \hyperref[homologia-singular-de-S1-prop]{Homologia singular da circunferência};\\
    \item \hyperref[grupo-de-homologia-singular-de-n-esfera-prop]{Grupo de homologia singular de n-esfera};\\
    \item \hyperref[grau-de-funcoes-em-esferas-def]{Grau de funções em esferas}.
\end{titlemize}

\begin{prop}
    O grau $deg(-)$ satisfaz as seguintes propriedades:
    \begin{enumerate}
        \item A função identidade $id:\mathbb{S}^n\rightarrow \mathbb{S}^n$ tem grau 1.
        \item Se $f,g:\mathbb{S}^n\rightarrow\mathbb{S}^n$ são contínuas, então $deg(f\circ g)=deg(f)deg(g)$.
        \item Se $f,g:\mathbb{S}^n\rightarrow\mathbb{S}^n$ são homotópicas, então $deg(f)=deg(g)$.
        \item Se $f:\mathbb{S}^n\rightarrow\mathbb{S}^n$ é uma equivalência de homotopia, então $deg(f)=+1\;\text{ ou }-1$.
        \item Se $f:\mathbb{S}^n\rightarrow\mathbb{S}^n$ é contínua e não sobrejetora, então $deg(f)=0$.
    \end{enumerate}
\end{prop}

\begin{dem}
    Os itens 1), 2) e 3) seguem dos resultados em \ref{homomorfismo-de-homologias-singulares-induzido-prop}. O item 4) segue dos itens 1), 2) e 3). 

    Agora, provamos o item 5): Como $f$ não é sobrejetor, podemos escolher um ponto $x\in \mathbb{S}^n\setminus f(\mathbb{S}^n)$. Então, $f$ fatora-se como uma composição \[\mathbb{S}^n\xrightarrow{g} \mathbb{S}^n\setminus\{x\}\hookrightarrow \mathbb{S}^n,\]
    onde denotamos a inclusão $\mathbb{S}^n\setminus\{x\}\hookrightarrow \mathbb{S}^n$ por $i$.
    Como $\mathbb{S}^n\setminus\{x\}$ é contrátil, obtemos $H_n(\mathbb{S}^n\setminus \{x\})=0$. Isso implica que $f_*=i_*\circ g_*=0$.
\end{dem}

%\begin{titlemize}{Lista de consequências}
    %\item %\hyperref[homomorfismo-de-homologias-singulares-induzido-prop]{Homomorfismo de homologias singulares induzido}.\\
	%\item \hyperref[]{}
%\end{titlemize}

\subsection{Grau da reflexão} %afirmação aqui significa teorema/proposição/colorário/lema
\label{grau-da-reflexao-prop}
\begin{titlemize}{Lista de dependências}
    \item \hyperref[homologia-singular-def]{Homologia singular};\\
    \item \hyperref[homomorfismo-de-homologias-singulares-induzido-prop]{Homomorfismo de homologias singulares induzido};\\
    \item \hyperref[homologia-singular-de-um-espaco-contratil-prop]{Homologia singular de um espaço contrátil};\\
    \item \hyperref[sequencia-de-mayer-vietoris-prop]{Sequência de Mayer-Vietoris};\\
    \item \hyperref[homologia-singular-de-S1-prop]{Homologia singular da circunferência};\\
    \item \hyperref[grupo-de-homologia-singular-de-n-esfera-prop]{Grupo de homologia singular de n-esfera};\\
    \item \hyperref[grau-de-funcoes-em-esferas-def]{Grau de funções em esferas};\\
    \item \hyperref[propriedades-de-grau-de-funções-prop]{Propriedade de grau de funções em esferas}
\end{titlemize}

\begin{defi}
    Uma \textbf{reflexão} é uma função de $r:\mathbb{S}^n\rightarrow \mathbb{S}^n$ que troca o sinal de uma coordenada. Quando a troca de sinal ocorrer na i-ésima coordenada, diremos que $r$ é uma \textbf{reflexão da i-ésima coordenada}. 
\end{defi}

\begin{lemma}
    A reflexão da primeira coordenada na esfera $\mathbb{S}^n$ tem grau $-1$.
\end{lemma}

\begin{dem}
    Provamos o lema por indução sobre a dimensão n.

    Base de indução: consideramos a reflexão $r:\mathbb{S}^1\rightarrow \mathbb{S}^1$ dada por $r(x,y)=(-x,y)$. Pelo exemplo \ref{homologia-singular-de-S1-prop}, $H_1(\mathbb{S}^1)$ é gerado pela classe de homologia do 1-ciclo $z_1=c_1+c_2$, onde $c_1$ e $c_2$ são os 1-simplexos singulares correspondentes aos arcos hemisféricos da circunferência, orientados no sentido anti-horário. A função $r$ mapeia $c_1$ a $-c_1$ e $c_2$ a $-c_2$. Isso mostra que $r\circ (z_1)=-z_1$ e, por conseguinte, o homomorfismo $r_*:H_1(\mathbb{S}^1)\rightarrow H_1(\mathbb{S}^1)$ tem grau $-1$.

    Passo de indução: Suponhamos que o resultado valha para a dimensão $n-1\ge 1$ e seja $r:\mathbb{S}^n\rightarrow \mathbb{S}^n$ a reflexão da primeira coordenada. Considerando a inclusão $e:\mathbb{S}^{n-1}\hookrightarrow \mathbb{S}^n$ dada por $(x_1,...,x_n)\mapsto (x_1,...,x_n,0)$ e considerando os abertos $U=\mathbb{S}^n\setminus\{(0,...,0,-1)\}$ e $V=\mathbb{S}^n\setminus \{(0,...,0,1)\}$. Note que a inclusão $i:\mathbb{S}^{n-1}\hookrightarrow U\cap V$ é uma equivalência de homotopia. Como o equador $\mathbb{S}^{n-1}$ é invariante por $r$ (i.e., $r(\mathbb{S}^{n-1})\subseteq\mathbb{S}^{n-1}$), a função restrição $r|:\mathbb{S}^{n-1}\rightarrow \mathbb{S}^{n-1}$ é uma reflexão da primeira coordenada bem definida. Pela hipótese de indução, $r|$ tem grau $-1$. Como $n\ge 2$ e $U$ e $V$ são contráteis, o seguinte trecho da sequência de Mayer-Vietoris 
    \[0=H_{n}(U)\oplus H_n(V)\rightarrow H_n(\mathbb{S}^n)\xrightarrow{\delta}H_{n-1}(U\cap V)\rightarrow H_{n-1}(U)\oplus H_{n-1}(V)=0\]
    é exata, o que mostra que o homomorfismo $\delta$ é um isomorfismo. Com isso, obtemos o diagrama comutativo seguinte 
    % https://q.uiver.app/#q=WzAsNixbMCwwLCJIX3tufShcXG1hdGhiYntTfV5uKSJdLFsxLDAsIkhfe24tMX0oVVxcY2FwIFYpIl0sWzIsMCwiSF9uKFxcbWF0aGJie1N9XntuLTF9KSJdLFswLDEsIkhfe259KFxcbWF0aGJie1N9Xm4pIl0sWzEsMSwiSF97bi0xfShVXFxjYXAgVikiXSxbMiwxLCJIX24oXFxtYXRoYmJ7U31ee24tMX0pIl0sWzAsMSwiXFxkZWx0YSIsMCx7InN0eWxlIjp7InRhaWwiOnsibmFtZSI6ImFycm93aGVhZCJ9fX1dLFswLDMsInJfKiIsMl0sWzEsNCwicnxfKiIsMl0sWzMsNCwiXFxkZWx0YSIsMix7InN0eWxlIjp7InRhaWwiOnsibmFtZSI6ImFycm93aGVhZCJ9fX1dLFsyLDUsInJ8XyoiXSxbMiwxLCJpXyoiLDIseyJzdHlsZSI6eyJ0YWlsIjp7Im5hbWUiOiJhcnJvd2hlYWQifX19XSxbNSw0LCJpXyoiLDAseyJzdHlsZSI6eyJ0YWlsIjp7Im5hbWUiOiJhcnJvd2hlYWQifX19XV0=
\[\begin{tikzcd}
	{H_{n}(\mathbb{S}^n)} & {H_{n-1}(U\cap V)} & {H_n(\mathbb{S}^{n-1})} \\
	{H_{n}(\mathbb{S}^n)} & {H_{n-1}(U\cap V)} & {H_n(\mathbb{S}^{n-1})}
	\arrow["\delta", tail reversed, from=1-1, to=1-2]
	\arrow["{r_*}"', from=1-1, to=2-1]
	\arrow["{r|_*}"', from=1-2, to=2-2]
	\arrow["{i_*}"', tail reversed, from=1-3, to=1-2]
	\arrow["{r|_*}", from=1-3, to=2-3]
	\arrow["\delta"', tail reversed, from=2-1, to=2-2]
	\arrow["{i_*}", tail reversed, from=2-3, to=2-2]
\end{tikzcd}\]
    onde flechas horizontais são isomorfismo.
    Seja $\overline{z}$ um gerador de $H_n(\mathbb{S}^n)$. Então, 
    \begin{align*}
        r_*(\overline{z})=&\delta^{-1}\circ r|_*\circ \delta(\overline{z})=\delta^{-1}\circ i_*\circ r|_*\circ i_*^{-1}\circ \delta(\overline{z})\\
        =& -\delta^{-1}\circ i_*\circ i_*^{-1}\circ \delta(\overline{z})=-\overline{z}.
    \end{align*}
    Isso demonstra que $deg(r)=-1$. Assim concluímos o passo de indução e a prova do lema.
\end{dem}

\begin{corol}
    Qualquer reflexão na esfera $\mathbb{S}^n$ tem grau -1.
\end{corol}

\begin{dem}
    Para cada $1\le i\le n+1$, seja $r_i: \mathbb{S}^n\rightarrow \mathbb{S}^n$ a reflexão da i-ésima coordenada. Pelo lema anterior, $deg(r_1)=-1$. Para $i\ge 2$, temos $r_i=h_i\circ r_1\circ h_i$, onde $h_i:\mathbb{S}^n\rightarrow \mathbb{S}^n$ é o homeomorfismo que permuta a primeira e a i-ésima coordenadas. Pela propriedades de grau de funções, segue que 
    \[deg(r_i)=deg(h_i)deg(r_1)deg(h_i)=deg(h_i)^2deg(r_1)=deg(r_1)=-1\]
\end{dem}

\begin{titlemize}{Lista de consequências}
    \item \hyperref[grau-de-antipoda-prop]{Grau de antípoda}.\\
	%\item \hyperref[]{}
\end{titlemize}

\subsection{Grau de antípoda} %afirmação aqui significa teorema/proposição/colorário/lema
\label{grau-de-antipoda-prop}
\begin{titlemize}{Lista de dependências}
    \item \hyperref[homologia-singular-def]{Homologia singular};\\
    \item \hyperref[homomorfismo-de-homologias-singulares-induzido-prop]{Homomorfismo de homologias singulares induzido};\\
    \item \hyperref[homologia-singular-de-S1-prop]{Homologia singular da circunferência};\\
    \item \hyperref[grupo-de-homologia-singular-de-n-esfera-prop]{Grupo de homologia singular de n-esfera};\\
    \item \hyperref[grau-de-funcoes-em-esferas-def]{Grau de funções em esferas};\\
    \item \hyperref[propriedades-de-grau-de-funções-prop]{Propriedade de grau de funções em esferas};\\
    \item \hyperref[grau-da-reflexao-prop]{Grau da reflexão}
\end{titlemize}

\begin{defi}
    Uma \textbf{antípoda} é uma função de $a:\mathbb{S}^n\rightarrow \mathbb{S}^n$ dada por $a(x)=-x$.
\end{defi}

\begin{lemma}
    A antípoda da esfera $\mathbb{S}^n$ tem grau $(-1)^{n+1}$
\end{lemma}

\begin{dem}
    Para cada $1\le i\le n+1$, seja $r_i: \mathbb{S}^n\rightarrow \mathbb{S}^n$ a reflexão da i-ésima coordenada. A antípoda fatora-se como a composição $a=r_1\circ ...\circ r_{n+1}$. Segue das propriedades de grau de funções em esferas que 
    \[deg(a)=deg(r_1)\cdot...\cdot deg(r_{n+1})=(-1)^{n+1}\]
\end{dem}

\begin{titlemize}{Lista de consequências}
    \item \hyperref[homomorfismo-de-homologias-singulares-induzido-prop]{Homomorfismo de homologias singulares induzido}.\\
	%\item \hyperref[]{}
\end{titlemize}

