\section{Homologia}
\label{homologia}

\begin{titlemize}{Lista de Dependências}
    \item \hyperref[homotopia]{Homotopia};\\ %assunto1 é o label onde o Assunto 1 aparece
    \item \hyperref[grupos-livres]{Grupos Livres};\\
    \item \hyperref[simplexo-def]{Simplexos};\\
\end{titlemize}

Em topologia algébrica, a homologia é a sequência de grupos de homologia associada a um espaço topológico. Esses grupos capturam de forma algébrica a ideia dos "buracos" em diferentes dimensões no espaço. Assim, a homologia é uma ferramenta fundamental para distinguir e classificar espaços topológicos.

\subsection{Complexo de cadeias}
\label{complexo-de-cadeias-def}

%\begin{titlemize}{Lista de dependências}
	%\item %\hyperref[homologia-simplicial-def]{Homologia Simplicial};\\ %'dependencia1' é o label onde o conceito Dependência 1 aparece (--à arrumar um padrão para referencias e labels--) 
% quantas dependências forem necessárias.
%\end{titlemize}

\begin{defi}[Complexo de cadeias]
	Um \textbf{Complexo de cadeias} é uma sequência $C_{-1}=0,C_0,C_1, C_2,...$ de grupos abelianos acompanhada de homomorfismos $d_n:C_n\rightarrow C_{n-1}$ para cada $n\ge 0$, tais que $d_{n}\circ d_{n+1}=0$. Denotamos esse complexo de cadeias por $C_{\bullet}$, e os homomorfismos $d_n$ são chamados de \textbf{diferenciais} de $C_\bullet$. A n-ésima \textbf{homologia} de $C_\bullet$ é definida por
    \[H_n(C_\bullet):=\frac{\text{Ker}(d_n)}{\text{Im}(d_{n+1})}\]
    Usualmente, $\text{Ker}(d_n)$ é chamado de grupo abeliano de $n$-ciclo em $C_\bullet$ e é denotado por $Z_n(C_\bullet)$. Por outro lado, $\text{Im}(d_{n+1})$ é chamada de grupo abeliano de $n$-bordos em $C_\bullet$ e é denotada por $B_n(C_\bullet)$.
\end{defi}

A homologia é a medida não numérica de quão diferentes $Z_n(C_\bullet)$ e $B_n(C_\bullet)$ são.

\begin{titlemize}{Lista de consequências}
    \item \hyperref[aplicacao-de-cadeias-def]{Aplicação de cadeias};\\
    \item \hyperref[homotopia-de-cadeias-def]{Homotopia de cadeia};\\
    \item \hyperref[homomorfismo-induzido-de-cadeias-prop]{Homomorfismo induzido de cadeias};\\
    \item \hyperref[equivalencia-de-homotopia-de-cadeias-def]{Equivalência de homotopia de cadeias};\\
    \item \hyperlink{homologia-simplicial-def}{Homologia simplicial};\\
    \item \hyperref[homologia-singular-def]{Homologia singular};\\
    \item \hyperref[homomorfismo-de-homologias-singulares-induzido-prop]{Homomorfismo de homologias singulares induzido}.
\end{titlemize}

\input{conteudo/aplicacao-de-cadeias-def}
\input{conteudo/homotopia-de-cadeias-def}
\input{conteudo/homomorfismo-induzido-de-cadeias-prop}
\input{conteudo/equivalencia-de-homotopia-de-cadeias-def}
\input{conteudo/sequencia-exata-def}
\input{conteudo/homomorfismo-conectante-def}
\input{conteudo/sequencia-exata-longa-induzida-prop}
\subsection{Homologia Simplicial}
\label{homologia-simplicial-def}
\begin{titlemize}{Lista de dependências}
    \item \hyperref[complexo-simplicial-def]{Complexos simpliciais};\\
	\item \hyperref[complexo-de-cadeias-def]{Complexo de cadeias};\\ 
    \item \hyperref[aplicacao-de-cadeias-def]{Aplicação de cadeias};\\
    \item \hyperref[homotopia-de-cadeias-def]{Homotopia de cadeia}.
% quantas dependências forem necessárias.
\end{titlemize}
\begin{defi}
	Seja $K$ um complexo simplicial, e seja $O_n (K)$ um grupo abeliano livre com base dada por símbolos 
    \[\{[v_0,...,v_n]:v_0,v_1,...,v_n \text{ geram um simplexo em }K\}.\]
    Aqui, $v_i$ são considerado ordenado, e eles podem gerar um simplexo de dimensão menor que $n$ (i.e. a lista pode repetir).

    Seja $T_n(K)\le O_n(K)$ um subgrupo gerado por seguintes elementos 
    \begin{itemize}
        \item a sequência $[v_0,...,v_n]$ tem vertices repetidos,
        \item $[v_0,v_1,...,v_n]-sign(\sigma)\cdot[v_{\sigma(0)},v_{\sigma(1)},...,v_{\sigma(n)}]$, onde $\sigma$ é uma permutação em $\{0,1,...,n\}$. 
    \end{itemize}
    Definoms $C_n(K):=O_n(K)/T_n(K)$ como grupo quociente.
\end{defi}

\begin{defi}
    O \textbf{operador bordo} é um homomorfismo de grupo dado por 
    \begin{align*}
        d_n:C_n(K)&\longrightarrow C_{n-1}(K)\\
        [v_0,v_1,...,v_n]&\longmapsto \sum_{i=0}^n (-1)^i \cdot[v_0,v_1,...,\widehat{v_i},...,v_n],
    \end{align*}
    onde $[v_0,v_1,...,\widehat{v_i},...,v_n]$ denota a sequência obtida pela remoção de $v_i$.
\end{defi}

\begin{lemma}
    O homomorfismo $d_{n-1}\circ d_n:C_n(K)\rightarrow C_{n-2}(K)$ é nulo.
\end{lemma}

\begin{dem}
    Seja $[v_0,...,v_n]\in C_n(K)$, então 
    \begin{align*}
        d_{n-1}\circ d_n ( [v_0,...,v_n])&=d_{n-1}\Bigl(\sum_{i=0}^n (-1)^i \cdot[v_0,v_1,...,\widehat{v_i},...,v_n] \Bigr) \\
        &=\sum_{i=0}^n (-1)^i  \Bigl( \sum_{k=0}^{i-1}(-1)^k[v_0,...,\widehat{v_k},...,\widehat{v_i},...,v_n])\\
        &+\sum_{k=i}^{n-1} (-1)^k[v_0,...,\widehat{v_i},...,\widehat{v_{k+1}},...,v_n]  \Bigr).
    \end{align*}
    O coeficiente de $[v_0,...,\widehat{v_a},...,\widehat{v_b},...,v_n]$ é $(-1)^a(-1)^b$ de $k=a$ e $i=b$ mais $(-1)^a(-1)^{b-1}$ de $i=a$ e $k+1=b$. Assim, cada termo se cancela, o que implica que $d_{n-1}\circ d_n([v_0,...,v_n])=0$. Como $C_n(K)$ é gerado pelos simplexos $[v_0,...,v_n]$, concluímos que $d_{n-1}\circ d_n=0$.
\end{dem}

Como a consequência, esse lema garante que $\text{Im}(d_n)\subseteq \text{Ker}(d_{n-1})$. Ou seja, a sequência 
\[...\rightarrow C_{n+1}(K)\xrightarrow{d_{n+1}}C_n(K)\xrightarrow{d_n} C_{n-1}(K)\rightarrow...\rightarrow 0\]
é um complexo de cadeias.

\begin{defi}
    O n-ésima \textbf{grupo de homologia simplicial} de um complexo simplicial $K$ é 
    \[H_n(K):=\frac{\text{Ker}(d_n)}{\text{Im}(d_{n+1})}.\]
\end{defi}



%\begin{titlemize}{Lista de consequências}
	%\item %\hyperref[consequencia1]{Consequência 1};\\ %'consequencia1' é o label onde o conceito Consequência 1 aparece
	%\item \hyperref[]{}
%\end{titlemize}



\subsection{Homologia singular} %afirmação aqui significa teorema/proposição/colorário/lema
\label{homologia-singular-def}
\begin{titlemize}{Lista de dependências}
    \item \hyperref[simplexo-def]{Simplexos}
	\item \hyperref[complexo-de-cadeias-def]{Complexo de cadeias};\\ 
    \item \hyperref[aplicacao-de-cadeias-def]{Aplicação de cadeias};\\
    \item \hyperref[homotopia-de-cadeias-def]{Homotopia de cadeia}.\\
\end{titlemize}

\begin{defi}
    Seja $X$ um espaço topológico. Um p-\textbf{simplexo singular} em $X$ é uma função contínua 
    \[\phi:\Delta^p\longrightarrow X.\]
\end{defi}

\begin{defi}
    Se $\phi$ é um $p$-simplexo singular em um espaço topológico $X$, e $i$ é um inteiro tal que $0\le i\le p$, definimos $\partial_i (\phi)$, um (p-1)-simplexo singular em $X$, por 
    \[\partial_i \phi(t_0,...,t_{p-1})=\phi(t_0,...,t_{i-1},0,t_{i+1},...,t_{p-1}).\]
    Ou seja, $\partial_i \phi=\phi|_{[v_0,...,\widehat{v_i},...,v_{p}]}$ é a $i$-ésima face de $\phi$, obtida pela substituição do parâmetro $t_i$ por zero, onde $[v_0,...,v_p]=\Delta^p$
\end{defi}

\begin{defi}
    Seja $X$ um espaço topológico, definimos $S_n(X)$ como grupo abeliano livre cujo base é o conjunto de todos $n$-simplexos singulares de $X$. Um elemento de $S_n(X)$ é dito $n$-\textbf{cadeia singular} de $X$ e tem a forma 
    \[\sum_\phi n_\phi \phi\]
    onde $n_\phi$ é um inteiro igual a zero para todos, exceto um número finito de $\phi$.
\end{defi}

Podemos estender o operador de $i$-ésima face para um homomorfismo de $S_n(X)$ em $S_{n-1} (X)$. 

\begin{defi}
    Seja $X$ um espaço topológico, definimos o operador $\partial_i$ como
    \begin{align*}
        \partial_i: S_n(X)&\longrightarrow S_{n-1}(X)\\
        \sum_\phi n_\phi \phi&\longmapsto \sum_\phi n_\phi \partial_i\phi.
    \end{align*}
    O \textbf{operador bordo} é então um homomorfismo definido por
    \begin{align*}
        \partial_{(n)}=\sum_{i=0}^n (-1)^i \partial_i:S_n(X)\longrightarrow S_{n-1}(X).
    \end{align*}
    Para simplificar a notação, omitiremos o índice do operador $\partial_{(n)}$.
\end{defi}

\begin{lemma}
    O homomorfismo $\partial\circ \partial:S_n(X)\rightarrow S_{n-2}(X)$ é nulo.
\end{lemma}

\begin{dem}
    Seja $\phi\in S_n(X)$, então 
    \begin{align*}
        \partial\circ \partial(\phi)&=\partial\Bigl(\sum_{i=0}^n (-1)^i \partial_i\phi \Bigr) \\
        &=\sum_{i=0}^n (-1)^i  \Bigl( \sum_{k=0}^{i-1}(-1)^k \partial_k\circ\partial_i \phi)+\sum_{k=i}^{n-1} (-1)^k\partial_{k}\circ\partial_i \phi  \Bigr).
    \end{align*}
    Note que $\partial_k\circ\partial_i\phi=\partial_{i-1}\circ\partial_k \phi$ se $k<i$. Logo, o coeficiente de $\partial_a\circ\partial_b \phi$ é $(-1)^a(-1)^b$ de $k=a$ e $i=b$ mais $(-1)^a(-1)^{b-1}$ de $i=a$ e $k=b-1$. Assim, cada termo se cancela, o que implica que $\partial\circ\partial\phi=0$. Como $S_n(K)$ é gerado pelos n-simplexos singulares, concluímos que $\partial\circ\partial=0$.
\end{dem}

Como a consequência, esse lema garante que $\text{Im}(\partial_{(n+1)})\subseteq \text{Ker}(\partial_{(n)})$. Ou seja, a sequência 
\[...\rightarrow S_{n+1}(X)\xrightarrow{\partial}S_n(X)\xrightarrow{\partial} S_{n-1}(X)\rightarrow...\rightarrow 0\]
é um complexo de cadeias. Denotamos esse complexo por $S(X)_*$.

Assim como no complexo de cadeias, denotamos $\text{Im}(\partial_{(n+1)})$ por $B_n(X)$ e $\text{Ker}(\partial_{(n)})$ por $Z_n(X)$.

\begin{defi}
    O n-ésima \textbf{grupo de homologia singular} de um espaço topológico $X$ é 
    \[H_n(X):=\frac{Z_n(X)}{B_n(X)}.\]
\end{defi}

\begin{titlemize}{Lista de consequências}
    \item \hyperref[homomorfismo-de-homologias-singulares-induzido-prop]{Homomorfismo de homologias singulares induzido}.\\ %'consequencia1' é o label onde o conceito Consequência 1 aparece
	%\item \hyperref[]{}
\end{titlemize}

\input{conteudo/homomorfismo-de-homologias-singulares-induzido-prop}
\input{conteudo/homologia-singular-de-um-ponto-prop}
\input{conteudo/0-esimo-grupo-de-homologia-de-espaco-zero-conexo-prop}
\input{conteudo/0-conexo-e-homomorfismo-de-homologia-induzido-prop}
\input{conteudo/homologia-singular-de-um-espaco-contratil-prop}
\input{conteudo/simplexos-singulares-subordinados-a-uma-cobertura-def}
\subsection{Sequência de Mayer-Vietoris} %afirmação aqui significa teorema/proposição/colorário/lema
\label{sequencia-de-mayer-vietoris-prop}
\begin{titlemize}{Lista de dependências}
	\item \hyperref[complexo-de-cadeias-def]{Complexo de cadeias};\\ 
    \item \hyperref[aplicacao-de-cadeias-def]{Aplicação de cadeias};\\
    \item \hyperref[homomorfismo-induzido-de-cadeias-prop]{Homomorfismo induzido de cadeias};\\
    \item \hyperref[sequencia-exata-def]{Sequência exata};\\
    \item \hyperref[homomorfismo-conectante-def]{Homomorfismo conectante};\\
    \item \hyperref[sequencia-exata-longa-induzida-prop]{Sequência exata longa induzida};\\
    \item \hyperref[homologia-singular-def]{Homologia singular};\\
    \item \hyperref[homomorfismo-de-homologias-singulares-induzido-prop]{Homomorfismo de homologias singulares induzido};\\
    \item \hyperref[simplexos-singulares-subordinados-a-uma-cobertura-def]{Simplexos singulares subordinados a uma cobertura};\\
    
\end{titlemize}

A sequência de Mayer-Vietoris é uma das mais poderosas ferramentas para o cálculo da homologia de um espaço topológico.
\begin{thm}
    Sejam $X$ um espaço topológico e $U$ e $V$ subconjuntos de $X$ tais que $int(U)\cup int(V)=X$. Considere as inclusões 
    \[i:U\cap V\hookrightarrow U,\;j:U\cap V\hookrightarrow V,\;k:U\hookrightarrow X,\;l:V\hookrightarrow X.\]
    Então, a sequência 
    \[...H_{n+1}(X)\xrightarrow{\delta} H_n(U\cap V)\xrightarrow{\Phi}H_n(U)\oplus H_n(V)\xrightarrow{\Psi} H_n(X)\rightarrow ...\]
    é exata, onde $\Phi:=i_*\oplus-j_*$ e $\Psi:=k_*+l_*$ 
\end{thm}

\begin{dem}
    Tomamos a cobertura $\mathcal{U}=\{U,V\}$ e consideramos, para cada $n\ge 0$, os homomorfismos $\phi:S_n(U\cap V)\rightarrow S_n(U)\oplus S_n(V)$ e $\psi:S_n(U)\oplus S_n(V)\rightarrow S_n^{\mathcal{U}}(X)$ dados por 
    \[\phi_n(c)=i_n(c)\oplus -j_n(c)\;\;\text{ e }\;\;\psi_n(c_1\oplus c_2)=k_n(c_1)+l_n(c_2).\]
    Para uma n-cadeia $c=a_1\sigma_1+...+a_r\sigma_r\in S_n(U\cap V)$, temos que 
    \[i_n(c)=a_1 (i\circ \sigma_1)+...+a_r(i\circ \sigma_r).\]
    Como $i\circ\sigma_1,...,i\circ \sigma_r$ são elementos da base do grupo abeliano livre $S_n(U)$, temos que $i_n(c)=0$ se, e somente se, $c=0$, o que mostra que $i_n$ é injetor. Analogamente, $j_n$ também é injetor. Portanto, $\phi_n$ é injetor.

    Pela definição, cada n-cadeia $c\in S_n^{\mathcal{U}}(X)$ é, da forma $c=c_1+c_2$ onde $c_1\in S_n (U)$ e $c_2\in S_n(V)$. Como $c=\psi_n(c_1\oplus c_2)$, provamos que $\psi_n$ é sobrejetor.

    Dada uma n-cadeia $c=a_1\sigma_1+...+a_r\sigma_r\in S_n(U\cap V)$, temos 
    \begin{align*}
        \psi_n\circ \phi_n(c)=& a_1(k\circ i\circ \sigma_1)+...+a_r(k\circ i\circ \sigma_r)\\
        &- (a_1(l\circ j\circ \sigma_1)+...+a_r (l\circ j\circ \sigma_r)).
    \end{align*}
    Como para cada n-simplexo singular $\sigma_i$, as composições $k\circ i\circ \sigma_i$ e $l\circ j \circ \sigma_i$ são iguais em $X$, obtemos $\psi_n\circ\phi_n(c)=0$. Portanto $\text{Im}(\phi_n)\subseteq \text{Ker}(\psi_n)$.

    Por outro lado, se as n-cadeias $c_1=a_1\sigma_1+...+ a_r\sigma_r \in S_n(U)$ e $c_2=a_1'\sigma_1'+...+a_s' \sigma_s'\in S_n(V)$ são não nulas tais que $c_1\oplus c_2\in \text{Ker}(\psi_n)$, então 
    \[a_1(k\circ \sigma_1)+...+a_r (k\circ \sigma_r)=-(a_1'(l\circ \sigma_1')+...+a_s'(l\circ \sigma_s')).\]
    Como só há uma única expressão de cada elemento do grupo abeliano livre $S_n(X)$ como combinação linear dos n-simplexos singulares, $r=s$ e, a menos de reordenação dos índices, $\sigma_t=\sigma_{t}'$ e $a_t'=-a_t$ para cada $1\le t\le r=s$. Dessa forma, $c_1,c_2\in S_n(U\cap V)$ e $c_2=-c_1$. Isso mostra que $c_1\oplus c_2=\phi_n (c_1)$, o que implica que $\text{Ker}(\psi_n)\subseteq \text{Im}(\phi_n)$.

    Pelos resultados obtidos acima, temos a sequência 
    \[0\rightarrow S_n (U\cap V)\xrightarrow{\phi_n} S_n (U)\oplus S_n(V)\xrightarrow{\psi_n} S_n^{\mathcal{U}}(X)\rightarrow 0\]
    é exata para cada $n\ge 0$. Logo a
    sequência dos complexos de cadeias 
    \[0\rightarrow S (U\cap V)_*\xrightarrow{\phi} S (U)_*\oplus S(V)_*\xrightarrow{\psi} S^{\mathcal{U}}(X)_*\rightarrow 0\]
    é exata, onde $\phi=(\phi_n)_{n\ge 0}$ e $\psi=(\psi_n)_{n\ge 0}$.

    Pelo Teorema \ref{sequencia-exata-longa-induzida-prop}, essa sequência induz a sequência 
    \[...\rightarrow H_{n+1}^\mathcal{U}(X)\xrightarrow{\delta} H_n(U\cap V)\xrightarrow{\phi_*}H_n(U)\oplus H_n(V)\xrightarrow{\psi_*} H^\mathcal{U}_n(X)\rightarrow ...\;.\]
    Como $\Phi=\phi_*$ e $\Psi=\psi_*$, a proposição \ref{simplexos-singulares-subordinados-a-uma-cobertura-def} garante que a sequência do enunciado do teorema 
    \[...H_{n+1}(X)\xrightarrow{\delta} H_n(U\cap V)\xrightarrow{\Phi}H_n(U)\oplus H_n(V)\xrightarrow{\Psi} H_n(X)\rightarrow ...\]
    é exata.
\end{dem}
Pela definição do homomorfismo conectante e pelo teorema \ref{sequencia-exata-longa-induzida-prop}, podemos caracterizar o homomorfismo $\delta$ na seguinte forma: uma classe de homologia $\overline{z}\in H_n (X)$ é da forma $\overline{z}=z+B_n(X)\in H_n(X)$, com o n-ciclo $z$ escrito como a soma $z=z_1+z_2$ de uma n-cadeia $z_1$ em $U$ e uma n-cadeia $z_2$ em $V$ (pelo proposição \ref{simplexos-singulares-subordinados-a-uma-cobertura-def}). Como $z$ é um n-ciclo, temos que $\partial z_1=-\partial z_2$ são (n-1)-ciclos em $U\cap V$. Então, 
\[\delta(\overline{z})=\partial z_1+B_{n-1}(U\cap V).\]

Sob certas condições, uma função contínua induz uma aplicação natural entre sequências de Mayer-Vietoris.

\begin{prop}
    Sejam $X, X'$ espaços topológicos decompostos como $X=int(U)\cup int(V)$ e $X'=int(U')\cup int(V')$. Seja $f:X\rightarrow X'$ uma função contínua tal que $f(U)\subseteq U'$ e $f(V)\subseteq V'$. Então, é comutativo o diagrama seguinte
    % https://q.uiver.app/#q=WzAsOCxbMCwwLCJIX3tuKzF9KFgpIl0sWzEsMCwiSF9uKFVcXGNhcCBWKSJdLFsyLDAsIkhfbihVKVxcb3BsdXMgSF9uKFYpIl0sWzMsMCwiSF9uKFgpIl0sWzAsMSwiSF97bisxfShYJykiXSxbMSwxLCJIX24oVSdcXGNhcCBWJykiXSxbMiwxLCJIX24oVScpXFxvcGx1cyBIX24oVicpIl0sWzMsMSwiSF9uKFgnKSJdLFswLDEsIlxcZGVsdGEiXSxbMSwyLCJcXFBoaSJdLFsyLDMsIlxcUHNpIl0sWzAsNCwiZl8qIiwyXSxbMSw1LCJmfF8qIiwyXSxbNCw1LCJcXGRlbHRhJyIsMl0sWzUsNiwiXFxQaGknIiwyXSxbNiw3LCJcXFBzaSciLDJdLFszLDcsImZfKiJdLFsyLDYsImZ8XypcXG9wbHVzIGZ8XyoiXV0=
\[\begin{tikzcd}
	{H_{n+1}(X)} & {H_n(U\cap V)} & {H_n(U)\oplus H_n(V)} & {H_n(X)} \\
	{H_{n+1}(X')} & {H_n(U'\cap V')} & {H_n(U')\oplus H_n(V')} & {H_n(X')}
	\arrow["\delta", from=1-1, to=1-2]
	\arrow["{f_*}"', from=1-1, to=2-1]
	\arrow["\Phi", from=1-2, to=1-3]
	\arrow["{f|_*}"', from=1-2, to=2-2]
	\arrow["\Psi", from=1-3, to=1-4]
	\arrow["{f|_*\oplus f|_*}", from=1-3, to=2-3]
	\arrow["{f_*}", from=1-4, to=2-4]
	\arrow["{\delta'}"', from=2-1, to=2-2]
	\arrow["{\Phi'}"', from=2-2, to=2-3]
	\arrow["{\Psi'}"', from=2-3, to=2-4]
\end{tikzcd}\]
    onde cada linha é um trecho da sequência de Mayer-Vietoris correspondente, e $f|$ denota a restrição de $f$ em domínio associado.
\end{prop}

\begin{dem}
    Como $i'\circ f|=f|\circ i$, $j'\circ f|=f|\circ j$, $f\circ k=k'\circ f|$ e $f\circ l=l'\circ f|$, temos que $i'_*\circ f|_*=f|_*\circ i_*$, $j'_*\circ f|_*=f|_*\circ j_*$, $f_*\circ k_*=k'_*\circ f|_*$ e $f_*\circ l_*=l'_*\circ f|_*$. Isso implica que os quadrados do centro e da direita são comutativos.

    Agora, vamos provar que o quadrado à esquerda também é comutativo. Seja $\overline{z}=z+B_{n+1}(X)\in H_{n+1}(X)$, pela observação acima, o $z$ pode ser escolhido como a soma $z=z_1+z_2$ de uma (n+1)-cadeia $z_1$ em $U$ e uma (n+1)-cadeia $z_2$ em V. Como $f:X\rightarrow X'$ induz uma aplicação de cadeias, temos 

    \begin{align*}
        f|_*\circ \delta(\overline{z}) & =f|_*(\partial z_1+B_{n}(U\cap V))=f(\partial z_1)+B_n(U'\cap V')\\
        &=\partial f(z_1)+B_n(U'\cap V')=\delta'(f(z_1)+f(z_2)+B_{n+1}(X'))\\
        &=\delta'(f(z)+B_{n+1}(X'))=\delta'(f_*(\overline{z})).
    \end{align*}
    Isso mostra que o quadrado à esquerda é comutativo.
\end{dem}
\begin{titlemize}{Lista de consequências}
    \item \hyperref[homologia-singular-de-S1-prop]{Homologia singular da circunferência};\\
    \item \hyperref[sequencia-exata-da-colagem-prop]{Sequência exata da colagem};\\
    \item \hyperref[grau-da-reflexao-prop]{Grau da reflexão}
	%\item \hyperref[]{}
\end{titlemize}

\subsection{Homologia singular da circunferência} %afirmação aqui significa teorema/proposição/colorário/lema
\label{homologia-singular-de-S1-prop}
\begin{titlemize}{Lista de dependências}
    \item \hyperref[sequencia-exata-def]{Sequência exata};\\
    \item \hyperref[homomorfismo-conectante-def]{Homomorfismo conectante};\\
    \item \hyperref[homologia-singular-def]{Homologia singular};\\
    \item \hyperref[homomorfismo-de-homologias-singulares-induzido-prop]{Homomorfismo de homologias singulares induzido};\\
    \item \hyperref[homologia-singular-de-um-ponto-prop]{Homologia singular de um ponto};\\
    \item \hyperref[0-esimo-grupo-de-homologia-de-espaco-zero-conexo-prop]{0-ésimo grupo de homologia singular de um espaço 0-conexo};\\
    \item \hyperref[homologia-singular-de-um-espaco-contratil-prop]{Homologia singular de um espaço contrátil};\\
    \item \hyperref[sequencia-de-mayer-vietoris-prop]{Sequência de Mayer-Vietoris}.

    
    
\end{titlemize}

\begin{prop}
    O n-ésimo grupo de homologia da circunferência é igual a 
    \begin{align*}
        H_n(\mathbb{S}^1)\cong\begin{cases}
            \mathbb{Z}&\text{se }n=0,1\\
            0&\text{se }n>1.
        \end{cases}
    \end{align*}
\end{prop}

\begin{proof}
    Denotamos os polos norte e sul de $\mathbb{S}^1\subseteq \mathbb{R}^2$ por $pn=(0,1)$ e $ps=(0,-1)$ respectivamente. Tomamos os abertos $U=\mathbb{S}^1\setminus \{ps\}$ e $V=\mathbb{S}^1\setminus \{pn\}$, cuja união $U\cup V=\mathbb{S}^1$. Pelo Teorema \ref{sequencia-de-mayer-vietoris-prop}, a sequência de Mayer-Vietoris
    \[...H_{n+1}(\mathbb{S}^1)\xrightarrow{\delta} H_n(U\cap V)\xrightarrow{\Phi}H_n(U)\oplus H_n(V)\xrightarrow{\Psi} H_n(\mathbb{S}^1)\rightarrow ...\]
    é exata.

    Os abertos $U$ e $V$ são ambos contráteis e, além disso, existe uma equivalência de homotopia sobrejetora $r:U\cap V\rightarrow\{q_1,q_2\}$, onde os pontos $q_1=(-1,0)$ e $q_2=(1,0)$. Dessa forma, os grupos de homologias $H_n(U),H_n(V)$ e $H_n(U\cup V)$ são todos triviais para $n\ge 1$, enquanto que $H_0(U)\cong \mathbb{Z}\cong H_0 (V)$ e $H_0 (U\cap V)\cong \mathbb{Z}\oplus \mathbb{Z}$.

    Como $\mathbb{S}^1$ é 0-conexo, segue que $H_0(\mathbb{S}^1)\cong \mathbb{Z}$.

    Para $n\ge 2$, pela sequência de Mayer-Vietoris, o trecho  
    \[0\rightarrow H_n (\mathbb{S}^1)\rightarrow 0\]
    é exata, consequentemente, $H_n(\mathbb{S}^1)=0.$

    O grupo $H_1(\mathbb{S}^1)$ aparece no trecho 
    \[0\rightarrow H_1(\mathbb{S}^1)\xrightarrow{\delta} H_0 (U\cap V)\xrightarrow{\Phi} H_0(U)\oplus H_0 (V),\]
    onde $\Phi=i_*\oplus -j_*$, sendo $i:U\cap V\hookrightarrow U$ e $j:U\cap V\hookrightarrow V$ as inclusões. Os pontos $q_1$ e $q_2$ podem vistos como 0-ciclos, representam geradores do grupo $H_0(U\cap V)\cong \mathbb{Z}\oplus \mathbb{Z}$. Por outro lado, esses elementos também são geradores tanto de $H_0 (U)$ quanto de $H_0(V)$, pois, em ambos os grupos, eles representam a mesma classe. Isso ocorre porque $q_1-q_2$ é o bordo de um arco hemisféricos da circunferência de $q_2$ para $q_1$ passando por cima (ou orientados no sentido anti-horário) em $U$ e de um arco hemisféricos passando por baixo (orientados no sentido horário) em $V$. Logo, existe um $q\in U\cap V$ tal que $q_1,q_2\in [q]_U=q+B_0 (U)$ e $q_1,q_2\in [q]_V=q+B_0 (V)$. Portanto, $\Phi$ é dado por 
    \begin{align*}
        \Phi(q_1+B_0(U\cap V))=[q]_U\oplus-[q]_V;\\
        \Phi(q_2+B_0(U\cap V))=[q]_U\oplus -[q]_V.
    \end{align*}
    Resulta que $\text{Ker}(\Phi)\cong \mathbb{Z}$, correspondendo ao subgrupo $\langle (1,-1) \rangle\subseteq \mathbb{Z}\oplus \mathbb{Z}\cong H_0(U\cap V)$. Por exatidão da última sequência, obtemos 
    \[H_1(\mathbb{S}^1)\cong \text{Im}(\delta)=\text{Ker}(\Phi)\cong \mathbb{Z}.\]
    Portanto, $H_n(\mathbb{S}^1)\cong \mathbb{Z}$ para $n=0$ ou $n=1$, e $H_n(\mathbb{S}^1)=0$ para todo $n\ge 2$.
\end{proof}

Para identificar um 1-ciclo $z_1\in Z_1(\mathbb{S}^1)$ cuja classe de homologia $\overline{z_1}=z_1+B_1 (\mathbb{S}^1)$ seja um gerador de $H_1(\mathbb{S}^1)$, podemos escolher $z_1$ como a soma de 1-cadeias, ou seja $z_1=c_1+c_2$, com $c_1\in S_1(U)$ e $c_2\in S_1(V)$. Como $\delta(\overline{z_1})=\partial c_1+B_0(U\cap V)$ (a observação no final de \ref{sequencia-de-mayer-vietoris-prop}), pelos isomorfismos $H_1(\mathbb{S}^1)\cong\text{Im}(\delta))\cong \text{Ker}(\Phi)$, temos que $\partial c_1=q_1-q_2=-\partial c_2$. Nessas condições, $c_1$ e $c_2$ são os 1-simplexos singulares correspondentes aos arcos hemisféricos da circunferência, orientados no sentido anti-horário.


\begin{titlemize}{Lista de consequências}
    \item \hyperref[grupo-de-homologia-singular-de-n-esfera-prop]{Grupo de homologia singular de n-esfera}.\\
	%\item \hyperref[]{}
\end{titlemize}

\input{conteudo/sequencia-exata-da-colagem-prop}
\input{conteudo/grupo-de-homologia-singular-de-n-esfera-prop}
\input{conteudo/teorema-de-invariancia-de-dimensao-de-esfera-prop}
\input{conteudo/grau-de-funcoes-em-esferas-def}
\input{conteudo/propriedades-de-grau-de-funções-prop}
\input{conteudo/grau-da-reflexao-prop}
\subsection{Grau de antípoda} %afirmação aqui significa teorema/proposição/colorário/lema
\label{grau-de-antipoda-prop}
\begin{titlemize}{Lista de dependências}
    \item \hyperref[homologia-singular-def]{Homologia singular};\\
    \item \hyperref[homomorfismo-de-homologias-singulares-induzido-prop]{Homomorfismo de homologias singulares induzido};\\
    \item \hyperref[homologia-singular-de-S1-prop]{Homologia singular da circunferência};\\
    \item \hyperref[grupo-de-homologia-singular-de-n-esfera-prop]{Grupo de homologia singular de n-esfera};\\
    \item \hyperref[grau-de-funcoes-em-esferas-def]{Grau de funções em esferas};\\
    \item \hyperref[propriedades-de-grau-de-funções-prop]{Propriedade de grau de funções em esferas};\\
    \item \hyperref[grau-da-reflexao-prop]{Grau da reflexão}
\end{titlemize}

\begin{defi}
    Uma \textbf{antípoda} é uma função de $a:\mathbb{S}^n\rightarrow \mathbb{S}^n$ dada por $a(x)=-x$.
\end{defi}

\begin{lemma}
    A antípoda da esfera $\mathbb{S}^n$ tem grau $(-1)^{n+1}$
\end{lemma}

\begin{dem}
    Para cada $1\le i\le n+1$, seja $r_i: \mathbb{S}^n\rightarrow \mathbb{S}^n$ a reflexão da i-ésima coordenada. A antípoda fatora-se como a composição $a=r_1\circ ...\circ r_{n+1}$. Segue das propriedades de grau de funções em esferas que 
    \[deg(a)=deg(r_1)\cdot...\cdot deg(r_{n+1})=(-1)^{n+1}\]
\end{dem}

\begin{titlemize}{Lista de consequências}
    \item \hyperref[homomorfismo-de-homologias-singulares-induzido-prop]{Homomorfismo de homologias singulares induzido}.\\
	%\item \hyperref[]{}
\end{titlemize}

