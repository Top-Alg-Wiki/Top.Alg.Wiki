\section{Grupo Fundamental}
\label{grupo-fundamental}

\begin{titlemize}{Lista de Dependências}
	\item \hyperref[homotopia]{Homotopia}\\ %homotopia
\end{titlemize}

Considere um espaço topológico com um ponto base fixado. O seu grupo fundamental é o grupo das classes de equivalência (sob homotopia relativa aos extremos) dos laços no espaço saindo do ponto base. Tal grupo armazena certas informações sobre buracos do espaço topologico, e é invariante sobre a equivalência homotópica. Esta é uma ferramenta poderosa para verificar se dois espaços topológicos são homeomorfos. % (homotópicos). %retirei, não entendi o que querem dizer
Veremos como sua construção se dá com mais detalhes.

%---------------------------------------------------------------------------------------------------------------------!Draft!-----------------------------------------------------------------------------------------------------------------
\subsection{Espaço de Laços}
\label{espaco-lacos-def}
\begin{titlemize}{Lista de dependências}
	%\item \hyperref[dependecia1]{Dependência 1};\\ %'dependencia1' é o label onde o conceito Dependência 1 aparece (--à arrumar um padrão para referencias e labels--)
    \item \hyperref[homotopia-relativa-def]{Homotopia Relativa}
	\item \hyperref[homotopia-relaçao-de-equivalencia-prop]{Homotopia é relação de equivalência};\\
% quantas dependências forem necessárias.
\end{titlemize}
\begin{defi}[Espaço de Laços]
	Seja $X$ um espaço topológico e seja $x_0\in X$ um ponto base. O \textbf{espaço de laços} em $X$ que saem de $x_0$ é definido como
\[\Omega(X,x_0) = \left\{\gamma: I \to X ~|~ \gamma\text{ é contínua e }\gamma(0)=\gamma(1)=x_0\right\}.\]
\end{defi}

Investigaremos a fundo o conjunto $\pi_1(X,x_0) = \Omega(X,x_0)/\sim$, onde $\alpha \sim \beta$ se, e somente se, $\alpha$ e $\beta$ são homotópicas relativo a $\partial I = \{0,1\}$.


\begin{titlemize}{Lista de consequências}
	\item \hyperref[homotopia-relaçao-de-equivalencia-prop]{Homotopia como relação de equivalência};\\ %'consequencia1' é o label onde o conceito Consequência 1 aparece
	\item \hyperref[teorema-bola-cabeluda-prop]{Teorema da bola cabeluda}
\end{titlemize}

%[Bianca]: é mais fácil criar a lista de dependências do que a de consequências.

\input{conteudo/produto-concatenacao-def}
\input{conteudo/produto-bem-definido-gr-fundamental-prop}
\subsection{Grupo fundamental}
\label{grupo-fundamental-def}
\begin{titlemize}{Lista de dependências}
	\item \hyperref[espaco-lacos-def]{O espaço de laços}
	\item \hyperref[produto-bem-definido-prop]{O produto do grupo fundamental};\\ %'dependencia1' é o label onde o conceito Dependência 1 aparece (--à arrumar um padrão para referencias e labels--) 
% quantas dependências forem necessárias.
\end{titlemize}
\begin{defi}[Grupo fundamental]
    Seja $X$ um espaço topológico e seja $x_0$ um ponto de $X.$ \textbf{O grupo fundamental de} $X$ em $x_0$ é $(\pi_1(X,x_0),\cdot)$, onde $\pi_1(X,x_0) = \Omega(X,x_0)/\sim$, onde $\alpha \sim \beta$ se, e somente se, $\alpha$ e $\beta$ são homotópicas relativo aos extremos, e o produto $\cdot$ é dado por $[\alpha]\cdot[\beta] = [\alpha \ast \beta]$, em que $\alpha \ast \beta$ é a concatenação de $\alpha$ e $\beta$.
\end{defi}

No geral, o grupo fundamental depende da escolha do ponto base $x_0$. A seguir, apresentamos um exemplo elementar de grupo fundamental.
\begin{ex}
    Seja $X=\{x\}$ é um espaço topológico contendo apenas um ponto. Nesse caso, o único laço em $X$ é a função constante $c_x:I\rightarrow \{x\}$. Assim, a única classe de homotopia é $[c_x]$, o que implica que $\pi_1(\{x\},x)=0$.
\end{ex}

\begin{titlemize}{Lista de consequências}
	\item \hyperref[hom-grupo-fundamental]{Homomorfismo de grupos fundamentais};%'consequencia1' é o label onde o conceito Consequência 1 aparece
	%\item \hyperref[]{}
\end{titlemize}

\input{conteudo/homomorfismo-de-grupo-fundamental-prop}
%---------------------------------------------------------------------------------------------------------------------!Draft!-----------------------------------------------------------------------------------------------------------------
\subsection{Conjugação por uma Curva} %[conjugacao-por-curva-prop]{Conjugação por uma Curva}
\label{conjugacao-por-curva-prop}
\begin{titlemize}{Lista de dependências}
    \item \hyperref[espaco-lacos-def]{Espaço de Laços};\\
    \item \hyperref[produto-bem-definido-prop]{O produto do grupo fundamental};\\
	\item \hyperref[grupo-fundamental-def]{O Grupo Fundamental};
% quantas dependências forem necessárias.
\end{titlemize}
%Comentário sobre os objetos envolvidos na afirmação.
\begin{defi}[Conjugação de laços por uma curva] 
	Sejam $x_0$ e $x_1$ pontos em um espaço topológico $X$ e seja $\gamma: I \to X$ uma curva contínua ligando $x_0$ a $x_1$; isto é, $\gamma(0)=x_0$ e $\gamma(1) = x_1$. Seja também $\eta \in \Omega(X,x_0)$ um laço saindo de $x_0$. Definimos a conjugação de $\eta$ por $\gamma$ como $\overline{\gamma} * \eta * \gamma \in \Omega(X,x_1)$, laço saindo de $x_1$. Isto define uma função $A_{\gamma}: \Omega(X,x_0) \to \Omega(X,x_1)$.
\end{defi}

\begin{prop}[Isomorfismo de grupos induzido por $A_{\gamma}$]
    Sejam $x_0$ e $x_1$ pontos em um espaço topológico $X$ e seja $\gamma: I \to X$ uma curva ligando $x_0$ a $x_1$.
    
    Então $A_{\gamma}$ induz um isomorfismo de grupos \begin{align*}
        \hat{A}_{\gamma}: \pi_1(X,x_0)&\to \pi_1(X,x_1)\\
        \hat{A}_{\gamma}([\eta]) &= [A_{\gamma}(\eta)] = [\overline{\gamma} * \eta * \gamma].
    \end{align*}

    \begin{dem}
        Provemos primeiramente que $\hat{A}_{\gamma}$ está bem definida. Considere $c_{\gamma}: \gamma \Rightarrow \gamma$ e $c_{\overline{\gamma}}: \overline{\gamma} \Rightarrow \overline{\gamma}$ as homotopias constantes. Assim, se $\eta, \nu \in \Omega(X,x_0)$ e $H: \eta \Rightarrow \nu$ é uma homotopia relativa a $\partial I$ então é claro que $c_{\overline{\gamma}}*H*c_{\gamma}: A_{\gamma}(\eta) \Rightarrow A_{\gamma}(\nu)$ também é homotopia relativa a $\partial I$.

        $\hat{A}_{\gamma}$ é um homomorfismo de grupos, já que dadas $\eta, \nu \in \Omega(X,x_0)$,
        \begin{align*}
            \hat{A}_{\gamma}([\eta]\cdot[\nu]^{-1})
            &= \hat{A}_{\gamma}([\eta * \overline{\nu}])\\
            &= [\overline{\gamma} * (\eta * \overline{\nu}) * \gamma]\\
            &= [(\overline{\gamma} * \eta * \gamma)*(\overline{\gamma} * \overline{\nu} * \gamma)]\\
            &= [\overline{\gamma} * \eta * \gamma]\cdot[\overline{\overline{\gamma} * \nu * \gamma}]\\
            &= \hat{A}_{\gamma}([\eta]) \cdot \hat{A}_{\gamma}([\nu])^{-1}.
        \end{align*}
        
        Por fim, note que $\hat{A}_{\gamma}$ e $\hat{A}_{\overline{\gamma}}$ são inversas, pois \[A_{\gamma} \circ A_{\overline{\gamma}}(\eta) = (\overline{\gamma} * \gamma) * \eta * (\overline{\gamma} * \gamma) \sim \eta\text{ relativa a }\partial I\]
        para toda curva $\gamma: I \to X$. Desse modo $\hat{A}_{\gamma}$ é um isomorfismo de grupos.
    \end{dem}
\end{prop}

Um fato importante decorrente de tal proposição é o seguinte.

\begin{corol}
    Se $X$ é um espaço topológico então $\pi_1(X,x_0)$ é isomorfo a $\pi_1(X,x_1)$, para quaisquer $x_0, x_1 \in X$ na mesma componente conexa por caminhos de $X$. Em especial, o grupo fundamental independe do ponto base caso $X$ seja conexo por caminhos.
\end{corol}

Dessa forma, se $X$ é um espaço conexo por caminhos, podemos denotar o grupo fundamental de $X$ por $\pi_1(X)$, omitindo o ponto base.

\begin{nota}
    Sejam $X$ e $Y$ espaços topológicos, $x_0, x_1\in X$ e $\gamma:I\to X$ uma curva ligando $x_0$ a $x_1$. Seja também $f: X\to Y$ uma função contínua e denotemos $y_0 = f(x_0)$ e $y_1 = f(x_1)$. Então $f(\gamma):I \to Y$ liga $y_0$ a $y_1$, e vale que
    \[f_{*,x_1} \circ \hat{A}_{\gamma} = \hat{A}_{f(\gamma)} \circ f_{*,x_0}.\]
    \begin{dem}
        Para cada $\eta \in \Omega(X,x_0)$,
        \begin{align*}
            \hat{A}_{f(\gamma)} \circ f_{*,x_0}([\eta])
            &= \hat{A}_{f(\gamma)} ([f(\eta]))\\
            &= [\overline{f(\gamma)} * f(\eta) * f(\gamma)]\\
            &= [f(\overline{\gamma}) * f(\eta) * f(\gamma)]\\
            &= [f(\overline{\gamma} * \eta * \gamma)]\\
            &= [f(A_{\gamma}(\eta)]
            = f_{*,x_1}\circ \hat{A}_{\gamma}([\eta]).
        \end{align*}
    \end{dem}
\end{nota}

\begin{titlemize}{Lista de consequências}
	\item \hyperref[equiv-homotopia-induz-iso]{Equivalência de homotopia e o grupo fundamental}%;\\ %'consequencia1' é o label onde o conceito Consequência 1 aparece
	%\item \hyperref[]{}
\end{titlemize}

%[Bianca]: Um arquivo tex pode ter mais de uma afirmação (ou definição, ou exemplo), mas nesse caso cada afirmação deve ter seu próprio label. Dar preferência para agrupar afirmações que dependam entre sí de maneira próxima (um teorema e seu corolário, por exemplo)

\input{conteudo/equivalencia-de-homotopia-induz-iso-thm}
\subsection{O Grupo Fundamental de um Espaço Convexo}
\label{grupo-fundamental-convexo}
\begin{titlemize}{Lista de dependências}
	\item \hyperref[grupo-fundamental-def]{Grupo Fundamental};\\ %'dependencia1' é o label onde o conceito Dependência 1 aparece (--à arrumar um padrão para referencias e labels--) 
% quantas dependências forem necessárias.
\end{titlemize}

\begin{ex}[Grupo fundamental de um espaço convexo]
O grupo fundamental de um espaço convexo X é sempre trivial, i.e, $\pi_1(X, x_0) = \{ 1\}$.
\end{ex}

De fato, isso se verifica pois, se $\alpha$ é um laço em um espaço topológico $X$ começando em um ponto $x_0$, tomando a homotopia $F:I \times I \longrightarrow X$, onde $F(s,t) = (1 - t)\alpha(s) + tx_0$, tem-se que $\alpha \sim c_{x_0}$. Assim, $\pi_1(X, x_0) = \{1\}$.

%\begin{figure}[]
%	\centering
%	\includegraphics[width=0.8\textwidth]{}
%	\caption{}
%	\label{fig:}
%\end{figure}


\subsection{Grupo fundamental de espaço de produtos}
\label{grupo-fundamental-de-espaco-de-produtos-prop}
\begin{titlemize}{Lista de dependências}
    \item \hyperref[homotopia-def]{Homotopia};\\
    \item \hyperref[grupo-fundamental]{Grupo fundamental};\\
    \item \hyperref[hom-grupo-fundamental]{Homomorfismo de grupos fundamentais}.
\end{titlemize}

\begin{prop}
    Sejam $X,Y$ espaços topológicos, com $x_0\in X$ e $y_0\in Y$, e sejam $p:X\times Y\rightarrow X$ e $q:X\times Y\rightarrow Y$ projeções canônicas. Então, o homomorfismo
    \begin{align*}
        ((p_*,q_*):\pi_1(X\times Y,(x_0,y_0))&\longrightarrow \pi_1 (X,x_0)\times \pi_1(Y,y_0)\\
        [\alpha]&\longmapsto ([p\circ \alpha],[q\circ \alpha]) 
    \end{align*}
    é um isomorfismo.
\end{prop}

\begin{dem}
    Como $p_*$ e $q_*$ são homomorfismos de grupos, $(p_*,q_*)$ também é um homomorfismo de grupos. Vamos agora verificar que $(p_*,q_*)$ é bijetivo.\\
    Injetividade: Note que $([c_{x_0}],[c_{y_0}])$ é a unidade de $\pi_1 (X,x_0)\times \pi_1(Y,y_0)$. Assim, se $[\alpha]\in \text{Ker}(p_*,q_*)$, então $[p\circ \alpha]=[c_{x_0}]$ e $[q\circ \alpha]=[c_{y_0}]$. Ou seja, existem homotopias relativa a $\partial I$, $H_1:p\circ\alpha \Rightarrow c_{x_0}$ e $H_2:q\circ \alpha \Rightarrow c_{y_0}$, o que implica que a função $H:(X\times Y)\times I\rightarrow X\times Y$ definida por
    \begin{align*}
        H((x,y),t):=(H_1(x,t),H_2(y,t))
    \end{align*}
    é uma homotopia relativa a $\partial I$ entre $(p\circ \alpha,q\circ\alpha)$ e $c_{(x_0,y_0)}$. Assim, concluímos que $[\alpha]=[c_{(x_0,y_0)}]$, o que implica que $(p_*,q_*)$ é injetivo.\\
    Sobrejetividade: Sejam $(\alpha,\beta)\in \Omega(X,x_0)\times \Omega(Y,y_0)$. Basta mostrar que existe um $\gamma\in \Omega(X\times Y,(x_0,y_0))$ tal que $p\circ\gamma=\alpha$ e $q\circ \gamma=\beta$. Porém, o laço $(\alpha,\beta)$ satisfaz exatamente essa condição. Portanto, concluímos que $(p_*,q_*)$ é sobrejetivo.
\end{dem}



%%% Local Variables:
%%% mode: LaTeX
%%% TeX-master: "../Alg.Top-Wiki"
%%% End:
