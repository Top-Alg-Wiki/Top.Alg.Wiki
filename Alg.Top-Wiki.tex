\documentclass{article}
\title{Topologia Algébrica}
%\usepackage{import}
\usepackage{amsmath} %propósitos gerais
\usepackage{amssymb} % comandos como \mathbb
\usepackage{amsthm} % criar novos teoremas

\newif\ifplastex
\plastexfalse
\ifplastex
\else
\usepackage{mdframed} %criar caixas em volta de ambientes
\fi
%-----------------------------------------------------------------------------------------------!!!!Adicionar pacotes apenas acima dessa linha!!!!--------------------------------------------------------------------------------------------
\usepackage{hyperref} %referencias internas e externas


%novos estilos
\ifplastex

\else
\mdfdefinestyle{MyFrame}{%
	innertopmargin=\baselineskip,
	innerbottommargin=\baselineskip,
	innerrightmargin=20pt,
	innerleftmargin=20pt}
\fi

%novos ambientes e teoremas
\theoremstyle{definition}
\newtheorem{defi}{Definição}

\theoremstyle{plain}
\newtheorem{thm}{Teorema}
\newtheorem{prop}{Proposição}
\newtheorem{lemma}{Lema}
\newtheorem{af}{Afirmação}
\newtheorem{corol}{Corolário}
\newtheorem{ex}{Exemplo}

\theoremstyle{remark}
\newtheorem{nota}{Nota}
\ifplastex


\newenvironment{titlemize}[1]{%
	\textbf{{#1}}
	\begin{itemize}}
	{\end{itemize}}

\else

\newenvironment{titlemize}[1]{%
	\begin{mdframed}[style=MyFrame]
	\textbf{{#1}}
	\begin{itemize}}
	{\end{itemize} \end{mdframed}}

\fi
\newenvironment{dem}{
	\begin{proof}[{\bf Demonstração:}]}
	{\end{proof}}
%novos comandos
\usepackage{quiver}
%comandos renovados



\begin{document}
\maketitle
%-------------------------------------------------------------------------------------------------------------!Draft!-------------------------------------------------------------------------------------------------------------------------
\section{Topologia quociente}
\label{topologia-quociente}
Um assunto que aparece de forma recorrente na topologia algébrica é o conceito de topologia quociente, que exploraremos a seguir. 

\subsection{Topologia Quociente}
\label{topologia-quociente-def}
\begin{titlemize}{Lista de dependências}
	\item \hyperref[topologia-final]{Topologia Final}; 
\end{titlemize}
\begin{defi}[Topologia Quociente]
	Seja \(X\) um espaço topológico e \(\sim\) uma relação de equivalência em \(X\).
	Podemos conferir ao espaço \(X/\sim\) uma estrutura de espaço topologíco da seguinte maneira. Considere a função projeção
	\begin{align*}
		\pi:X&\to X/\sim;\\
		x&\mapsto [x].
	\end{align*}
	Podemos fazer com que \(\pi\) seja uma função contínua dando, para \(X/\sim\) a topologia final com relação à \(\pi\). Isto é, os abertos de \(X/\sim\) são exatamente imagens de abertos em \(X\) por \(\pi\).  
\end{defi}
Varios exemplos importantes de espaços topológicos com os quais trabalharemos no estudo de topologia algébrica podem ser construídos como espaços quocientes. Em particular uma construção muito útil é a de quocientar um espaço por um subespaço, como explicado na seguinte definição.
\begin{defi}[Quociente por um subespaço]
	Seja \(X\) um espaço topológico e \(A \subseteq X\) um subespaço. Definimos a seguinte relação binária, \(\sim_{A}\):\\
	\(a\sim b\) se e somente se \(a=b\) ou \(a,b\in A\). Essa relação é uma relação de equivalência, e assim define um espaço \(X/\sim_A\). Esse espaço será denotado por \(X/A\). 
\end{defi}
Um exemplo desse tipo de construção é a esfera \(S^1\) que pode ser construida como \(I/\{0,1\}\) onde \(I=[0,1]\). 
\begin{titlemize}{Lista de consequências}
	\item \hyperref[pinched-torus-ex]{Torus Pinçado};
\end{titlemize}


% onde conteudos.tex é o nome do arquivo tex que voce quer incluir nessa secção.
\subsection{Função contínua em topologia quociente}
\label{funcao-continua-em-topologia-quociente-prop}
\begin{titlemize}{Lista de dependências}
	\item \hyperref[topologia-quociente-def]{Topologia quociente}; 
\end{titlemize}

\begin{prop}
    Sejam \(X,Y\) espaços topológicos, e seja \(\sim\) uma relação de equivalência em \(X\). Uma função $f:(X/\sim) \longrightarrow Y$ é contínua se e somente se $f\circ \pi$ é contínua, onde $\pi$ é a função projeção associada ao quociente.
\end{prop}
\begin{dem}
    Por um lado, suponhamos que $f$ seja contínua. Como a composição de funções contínuas é contínua, a função $f\circ \pi$ também seré contínua.

    Por outro lado, suponhamos que $f\circ \pi$ seja contínua. Seja $V\subseteq Y$ um aberto, pela hipóteses, $\pi^{-1}(f^{-1}(V))$ é um aberto. Agora, pela definição de topologia quociente, $f^{-1}(V)$ é um aberto em $X/\sim$, o que implica que $f$ é contínua. 
\end{dem}

Dois exemplos importantes de espaços quociente são os seguintes.
%-------------------------------------------------------------------------------------------------------------!Draft!-------------------------------------------------------------------------------------------------------------------------
\section{Cone e Suspensão sobre Espaços Topológicos}
\label{cone-suspensao}
\subimport{}{cone-def}
\subimport{}{suspensao-def}
Seguem alguns resultados relevantes sobre cones e suspensões:
\subimport{}{suspensao-cone-duplo-prop}
\subimport{}{cone-euclidiano-prop}
\subimport{}{suspensao-euclidiano-prop}
\subimport{}{cone-esfera-prop}
\subimport{}{suspensao-esfera-prop}
%---------------------------------------------------------------------------------------------------------------------!Draft!-----------------------------------------------------------------------------------------------------------------
\subsection{Espaços Quociente e a propriedade Hausdorff} %afirmação aqui significa teorema/proposição/colorário/lema
\label{topologia-quociente-hausdorff-thm}
\begin{titlemize}{Lista de dependências}
	\item \hyperref[topologia-quociente-def]{Espaços Quociente};\\ %'dependencia1' é o label onde o conceito Dependência 1 aparece (--à arrumar um padrão para referencias e labels--) 
% quantas dependências forem necessárias.
\end{titlemize}
Comentário sobre os objetos envolvidos na afirmação.
\begin{thm}[Espaços quocientes Hausdorff]% ou af(afirmação)/prop(proposição)/corol(corolário)/lemma(lema)/outros ambientes devem ser definidos no preambulo de Alg.Top-Wiki.tex 
Sejam $X$ um espaço Hausdorff e $\sim$ uma relação de equivalência em $X$ para a qual a projeção $\pi: X \rightarrow X/\sim$ é uma aplicação aberta. Defina o conjunto $R=\{(x,x')\in X\times X| x\sim x'\}$.

Então $X/\sim$ é Hausdorff se, e somente se, $R\subset X\times X$ é fechado.

\end{thm}
\begin{dem}
    $(\Longrightarrow)$ Se $X/\sim$ é de Hausdorff, gostaríamos de mostrar que $X\times X\backslash R$ é aberto. Para qualquer ponto $(x,x')\in (X\times X)\backslash R$, $x$ e $x'$ são tais que $\pi(x)\neq \pi(x')$. Como $X/\sim$ é de Hausdorff, existem abertos $U_x$ e $U_{x'}$ disjuntos em $X/\sim$ que são vizinhanças abertas de $\pi(x)$ e de $\pi(x')$, respectivamente.% e tais que $U_x\cap U_x' = \emptyset$.

    Temos ainda que $\pi^{-1}(U_x)$ e $\pi^{-1}(U_{x'})$ são abertos, pois a topologia de $X/\sim$ é a topologia quociente, e o produto $U=\pi^{-1}(U_x)\times \pi^{-1}(U_{x'})$ é aberto de $X\times X$ na topologia produto. Além disso, $(x,x')\in U$. Afirmamos que $U\subset X\times X\backslash R$. De fato, se $U\cap R\neq \emptyset$, teríamos $(v_1,v_2)\in U\cap R$ tal que $\pi(v_1)=\pi(v_2)$, mas $v_1 \in \pi^{-1}(U_x)$ e $v_2\in \pi^{-1}(U_{x'})$, e desse modo $\pi(v_1) = \pi(v_2) \in U_x \cap U_{x'} = \varnothing$, absurdo. Portanto, para todo $(x,x')\in X\times X\backslash R$, é possível encontrar uma vizinhança aberta $U$ de $(x,x')$ contida em $X\times X\backslash R$; $R$ é fechado, como queríamos.\newline

    $(\Longleftarrow)$ Dado que $R$ é fechado, gostaríamos de encontrar vizinhanças disjuntas de $a,~b\in X/\sim$ quaisquer para concluir que $X/\sim$ é Hausdorff. Sabemos que existem $x,~y\in X$ tais que $\pi(x)=a$ e $\pi(y)=b$ pois a projeção $\pi$ é uma aplicação sobrejetora. Como $X$ é de Hausdorff, existem abertos disjuntos $U_x$ e $U_y$, vizinhanças de $x$ e de $y$, respectivamente. Além disso, uma vez que $R$ é fechado, $X\times X\backslash R$ é aberto e, portanto, $(U_x\times U_y)\cap((X\times X)\backslash R)$ é aberto na topologia produto.

    Sejam $p_1:X\times X\rightarrow X$ e $p_2:X\times X\rightarrow X$ definidos por $$p_1(x_1,x_2)=x_1,\qquad p_2(x_1,x_2)=x_2 \qquad\forall (x_1,x_2)\in X\times X.$$ Como a topologia produto em $X\times Y$ é gerada pela base dada pelos produtos de abertos $X$ e de $Y$, é possível concluir que $p_1$ e $p_2$ são aplicações abertas. Desse modo, $U_1=p_1((U_x\times U_y)\cap((X\times X)\backslash R))$ e $U_2=p_2((U_x\times U_y)\cap((X\times X)\backslash R))$ são abertos em $X$. Por fim, basta observar que os abertos $\pi(U_1)$ e $\pi(U_2)$ são tais que $\pi(U_1)\cap \pi(U_2)=\emptyset$ uma vez que se $v\in \pi(U_1)\cap\pi(U_2)$, teríamos $v=\pi(v_1)$ para algum $v_1\in U_1$ e $v=\pi(v_2)$ para algum $v_2\in U_2$, o que implicaria $v_1\sim v_2$, um absurdo pois, pela construção de $U_1$ e $U_2$, $(v_1,v_2)\not\in R$.  Também temos $a\in U_1$ e $b\in U_2$ pois, como $a\neq b$, $x\not\sim y$. Encontramos assim os dois abertos que separam $a$ e $b$, mostrando que $X/\sim$ é Hausdorff.
\end{dem}

Comentários sobre a afirmação.

\begin{titlemize}{Lista de consequências}
	\item \hyperref[consequencia1]{Consequência 1};\\ %'consequencia1' é o label onde o conceito Consequência 1 aparece
\end{titlemize}


O espaço quociente também é essencial para realizar a colagem dos espaços.
\subsection{Pushout de espaços topológicos} %afirmação aqui significa teorema/proposição/colorário/lema
\label{pushout-de-espacos-topologicos-def}
\begin{titlemize}{Lista de dependências}
	\item \hyperref[topologia-quociente-def]{Espaços Quociente};\\ %'dependencia1' é o label onde o conceito Dependência 1 aparece (--à arrumar um padrão para referencias e labels--) 
% quantas dependências forem necessárias.
\end{titlemize}

\begin{defi}
    Sejam $X,Y,Z$ espaços topológicos. E Sejam $f:Z\rightarrow X$ e $g:Z\rightarrow Y$ funções contínuas. O \textbf{Pushout} de $f$ e $g$ é o espaço quociente $X\sqcup Y/\sim$, onde $\sim$ é a menor relação de equivalência que contém $\{(f(z),g(z))\in X\times Y:z\in Z\}$.
    % https://q.uiver.app/#q=WzAsNCxbMCwwLCJaIl0sWzAsMSwiWCJdLFsxLDAsIlkiXSxbMSwxLCJYXFxzcWN1cCBZL1xcc2ltIl0sWzAsMSwiZiIsMl0sWzAsMiwiZyJdLFsxLDNdLFsyLDNdXQ==
\[\begin{tikzcd}
	Z & Y \\
	X & {X\sqcup Y/\sim}
	\arrow["g", from=1-1, to=1-2]
	\arrow["f"', from=1-1, to=2-1]
	\arrow[from=1-2, to=2-2]
	\arrow[from=2-1, to=2-2]
\end{tikzcd}\]
\end{defi}

\subsection{Colagem de n-célula} %afirmação aqui significa teorema/proposição/colorário/lema
\label{colagem-de-n-celula-def}
\begin{titlemize}{Lista de dependências}
	\item \hyperref[topologia-quociente-def]{Espaços Quociente};\\
    \item \hyperref[pushout-de-espacos-topologicos-def]{\emph{Pushout} de espaços topológicos}.%'dependencia1' é o label onde o conceito Dependência 1 aparece (--à arrumar um padrão para referencias e labels--) 
% quantas dependências forem necessárias.
\end{titlemize}

\begin{defi}
    Seja $X$ um espaço topológico, e sejam $f:\mathbb{S}^{n-1}\rightarrow X$ uma função contínua e $i:\mathbb{S}^{n-1}\hookrightarrow D^n$ uma inclusão, onde $n\ge 2$. O \textbf{espaço obtido de} $X$ \textbf{pela colagem de uma $n$-célula por meio da função} $f$ é o \emph{pushout} de $f$ e $i$, denotado por $X_f$ ou $D^n\cup_f X$.
\end{defi}

%\begin{titlemize}{Lista de consequências}
	%\item %\hyperref[consequencia1]{Consequência 1};\\ %'consequencia1' é o label onde o conceito Consequência 1 aparece
%\end{titlemize}

\subsection{Colagem de um disco com um ponto} %afirmação aqui significa teorema/proposição/colorário/lema
\label{colagem-de-um-disco-com-um-ponto-ex}
\begin{titlemize}{Lista de dependências}
	\item \hyperref[topologia-quociente-def]{Espaços Quociente};\\
    \item \hyperref[pushout-de-espacos-topologicos-def]{Pushout de espaços topológicos};\\
    \item \hyperref[colagem-de-n-celula-def]{Colagem de n-célula}%'dependencia1' é o label onde o conceito Dependência 1 aparece (--à arrumar um padrão para referencias e labels--) 
% quantas dependências forem necessárias.
\end{titlemize}

\begin{ex}
    Para $n\ge 2$, a colagem $\{x\}_f=D^n\cup_f \{x\}$, em que $f:\mathbb{S}^{n-1}\rightarrow \{x\}$ é a função constante, é a esfera $\mathbb{S}^n$. Visto que $int(D^n)$ é homeomorfo a $\mathbb{R}^n$, e $\mathbb{R}^n$ é homeomorfo a $\mathbb{S}^n\setminus\{\text{polo norte}\}$, e que a colagem garante que $\{x\}_f\setminus\{\text{origem do disco }D^n \}$ é homeomorfo a $\mathbb{S}^n\setminus\{\text{polo sul}\}$, conclui-se que $\{x\}_f$ é homeomorfo à esfera $\mathbb{S}^n$.
\end{ex}

%\begin{titlemize}{Lista de consequências}
	%\item %\hyperref[consequencia1]{Consequência 1};\\ %'consequencia1' é o label onde o conceito Consequência 1 aparece
%\end{titlemize}

%%% Local Variables:
%%% mode: LaTeX
%%% TeX-master: "../Alg.Top-Wiki"
%%% End:


\section{Homotopia}
\label{homotopia}
Um assunto que aparece na definição de objetos importantes na topologia algébrica, como os grupos de homotopia e, em particular, o grupo fundamental.

%---------------------------------------------------------------------------------------------------------------------!Draft!-----------------------------------------------------------------------------------------------------------------
\subsection{Homotopia}
\label{homotopia-def}
%\begin{titlemize}{Lista de dependências}
	%\item \hyperref[dependecia1]{Dependência 1};\\ %'dependencia1' é o label onde o conceito Dependência 1 aparece (--à arrumar um padrão para referencias e labels--) 
	%\item \hyperref[]{};\\
% quantas dependências forem necessárias.
%\end{titlemize}
\begin{defi}[Homotopia]
	Sejam $X$ e $Y$ espaços topológicos. Uma homotopia entre funções contínuas $f_0, f_1: X\rightarrow Y$ é uma função $$H:X\times I\rightarrow Y$$ também contínua tal que $H(x,0)=f_0(x)$ e $H(x,1)=f_1(x)$.
\end{defi}

Se existe homotopia entre os mapas $f_0$ e $f_1$, costumamos dizer que $f_0$ é homotópico a $f_1$, isto é, $f_0\sim f_1$. Ainda é possível dizer que $f_0$ é homotópico a $f_1$ através da homotopia $H$, isto é, $f_0 \underset{H}{\sim} f_1$.

\begin{titlemize}{Lista de consequências}
	\item \hyperref[homotopia-relaçao-de-equivalencia]{Homotopia como relação de equivalência};\\ %'consequencia1' é o label onde o conceito Consequência 1 aparece
	\item \hyperref[homotopia-teorema-da-bola-cabeluda]{Teorema da bola cabeluda}
\end{titlemize}

%[Bianca]: é mais fácil criar a lista de dependências do que a de consequências.
% onde conteudos.tex é o nome do arquivo tex que voce quer incluir nessa secção.

%---------------------------------------------------------------------------------------------------------------------!Draft!-----------------------------------------------------------------------------------------------------------------
\subsection{Homotopia Relativa}
\label{homotopia-relativa-def}
\begin{titlemize}{Lista de dependências}
	%\item \hyperref[dependecia1]{Dependência 1};\\ %'dependencia1' é o label onde o conceito Dependência 1 aparece (--à arrumar um padrão para referencias e labels--) 
	\item \hyperref[homotopia-def]{Homotopia};
\end{titlemize}
\begin{defi}[Homotopia Relativa a um Subconjunto]
	Sejam $X$ e $Y$ espaços topológicos, e seja $H:f_0\Rightarrow f_1$ uma homotopia, onde $f_0, f_1: X\rightarrow Y$ são funções contínuas. Dizemos que $H$ é uma \textbf{homotopia relativa} a um subconjunto $A\subset X$ se a homotopia mantém fixos os pontos de $A$; ou seja, se $H(x,t) = f(x) = g(x)$ para todos $(x,t) \in A\times I$. Em especial, $f$ e $g$ devem coincidir em $A$.
\end{defi}

\begin{titlemize}{Lista de consequências}
	\item \hyperref[homotopia-relaçao-de-equivalencia-prop]{Homotopia como relação de equivalência};
    \item \hyperref[espaco-lacos-def]{Espaço de Laços}
    \item \hyperref[produto-bem-definido-prop]{O produto do grupo fundamental}
\end{titlemize}

%[Bianca]: é mais fácil criar a lista de dependências do que a de consequências.

%---------------------------------------------------------------------------------------------------------------------!Draft!-----------------------------------------------------------------------------------------------------------------
\subsection{Homotopia como relação de equivalência} %afirmação aqui significa teorema/proposição/colorário/lema
\label{homotopia-relaçao-de-equivalencia-prop}
\begin{titlemize}{Lista de dependências}
	\item \hyperref[homotopia-def]{Homotopia};\\ %'dependencia1' é o label onde o conceito Dependência 1 aparece (--à arrumar um padrão para referencias e labels--) 
	%\item \hyperref[]{};\\
% quantas dependências forem necessárias.
\end{titlemize}


Provemos que a relação de homotopia definida anteriormente é de equivalência.


\begin{af}[Homotopia é relação de equivalência]% ou af(afirmação)/prop(proposição)/corol(corolário)/lemma(lema)/outros ambientes devem ser definidos no preambulo de Alg.Top-Wiki.tex 
    Sejam $f,~f_0,~f_1,~f_2:X\rightarrow Y$ funções contínuas quaisquer. Então temos:
	\begin{itemize}
	    \item Sempre vale que $f\sim f$.
        \item Se $f_0\sim f_1$, então $f_1\sim f_0$.
        \item Se $f_0\sim f_1$ e $f_1\sim f_2$, então $f_0\sim f_2$.
	\end{itemize}
	Desse modo, a relação de homotopia é de equivalência.
\end{af}
\begin{dem}
    No primeiro item, é possível provar que $f\sim f$ através da homotopia $H:X\times I \rightarrow Y$ definida por $$H(x,t)=f(x)\text{ para todo }t\in I\text{ e todo }x\in X.$$ A continuidade de $H$ segue diretamente do fato de $f$ ser contínua e, como $H(x,0)=f(x)$ e $H(x,1)=f(x)$, de fato $f \underset{H}{\sim} f$.\\

    No segundo item, se $f_0\sim f_1$, então existe uma homotopia $H:X\times I\rightarrow Y$ entre as duas funções. Assim, a função $\overline{H}:X\times I\rightarrow Y$ definida como $$\overline{H}(x,t)=H(x,1-t)\text{ para todo }t\in T\text{ e todo }x\in X$$ é contínua por ser composta das funções contínuas $H$ e $(x,t)\mapsto (x,1-t)$. Além disso, $\overline{H}(x,0)=H(x,1)=f_1$ e $\overline{H}(x,1)=H(x,)=f_0$. Portanto, $f_1 \underset{\overline{H}}{\sim} f_0$.\\%troquei para \overline{H} para usar a notação do professor

    Por último, se $f_0\underset{H_1}{\sim} f_1$ e $f_1\underset{H_2}{\sim}f_2$, considere os mapas contínuos $H_1(x,2t)$ e $H_2(x,2t-1)$, que são compostas de mapas contínuos definidos para $(x,t)\in[0,\frac{1}{2}]$ e $(x,t)\in X\times[\frac{1}{2},1]$, respectivamente. Na intersecção $X\times[0,\frac{1}{2}]\cap X\times[\frac{1}{2},1]=X\times \{\frac{1}{2}\}$  temos que $H_1(x,2t)=f_1(x)=H_2(x,2t-1)$ para todo $(x,t)\in X\times \{\frac{1}{2}\}$. Assim, pelo Lema da Colagem, é possível definir a função contínua $H_1*H_2:X\times I\rightarrow Y$ por $$H_1*H_2(x,t)=
    \begin{cases}
        H_1(x,2t), &\text{ se }t\in[0,\frac{1}{2}]\\
        H_2(x,2t-1), &\text{ se }t\in[\frac{1}{2},1]\\
    \end{cases}\text{ para todo }x\in X.$$
    De fato, $H_1*H_2$ é homotopia entre $f_0$ e $f_2$ porque $H_1*H_2(x,0)=H_1(x,0)=f_0(x)$ e $H_1*H_2(x,1)=H_2(x,1)=f_2(x)$ para todo $x\in X$.
    
\end{dem}

Tal resultado é imediatamente generalizável para homotopias relativas.

\begin{af}[Homotopia relativa é relação de equivalência]% ou af(afirmação)/prop(proposição)/corol(corolário)/lemma(lema)/outros ambientes devem ser definidos no preambulo de Alg.Top-Wiki.tex 
    Sejam $f,~f_0,~f_1,~f_2:X\rightarrow Y$ funções contínuas e seja $A\subset X$. Então temos:
	\begin{itemize}
	    \item Sempre vale que $f\sim f$ relativo a $A$ (de fato, relativo a $X$).
        \item Se $f_0\sim f_1$ relativo a $A$, então $f_1\sim f_0$ relativo a $A$.
        \item Se $f_0\sim f_1$ relativo a $A$ e $f_1\sim f_2$ relativo a $A$, então $f_0\sim f_2$ relativo a $A$.
	\end{itemize}
	Desse modo, a relação de homotopia relativa a $A$ é de equivalência.

    \begin{dem}
         Considere as mesmas homotopias que na afirmação anterior. É claro que $H:f\Rightarrow f$ é relativa a $X$. Além disso, $\overline{H}$ é relativa a $A$ se, e somente se, $H$ é relativa a $A$, e $H_1*H_2$ é relativa a $A$ se, e somente se $H_1$ e $H_2$ são relativas a $A$. Assim concluímos.
    \end{dem}
\end{af}

Denotamos por $[X,Y]$ o conjunto das classes de homotopia de funções de $X$ para $Y$. Isto é, $[X,Y]=C(X,Y)/\sim\hspace{5 pt}=\{[f]|f:X\rightarrow Y\text{ é contínua}\}$.

\begin{titlemize}{Lista de consequências}
	\item \hyperref[espaco-lacos-def]{Espaço de Laços};
	%\item \hyperref[]{}
\end{titlemize}

%[Bianca]: Um arquivo tex pode ter mais de uma afirmação (ou definição, ou exemplo), mas nesse caso cada afirmação deve ter seu próprio label. Dar preferência para agrupar afirmações que dependam entre sí de maneira próxima (um teorema e seu corolário, por exemplo)

%---------------------------------------------------------------------------------------------------------------------!Draft!-----------------------------------------------------------------------------------------------------------------
\subsection{Equivalência de Homotopia}
\label{equiv-homotopia}
\begin{titlemize}{Lista de dependências}
	\item \hyperref[homotopia-def]{Homotopia};\\
\end{titlemize}

\begin{defi}[Equivalência de Homotopia]
	Sejam $X$ e $Y$ espaços topológicos. Uma função contínua $f:X\to Y$ é dita uma \textbf{equivalência de homotopia} se existe outra função contínua $g:Y\to X$ tal que $f\circ g \sim \text{id}_Y$ e $g\circ f \sim \text{id}_X$. Nesse caso, dizemos que $X$ e $Y$ são \textbf{equivalentes homotópicos}, e $g$ é \textbf{inversa a menos de homotopia} de $f$.
\end{defi}

É claro que todo homeomorfismo é uma equivalência de homotopia.

\begin{titlemize}{Lista de consequências}
	\item \hyperref[equiv-homotopia-induz-iso]{Equivalência de homotopia e grupo fundamental}
\end{titlemize}

%[Bianca]: é mais fácil criar a lista de dependências do que a de consequências.

\subsection{Espaço contrátil}
\label{espaco-contratil-def}
\begin{titlemize}{Lista de dependências}
	\item \hyperref[homotopia-def]{Homotopia};\\
        \item \hyperref[equiv-homotopia]{Equivalência de Homotopia}.
\end{titlemize}

\begin{defi}
	Seja $X$ um espaço topológico. Diremos que o espaço $X$ é \textbf{contrátil} se $X$ é homotopicamente equivalente a um ponto.
\end{defi}

Ou seja, um espaço topológico $X$ é contrátil se existem $f:\{*\} \to X$ e $g: X\to \{*\}$ tais que $g\circ f \sim \text{id}_X$ e $f\circ g \sim \text{id}_{\{*\}}$. Mas como $f\circ g = \text{id}_{\{*\}}$, e substituindo $f:\{*\} \to X$ pela inclusão $\{f(*)\} \hookrightarrow X$, concluímos que $X$ é contrátil se, e somente se, existem $x_0\in X$ e $H: X\times I\to X$ tais que $H(x,0)=x$ e $H(x,1) = x_0$ para todo $x \in X$.

\begin{ex}
    \begin{itemize}
        \item $D^n$ é contrátil, para todo $n\geq 0$, %. Sejam $f: D^n\to\{0\}$ e $g:\{0\}\hookrightarrow D^n$. Então $f\circ g$ é a identidade em $\{0\}$ e $g\circ f$ é homotópico à identidade em $D^n$,
        via
        \begin{align*}
            H: D^n \times I &\to D^n\\
            (x,t) &\mapsto (1-t)x.
        \end{align*}
        \item Mais geralmente, se $S \subset \mathbb{R}^n$ é um subconjunto estrelado, então é contrátil. Seja $s_0\in S$ tal que $ts + (1-t)s_0 \in S$ para todos $t\in I$ e $s\in S$. Como $S-s_0 = \{s-s_0~|~s\in S\} \cong S$, podemos supor que $s_0 = 0$. %Além disso, note que, como $\mathbb{R}^n$ e $\text{int}(D^n)$ são homeomorfos, podemos supor que $S \subset D^n$. %<-- acho que não precisa
        Assim, definimos
        \begin{align*}
            H: S \times I &\to S\\
            (x,t) &\mapsto (1-t)x.
        \end{align*}
        %e concluímos como no caso anterior.
        \item Dado um espaço topológico $X$, o cone $C(X)$ é contrátil. Para ver isso, seja $v$ o vértice do cone e defina%, $f: C(X)\to\{v\}$ e $g: \{v\}\to C(X)$. Então $f\circ g$ é a identidade em $\{v\}$ e $g\circ f$ é homotópico à identidade em $C(X)$, pois
        \begin{align*}
            \text{id}_X\times m: X\times I \times I &\to X\times I\\
            (x,s,t) &\mapsto (x,st).
        \end{align*}
        Então $\text{id}_X\times m$ induz $H_0: C(X)\times I\to X\times I$ dada por $([x,s],t) \mapsto (x,st)$. Assim, definimos $H = \pi \circ H_0$, onde $\pi: X\times I \to C(X)$ é a aplicação quociente.
    \end{itemize}
\end{ex}
\subsection{lema-teo-fundamental-da-algebra} 
\label{lema-teo-fundamental-da-algebra}
\begin{titlemize}{Lista de dependências}
	\item \hyperref[homotopia]{homotopia};\\  
	\item \hyperref[grupo-fundamental]{grupo-fundamental};\\
    \item \hyperref[retração]{retração};

\end{titlemize}
O próximo lema será uma das bases para a demonstração do Teorema Fundamental da Álgebra.
\begin{thm}[Lema] 
	Seja $\mathbb{C}$ o conjunto dos números complexos e $r\mathbb{S}^1 \subseteq \mathbb{R}^2 \cong \mathbb{C}$ a esfera centrada em 0 e de raio $r$. Sejam $f_r^n: r\mathbb{S}^1 \rightarrow \mathbb{C}\backslash \{0\}$ as restrições das funções que levam $z$ em $z^n$. Se nenhuma das funções é homotópica a uma função constante, então vale o Teorema Fundamental da Álgebra, ou seja, todo polinômio com grau maior ou igual a 1 tem raiz complexa.
	
\end{thm}

\begin{dem}
    Seja $g(z) = a_nz^n + a_{n - 1}z^{n - 1} +...+ a_0$
um polinômio com raízes complexas e com $a_n \neq 0$. Sem perda de generalidade, podemos supor $a_n = 1$ (basta tomar $h(z) = \frac{g(z)}{a_n}$).\\ Seja $r > max\{1, \sum_{i = 1}^n|a_i|\}$ e defino $F:r\mathbb{S}^1\times I \rightarrow \mathbb{C}$ como

$$F(z, t) = z^n + \sum_{i = 1}^{n-1}(1 - t)a_iz^i$$

Nota-se facilmente que $F$ é homotopia de $f_r^n$ e $g|_{r\mathbb{S}^1}$. Ainda, o contradomínio da $F$ é $\mathbb{C}\backslash \{0\}$, pois se não, haveria $z_0$ com $|z_0| = r$ e $t_0 \in I$, tal que $F(z_0, t_0) = 0$, i.e. $z_0^n = -\sum_{i = 1}^{n - 1}(1 - t_0)a_iz^i$ e, então, 

$$|z_o^n| = r^n \leq \sum_{i = 1}^{n - 1}|(1 - t_0)a_iz_0^i| \leq \sum_{i = 1}^{n - 1}|a_iz_0^i| \leq r^{n-1}\sum_{i = 1}^{n - 1}|a_i|$$

que implicaria $r \leq \sum_{i = 1}^{n - 1}|a_i|$, contradizendo a escolha de $r$.

Agora, por contrapositiva, assumimos que $g$ não tem raízes complexas, i.e. $g:\mathbb{C} \rightarrow \mathbb{C}\backslash\{0\}$. Seja $G: r\mathbb{S}^1\times I \rightarrow \mathbb{C}\backslash\{0\}$ definida por $g((1 - t)z)$. Temos que $g \sim_G c$, onde $c(z) = g(0) = a_0$ para todo $z \in r\mathbb{S}^1$. Assim, temos que $f_r^n \sim_G c$ por transitividade, garantindo a contrapositiva.  

\end{dem}
A restrição no contradomínio da $f_r^n$ é importante, pois se o contradomínio for contrátil, então qualquer função é homotópica a uma constante. De fato, se $f:X \rightarrow Y$ e $Y$ é contrátil, então $1_Y \sim_H k$, onde $k$ é uma função constante e $H$ é homotopia. Então, $f\circ1_Y \sim_H f\circ k$ e, portanto, $f \sim_H f(k_0)$, onde $k_0 = k(z)$. \\

Note também que a homotopia de $g|_{r\mathbb{S}^1}$ e $f_r^n$ deforma continuamente a imagem do polinômio que, restrito a $r\mathbb{S}^1$, torna-se uma curva fechada no plano dos complexos, no círculo de raio $r$ em $\mathbb{C}. Veja a imagem abaixo.

\begin{figure}[h]
\includegraphics[width = 5cm]{homotopia.png}
\end{figure}

\begin{titlemize}{Lista de consequências}
	\item \hyperref[teo-fundamental-da-algebra]{teo-fundamental-da-algebra};\\ 
	
\end{titlemize}

\subsection{teo-fundamental-da-algebra} %afirmação aqui significa teorema/proposição/colorário/lema
\label{teo-fundamental-da-algebra}
\begin{titlemize}{Lista de dependências}
	\item \hyperref[lema-teo-fundamental-da-algebra]{lema-teo-fundamental-da-algebra};\\ %'dependencia1' é o label onde o conceito Dependência 1 aparece (--à arrumar um padrão para referencias e labels--) 
	\item \hyperref[extensão-de-função-na-esfera]{extensão-de-função-na-esfera};\\
  \item \hyperref[grupo-fundamental-de-S1-prop]{grupo-fundamental-de-S1-prop}
\end{titlemize}
\begin{thm}[Teorema Fundamental da Álgebra]

	Todo polinômio não constante com coeficientes complexos possui raíz em $\mathbb{C}$. \\
 
 Em outras palavras, esse teorema garante que o corpo dos complexos é um fecho algébrico dos corpos de característica 0.
 
\end{thm}

\begin{dem}

Pelo lema referenciado na lista de dependências, basta mostrar que as funções $f_r^n: r\mathbb{S}^1 \rightarrow \mathbb{C}\backslash\{0\}$, definidas por $f_r^n(z) = z^n$, não são homotópicas a uma constante.
De fato, caso $f_r^n$ fosse homotópica a uma constante, então $h: \mathbb{S}^1 \rightarrow r\mathbb{S}^1 \rightarrow \mathbb{C}\backslash\{0\} \rightarrow \mathbb{S}^1$ dada por $h(z) = \frac{f_r^n(rz)}{|f_r^n(rz)|}$ também seria homotópica a uma constante, já que se $H:r\mathbb{S}^1 \times I \rightarrow \mathbb{C}\backslash\{0\}$ é a homotopia entre $f_r^n$ e uma constante $c_0$, então $\frac{1}{r^n}H(rz, t)$ é homotopia entre $h$ e $\frac{c_0}{r^n}$.\\
No entanto, por uma das equivalências de \hyperref[extensão-da-função-na-esfera]{extensão-da-função-na-esfera}, se $h$ é homotópica a uma constante, então $h_*$ é trivial, implicando que $h_*([e^{2\pi i}])$ é elemento neutro em $\mathcal{S}^1$, ou seja $[e^{2\pi in}] = [1]$ e, portanto, com a linguagem do levantamento de homotopia na seção do grupo fundamental de $\mathbb{S}^1$, temos que $deg([e^{2\pi in]}) = deg([1]) = 0$, contrariando o fato de que, na verdade, $deg([e^{2\pi in}]) = n$. Portanto, $f_r^n$ não pode ser homotópico a uma constante e, dessa forma, vale o Teorema Fundamental da Álgebra.

    
\end{dem}

Note que a interpretação geométrica de $h$ é de alongar os pontos em $\mathbb{S}^1$, aumentando o seu raio, e depois rotacioná-los e contraí-los de volta a $\mathbb{S}^1$. A homotopia de $h$ à constante deforma continuamente a circunferência de raio $1$ no ponto $c_0$, trazendo-o mais próximo da origem.

%---------------------------------------------------------------------------------------------------------------------!Draft!-----------------------------------------------------------------------------------------------------------------
\subsection{Identidade e antípoda são homotópicas se há campo não nulo} %afirmação aqui significa teorema/proposição/colorário/lema
\label{identidade-e-antipoda-homotopicas-prop}
\begin{titlemize}{Lista de dependências}
	\item \hyperref[homotopia-def]{Homotopia};\\ %'dependencia1' é o label onde o conceito Dependência 1 aparece (--à arrumar um padrão para referencias e labels--) 
	%\item \hyperref[]{};\\
% quantas dependências forem necessárias.
\end{titlemize}




\begin{lemma}[Funções identidade e antípoda na esfera]% ou af(afirmação)/prop(proposição)/corol(corolário)/lemma(lema)/outros ambientes devem ser definidos no preambulo de Alg.Top-Wiki.tex 
	Se existe uma função contínua $v:S^n\rightarrow \mathbb{R}^{n+1}$ que leva todo $x\in S^n$ em $v(x)$ com $\langle x, v(x) \rangle =0$, isto é, um campo vetorial contínuo não nulo tangente à esfera, então as funções identidade $Id_{S^n}:S^n\rightarrow S^n$ e antípoda $A_{S^n}:S^n\rightarrow S^n$ dada por $A_{S^n}(x)=-x$ para todo $x\in S^n$ são homotópicas.
\end{lemma}
\begin{dem}
    Seja $v:S^n\rightarrow \mathbb{R}^{n+1}$ o campo vetorial contínuo tangente à esfera que é não nulo em todos os pontos e considere a função contínua $H:S^n\times I\rightarrow S^n$ definida por $$H(x,t)=cos(\pi t)x+sen(\pi t)\frac{v(x)}{||v(x)||}.$$
    De fato, $H$ está bem definida pois, utilizando o fato de que $\langle x, v(x) \rangle =0$, temos 

    $$|cos(\pi t)x+sen(\pi t)\frac{v(x)}{||v(x)||}|^2=$$$$=\langle cos(\pi t)x+sen(\pi t)\frac{v(x)}{||v(x)||} , cos(\pi t)x+sen(\pi t)\frac{v(x)}{||v(x)||}\rangle=$$$$=cos^2(\pi t)\langle x,x \rangle +sen^2(\pi t)\langle \frac{v(x)}{||v(x)||},\frac{v(x)}{||v(x)||}\rangle=$$$$=cos^2(\pi t)+sen^2(\pi t)=1.$$
    
    Isto é, $H(x,t)$ pertence à esfera para todo $(x,t)\in X\times I$. Além disso, $H$ é homotopia entre a identidade e a antípoda pois $$H(x,0)=cos(0)x+sen(0)\frac{v(x)}{||v(x)||}=x=Id(x)\text{ para todo } x\in S^n$$ $$\text{e } H(x,1)=cos(\pi)x+sen(\pi)\frac{v(x)}{||v(x)||}=-x=A(x)\text{ para todo }x\in S^n.$$
\end{dem}


\begin{titlemize}{Lista de consequências}
	\item \hyperref[teorema-bola-cabeluda-prop]{Teorema da Bola Cabeluda};\\ %'consequencia1' é o label onde o conceito Consequência 1 aparece
	%\item \hyperref[]{}
\end{titlemize}

%[Bianca]: Um arquivo tex pode ter mais de uma afirmação (ou definição, ou exemplo), mas nesse caso cada afirmação deve ter seu próprio label. Dar preferência para agrupar afirmações que dependam entre sí de maneira próxima (um teorema e seu corolário, por exemplo)

%---------------------------------------------------------------------------------------------------------------------!Draft!-----------------------------------------------------------------------------------------------------------------
\subsection{Teorema da Bola Cabeluda} %afirmação aqui significa teorema/proposição/colorário/lema
\label{teorema-bola-cabeluda-prop}
\begin{titlemize}{Lista de dependências}
	\item \hyperref[homotopia-def]{Homotopia};\\ %'dependencia1' é o label onde o conceito Dependência 1 aparece (--à arrumar um padrão para referencias e labels--) 
	\item \hyperref[identidade-e-antipoda-homotopicas-prop]{Homotopia entre identidade e antípoda};\\
% quantas dependências forem necessárias.
\end{titlemize}


\begin{thm}[Teorema da Bola Cabeluda]% ou af(afirmação)/prop(proposição)/corol(corolário)/lemma(lema)/outros ambientes devem ser definidos no preambulo de Alg.Top-Wiki.tex 
	A esfera $S^n$ possui um campo vetorial contínuo não nulo em todo ponto isto é, existe uma função contínua $v:S^n\rightarrow \mathbb{R}^{n+1}$ que leva todo $x\in S^n$ a $v(x)$ com $\langle x, v(x) \rangle =0$ e $v(x)\ne 0$, se e somente se $n$ é ímpar.\\
\end{thm}
\begin{dem}
    Se $n$ é ímpar, $n=2k-1$ para algum natural $k$ e basta tomar o campo $$v(x_1,~...,~x_{2k})=(-x_2,~x_1,~-x_4,~x_3,~...,~-x_{2k},~x_{2k-1})\text{ para todo }(x_1,~...,~x_{2k})\in S^n.$$
    De fato, a função é contínua, temos $$\langle(-x_2,~x_1,~...,~-x_{2k},~x_{2k-1}), (x_1,~x_2,~...,~x_{2k-1},~x_{2k}) \rangle =$$$$=-x_1x_2+x_1x_2-...-x_{2k-1}x_{2k}+x_{2k-1}x_{2k}=0$$ e vale $v(x)\ne 0$ para todo $x=(x_1,...,x_{2k})\in S^n$ pois $v(x)=(x_1,~x_2,~...,~x_{2k-1},~x_{2k})=0$ somente se $x_i=0$ para todo $i\in {1,~...,~2k}$, isto é, $v(x)=0$ apenas em ponto fora da esfera.\\

    Reciprocamente, se a esfera $S^n$ possui campo vetorial contínuo não nulo em cada ponto conforme condição do enunciado, segundo o lema \ref{identidade-e-antipoda-homotopicas-prop}, existe uma homotopia entre os mapas identidade e antípoda na esfera, o que não pode ocorrer de $n$ for par.

    
\end{dem}

Vale ressaltar que é possível demostrar que de fato não há homotopia entre a identidade e a antípoda em esferas de grau par através da noção de grau de uma função, existente em homologia. Esse tópico não será abordado neste material.

\begin{titlemize}{Lista de consequências}
	\item \hyperref[consequencia1]{Consequência 1};\\ %'consequencia1' é o label onde o conceito Consequência 1 aparece
	%\item \hyperref[]{}
\end{titlemize}

%[Bianca]: Um arquivo tex pode ter mais de uma afirmação (ou definição, ou exemplo), mas nesse caso cada afirmação deve ter seu próprio label. Dar preferência para agrupar afirmações que dependam entre sí de maneira próxima (um teorema e seu corolário, por exemplo)
%%% Local Variables:
%%% mode: LaTeX
%%% TeX-master: "../Alg.Top-Wiki"
%%% End:

\section{Grupo Fundamental}
\label{grupo-fundamental}

\begin{titlemize}{Lista de Dependências}
	\item \hyperref[Homotopia]{homotopia};\\ %homotopia
	\item \hyperref[]{};
\end{titlemize}

O grupo fundamental de um espaço topológico é o grupo das classes de equivalência sob homotopia dos laços contidos no espaço. Ele armazena certas informações sobre buracos do espaço topologico, e é invariante sobre a equivalência homotópica. Isso será um conceito importante quando queremos verificar se dois espaços topológicos são homeomorfos (homotópico).

\subimport{}{produto-concatenacao}% a construir
\subimport{}{produto-bem-definido-gr-fundamental-prop}
\subimport{}{grupo-fundamental-def}
\subimport{}{homomorfismo-de-grupo-fundamental-prop}

\section{Categorias}
\label{categorias}

A teoria das categorias pode ser vista como uma ferramenta usada para os estudos das conexões das diversas áreas da matemática. Nessa Wiki, usaremos linguagens de teoria das categorias para poder desenvolver a topologia algébrica, pois essa área possui conexões entre a topologia e a álgebra.

\subsection{Categorias}
\label{categorias-def}
\begin{defi}[Categorias]
	    Uma categoria $\mathcal{C}$ é formada pelas seguintes coisas:


\begin{itemize}
    \item Uma coleção de objetos $\text{Obj}(\mathcal{C})$, que geralmente serão denotados por letras maiúsculas $A$, $B$, $C$...
    \item Uma coleção de morfismos $\text{Mor}(\mathcal{C})$, que usualmente serão denotadas por letras minúsculas $f$, $g$, $h$...
\end{itemize}

Onde valem os seguintes axiomas:

\begin{enumerate}
    \item A cada morfismo $f$ de $\text{Mor}(\mathcal{C})$ são associados dois objetos $\text{Dom}(f)$ e $\text{Codom}(f)$ de $\text{Obj}(\mathcal{C})$. \\
    Escrevemos % https://tikzcd.yichuanshen.de/#N4Igdg9gJgpgziAXAbVABwnAlgFyxMJZABgBpiBdUkANwEMAbAVxiRAEEQBfU9TXfIRQBGclVqMWbAELdxMKAHN4RUADMAThAC2SMiBwQkokAzoAjGAwAK-PATYMYanCGr1mrRCDVyuQA
\begin{tikzcd}
A \arrow[r, "f"] & B
\end{tikzcd}
 para abreviar $f \in \text{Mor}(\mathcal{C})$, $\text{Dom}(f) = A$ e $\text{Codom}(f) = B$.
 \item A cada objeto $A$ de $\mathcal{C}$ está associado um morfismo $1_A \in \text{Mor}(\mathcal{C})$ tal que $\text{Codom}(1_A) = \text{Dom}(1_A) = A$.
 \item Para quaisquer dois morfismos $f$ e $g$, tais que $\text{Dom}(f)=\text{Codom}(g)$, há um morfismo associado $f \circ g$, onde $\text{Dom}(f \circ g) = \text{Dom}(g)$ e $\text{Codom}(f \circ g) = \text{Codom}(f)$.
 \\ Isso pode ser representado dizendo que o seguinte diagrama comuta:
% https://q.uiver.app/#q=WzAsMyxbMCwwLCJBIl0sWzEsMCwiQiJdLFsxLDEsIkMiXSxbMCwxLCJmIl0sWzEsMiwiZyJdLFswLDIsImYgXFxjaXJjIGciLDJdXQ==
\[\begin{tikzcd}[column sep=large]
	A & B \\
	& C
	\arrow["f", from=1-1, to=1-2]
	\arrow["{f \circ g}"', from=1-1, to=2-2]
	\arrow["g", from=1-2, to=2-2]
\end{tikzcd}\]

\item Para todo morfismo $f$ de $\text{Mor}(\mathcal{C})$ com $\text{Dom}(f) = A$ e $\text{Codom}(f) = B$, vale que $f \circ 1_A = f$ e $1_B \circ f = f$. Ou seja, o seguinte diagrama comuta:

% https://q.uiver.app/#q=WzAsNCxbMCwwLCJBIl0sWzEsMCwiQSJdLFsxLDEsIkIiXSxbMiwxLCJCIl0sWzAsMSwiMV9BIl0sWzEsMiwiZiJdLFswLDIsImYiLDJdLFsxLDMsImYiXSxbMiwzLCIxX0IiLDJdXQ==
\[\begin{tikzcd}[sep=large]
	A & A \\
	& B & B
	\arrow["{1_A}", from=1-1, to=1-2]
	\arrow["f"', from=1-1, to=2-2]
	\arrow["f", from=1-2, to=2-2]
	\arrow["f", from=1-2, to=2-3]
	\arrow["{1_B}"', from=2-2, to=2-3]
\end{tikzcd}\]
\item Dados os morfismos $f$, $g$, $h$ de $\text{Mor}(\mathcal{C})$, vale que 
$(f \circ g) \circ h = f \circ (g \circ h)$. Ou seja, o seguinte diagrama comuta:
% https://q.uiver.app/#q=WzAsNCxbMCwwLCJBIl0sWzEsMCwiQiJdLFsxLDEsIkMiXSxbMiwxLCJEIl0sWzAsMSwiZiJdLFsxLDIsImciXSxbMCwyLCJnIFxcY2lyYyBmIl0sWzIsMywiaCJdLFsxLDMsImggXFxjaXJjIGciXV0=
\[\begin{tikzcd}[sep=large]
	A & B \\
	& C & D
	\arrow["f", from=1-1, to=1-2]
	\arrow["{g \circ f}", from=1-1, to=2-2]
	\arrow["g", from=1-2, to=2-2]
	\arrow["{h \circ g}", from=1-2, to=2-3]
	\arrow["h", from=2-2, to=2-3]
\end{tikzcd}\]
\end{enumerate}
\end{defi}



%[Bianca]: é mais fácil criar a lista de dependências do que a de consequências.
 
%---------------------------------------------------------------------------------------------------------------------!Draft!-----------------------------------------------------------------------------------------------------------------
\subsection{Categorias}
\label{categorias-ex}
\begin{titlemize}{Lista de dependências}
	\item \hyperref[categorias-def]{categorias-def};\\ %'dependencia1' é o label onde o conceito Dependência 1 aparece (--à arrumar um padrão para referencias e labels--) 
	\item \hyperref[]{};\\
% quantas dependências forem necessárias.
\end{titlemize}

\begin{ex}[Exemplos de Categorias]
	Alguns dos seguintes exemplos não serão tratados com detalhes. No entanto, pode-se consultá-los em quaisquer livros de teoria das categorias.
\begin{itemize}
\item \textbf{Mon} é uma categorial em que os objetos são monóides e os morfismos são homomorfismos de monóides.
\item \textbf{Grp} é a categoria dos grupos e homomorfismo de grupos (A categoria \textbf{Ab} é a categoria dos grupos abelianos.
\item A categoria \textbf{TOP} tem como objetos os espaços topológicos e como morfismos as funções contínuas (há também a categoria $\mathbf{TOP_*}$ dos espaços topológicos com um ponto selecionado, onde os morfismos $f:(X,x) \longrightarrow (Y,y)$ são funções contínuas tais que $f(x) = y$).
\item $\mathbf{Vec(\mathbb{K})}$ é a categoria dos espaços vetoriais sobre o corpo $\mathbb{K}$ e as transformações lineares dos espaços
\item A categoria \textbf{SET} tem como objetos os conjuntos e os morfismos são as funções entre os conjuntos. Ainda, pode-se definir $\mathbf{SET}_\omega$, a categoria dos conjuntos finitos e as funções entre eles.
\item A categoria \textbf{Ord} dos ordinais e das funções entre eles. Da mesma forma, $\mathbf{Ord}_\omega$ é a categoria dos ordinais finitos.

\item Uma relação $\leq$ é dita relação de ordem parcial se satisfaz:
\begin{itemize}
    \item $a \leq a$ para todo $a$.
    \item Se $a \leq b$ e $b \leq a$, então $a = b$ para todos $a$ e $b$.
    \item Se $a \leq b$ e $b \leq c$, então $a \leq c$ para todos $a$, $b$ e $c$.
\end{itemize}
A categoria $\mathbf{PO}$ (partial-order) é definida com morfismos estabelecendo a ordem entre os objetos, isto é, $A \leq B$ se, e somente se, existe $f$ em $Mor(\mathbf{PO})$, tal que % https://q.uiver.app/#q=WzAsMixbMCwwLCJBIl0sWzEsMCwiQiJdLFswLDEsImYiXV0=
\begin{tikzcd}[cramped]
	A & B
	\arrow["f", from=1-1, to=1-2]
\end{tikzcd}
\item Já a categoria \textbf{POS} tem como objetos os conjuntos parcialmente ordenados e os morfismos são funções que preservam a ordem, isto é, se $f:$ % https://q.uiver.app/#q=WzAsMixbMCwwLCJBIl0sWzEsMCwiQiJdLFswLDFdXQ==
\begin{tikzcd}[cramped]
	A & B
	\arrow[from=1-1, to=1-2]
\end{tikzcd}
, e $m \leq n$ em $A$, então $f(m) \leq f(n)$ em $B$.




\end{itemize}

\end{ex}


\begin{titlemize}{Lista de consequências}
	\item \hyperref[homotopia]{homotopia};\\ %'consequencia1' é o label onde o conceito Consequência 1 aparece
	\item \hyperref[]{}
\end{titlemize}

%---------------------------------------------------------------------------------------------------------------------!Draft!-----------------------------------------------------------------------------------------------------------------
\subsection{Isomorfismo}
\label{isomorfismo-em-categorias-def}
\begin{titlemize}{Lista de dependências}
	\item \hyperref[categorias-def]{categorias-def};\\ %'dependencia1' é o label onde o conceito Dependência 1 aparece (--à arrumar um padrão para referencias e labels--) 
\end{titlemize}
\begin{defi}[Isomorfismo]
	Um morfismo $f:A \longrightarrow B$ de uma categoria $\mathcal{C}$ é um isomorfismo se, e somente se, existe um morfismo $g:B \longrightarrow A$, tal que $f \circ g = 1_B$ e $g \circ f = 1_A$. Nesse caso, dizemos que $A$ e $B$ são isomorfos e escrevemos $A \cong B$.
\end{defi}


%[Bianca]: é mais fácil criar a lista de dependências do que a de consequências.

%---------------------------------------------------------------------------------------------------------------------!Draft!-----------------------------------------------------------------------------------------------------------------
\subsection{Funtor}
\label{funtor-categorias-def}
\begin{titlemize}{Lista de dependências}
	\item \hyperref[categorias-def]{Definição de Categoria};\\ %'dependencia1' é o label onde o conceito Dependência 1 aparece (--à arrumar um padrão para referencias e labels--) 
\end{titlemize}
\begin{defi}[Funtor Covariante]
	Um funtor é uma função entre categorias $F: \mathcal{C} \longrightarrow \mathcal{D}$, que associa para cada $A \in Obj(\mathcal{C})$ um único objeto $F(A) \in Obj(\mathcal{D})$ e associa cada morfismo $f \in Mor(\mathcal{C})$ um morfismo $F(f): (A) \longrightarrow F(B)$ , tal que $F(f \circ g) = F(f) \circ F(g) $ e $F(1_A) = 1_{F(A)}$.
\end{defi}

O conceito de funtor é extremamente importante, pois é ele que estabelece uma "ponte" para as diversas áreas da mátematica. Desse modo, podemos ver o grupo fundamental como um funtor da categoria dos espaços topológicos pontuados para a categoria de grupos e homomorfismo de grupos.

\begin{titlemize}{Lista de consequências}
	\item \hyperref[homotopia]{Homotopia};\\ %'consequencia1' é o label onde o conceito Consequência 1 aparece
	\item \hyperref[grupo-fundamental]{Grupo fundamental}
\end{titlemize}

%[Bianca]: é mais fácil criar a lista de dependências do que a de consequências.

%---------------------------------------------------------------------------------------------------------------------!Draft!-----------------------------------------------------------------------------------------------------------------
\subsection{Funtores-Exemplos}
\label{funtor-categorias-ex}
\begin{titlemize}{Lista de dependências}
	\item \hyperref{categorias-ex}{categorias-ex};\\
	\item \hyperref[funtor-categorias-def]{funtor-categorias-def};\\ %'dependencia1' é o label onde o conceito Dependência 1 aparece (--à arrumar um padrão para referencias e labels--) 
\end{titlemize}

\begin{ex}[Funtores Covariantes]
	Os detalhes dos exemplos a seguir são deixados para os leitores.
\begin{itemize}

    \item O funtor identidade $\mathbf{1}_{\mathcal{C}}: \mathcal{C} \longrightarrow \mathcal{C}$ leva todo objeto nele mesmo e todo morfismo nele mesmo, isto é, $\mathbf{1}_{\mathcal{C}}(A) = A$ e $\mathbf{1}_{\mathcal{C}}(f) = f$
    
    \item O funtor potência (power-set functor) $\mathcal{P}:\mathbf{SET} \longrightarrow \mathbf{SET}$, que leva um conjunto $A$ no conjuntos das partes $\mathcal{P}(A)$ e uma função $f:A \longrightarrow B$ para a função $\mathcal{P}(f): \mathcal{P}(A) \longrightarrow \mathcal{P}(B)$, tal que $\mathcal{P}(f)(S) = f(S)$ para todo $S \subseteq A$.

    \item Na topologia algébrica podemos obter de cada espaço topológico pontuado um grupo, chamado de n-ésimo grupo de homotopia. Além disso, para cada função contínua $f: A \longrightarrow B$, podemos obter um homomorfismo de grupos $\pi_n(f):\pi_n(A) \longrightarrow \pi_n(B)$, onde $\pi_n(A)$ e $\pi_n(B)$ são os n-ésimos grupos de homotopia. Dessa forma, construimos um funtor $\pi_n: \mathbf{TOP}_* \longrightarrow \mathbf{Grp}$. 

\end{itemize}
\end{ex}

 
%\begin{figure}[]
%	\centering
%	\includegraphics[width=0.8\textwidth]{}
%	\caption{}
%	\label{fig:}
%\end{figure}

\begin{titlemize}{Lista de consequências}
	\item \hyperref[grupo-fundamental]{grupo-fundamental};\\ %'consequencia1' é o label onde o conceito Consequência 1 aparece
\end{titlemize}

%---------------------------------------------------------------------------------------------------------------------!Draft!-----------------------------------------------------------------------------------------------------------------
\subsection{Transformação Natural}
\label{transformação-natural-categorias-def}
\begin{titlemize}{Lista de dependências}
	\item \hyperref[funtor-categorias-def]{Funtor};\\ %'dependencia1' é o label onde o conceito Dependência 1 aparece (--à arrumar um padrão para referencias e labels--) 
% quantas dependências forem necessárias.
\end{titlemize}
\begin{defi}[Transformação Natural]
	Dados dois funtores % https://q.uiver.app/#q=WzAsMixbMCwwLCJcXG1hdGhjYWx7Q30iXSxbMSwwLCJcXG1hdGhjYWx7RH0iXSxbMCwxLCJGIiwwLHsib2Zmc2V0IjotMX1dLFswLDEsIkciLDIseyJvZmZzZXQiOjF9XV0=
\begin{tikzcd}[cramped,sep=small]
	{\mathcal{C}} & {\mathcal{D}}
	\arrow["F", shift left, from=1-1, to=1-2]
	\arrow["G"', shift right, from=1-1, to=1-2]
\end{tikzcd}, definimos a transformação natural $\eta:F \Longrightarrow G$ da seguinte forma: \\
$\eta$ é uma família de flechas $(\eta_A: F(A) \longrightarrow G(A))_{A \in Obj(\mathcal{C})}$, tal que $\eta_B \circ F(f) = G(f) \circ \eta_A$. Isso equivale a dizer que o seguinte diagrama comuta.
% https://q.uiver.app/#q=WzAsNixbMiwwLCJGKEEpIl0sWzQsMCwiRyhBKSJdLFsyLDIsIkYoQikiXSxbNCwyLCJHKEIpIl0sWzAsMCwiQSJdLFswLDIsIkIiXSxbNCw1LCJmIiwyXSxbMCwyLCJGKGYpIiwyXSxbMSwzLCJHKGYpIiwyXSxbMCwxLCJcXGV0YV9BIiwxXSxbMiwzLCJcXGV0YV9CIiwxXV0=
\[\begin{tikzcd}[sep=large]
	A && {F(A)} && {G(A)} \\
	\\
	B && {F(B)} && {G(B)}
	\arrow["f"', from=1-1, to=3-1]
	\arrow["{\eta_A}"{description}, from=1-3, to=1-5]
	\arrow["{F(f)}"', from=1-3, to=3-3]
	\arrow["{G(f)}"', from=1-5, to=3-5]
	\arrow["{\eta_B}"{description}, from=3-3, to=3-5]
\end{tikzcd}\]

    
\end{defi}

A transformação natural identidade é a transformação $1_F:F \Longrightarrow F$, tal que $(1_F)_A: F(A) \longrightarrow F(A)$ é a identidade de $F(A)$. \\
Ainda, dadas as trasformações naturais $\eta:F \Longrightarrow G$ e $\mu: G \Longrightarrow H$, onde $F, G$ e $H$ são funtores de uma categoria $\mathcal{C}$ para uma categoria $\mathcal{D}$, podemos definir a transformação $(\mu \circ \eta): F \Longrightarrow H$ como sendo a família de flechas $(\mu_A \circ \eta_A: F(A) \longrightarrow G(A))_{A \in Obj(\mathcal{C})}$.

Dessa forma, podemos definir o que é a categoria de funtores: 

$\mathbf{Fun(\mathcal{C}, \mathcal{D})}$ é a categoria em que os objetos são funtores de $\mathcal{C}$ para $\mathcal{D}$ e os morfismos são transformações naturais dos funtores de $Obj(\mathbf{Fun(\mathcal{C}, \mathcal{D})})$.

\begin{titlemize}{Lista de consequências}
	\item \hyperref[grupo-fundamental]{Grupo fundamental};\\ %'consequencia1' é o label onde o conceito Consequência 1 aparece
	\item \hyperref[homotopia]{Homotopia}
\end{titlemize}

%[Bianca]: é mais fácil criar a lista de dependências do que a de consequências.

%---------------------------------------------------------------------------------------------------------------------!Draft!-----------------------------------------------------------------------------------------------------------------
\subsection{Transformação Natural}
\label{transformação-natural-categorias-ex}
\begin{titlemize}{Lista de dependências}
	\item \hyperref[transformação-natural-categorias-def]{Transformação natural};\\ %'dependencia1' é o label onde o conceito Dependência 1 aparece (--à arrumar um padrão para referencias e labels--) 
	\item \hyperref[categorias-ex]{Categorias-Exemplos};\\
% quantas dependências forem necessárias.
\end{titlemize}

\begin{ex}[Transformações Naturais]
	Alguns exemplos de transformações naturais.
 
    \begin{itemize}
        \item $J:\mathbf{1_{\mathbf{Vec(\mathbb{K})}}} \Longrightarrow ()^{**} $ é uma transformação natural do funtor identidade no funtor bidual $()^{**}$, de tal forma que $J_X(x)(z^*) = z^*(x)$. Ainda, $f^{**}: A^{**} \longrightarrow B^{**}$ para algum morfismo $f: A \longrightarrow B$ é a transformação linear, tal que $f^{**}(z^{**})(x^*) = z^{**}(x^{*} \circ f)$. Então o seguinte diagrama comuta:
       % https://q.uiver.app/#q=WzAsNixbMiwwLCJWIl0sWzQsMCwiVl57Kip9Il0sWzIsMiwiVyJdLFs0LDIsIldeeyoqfSJdLFswLDAsIlYiXSxbMCwyLCJXIl0sWzQsNSwiTCIsMl0sWzAsMiwiTCIsMl0sWzEsMywiTF57Kip9IiwyXSxbMCwxLCJKX1YiLDEseyJzdHlsZSI6eyJ0YWlsIjp7Im5hbWUiOiJob29rIiwic2lkZSI6InRvcCJ9fX1dLFsyLDMsIkpfVyIsMSx7InN0eWxlIjp7InRhaWwiOnsibmFtZSI6Imhvb2siLCJzaWRlIjoidG9wIn19fV1d
\[\begin{tikzcd}
	V && V && {V^{**}} \\
	\\
	W && W && {W^{**}}
	\arrow["L"', from=1-1, to=3-1]
	\arrow["{J_V}"{description}, hook, from=1-3, to=1-5]
	\arrow["L"', from=1-3, to=3-3]
	\arrow["{L^{**}}"', from=1-5, to=3-5]
	\arrow["{J_W}"{description}, hook, from=3-3, to=3-5]
\end{tikzcd}\].

\item Temos o funtor $\#: \mathbf{SET_\omega} \longrightarrow \mathbf{Ord}_\omega$ que leva um conjuto finito em seu respectivo ordinal. Dessa forma, uma classe de bijeções $(\alpha_A: A \hookrightarrow \#(A))_{A \in Obj(\mathbf{SET_\omega})}$ define uma transformação natural.

% https://q.uiver.app/#q=WzAsNixbMiwwLCJBIl0sWzQsMCwiXFwjQSJdLFsyLDIsIkIiXSxbNCwyLCJcXCNCIl0sWzAsMCwiQSJdLFswLDIsIkIiXSxbNCw1LCJmIiwyXSxbMCwyLCJmIiwyXSxbMSwzLCJcXCNmIiwyXSxbMCwxLCJcXGFscGhhX0EiLDFdLFsyLDMsIlxcYWxwaGFfQiIsMV1d
\[\begin{tikzcd}
	A && A && {\#A} \\
	\\
	B && B && {\#B}
	\arrow["f"', from=1-1, to=3-1]
	\arrow["{\alpha_A}"{description}, from=1-3, to=1-5]
	\arrow["f"', from=1-3, to=3-3]
	\arrow["{\#f}"', from=1-5, to=3-5]
	\arrow["{\alpha_B}"{description}, from=3-3, to=3-5]
\end{tikzcd}\]


        
    \end{itemize}
\end{ex}


\begin{titlemize}{Lista de consequências}
	\item \hyperref[hom-grupo-fundamental]{homomorfismo-de-grupo-fundamental};\\ %'consequencia1' é o label onde o conceito Consequência 1 aparece
\end{titlemize}

%---------------------------------------------------------------------------------------------------------------------!Draft!-----------------------------------------------------------------------------------------------------------------
\subsection{Equivalência de Categorias}
\label{equivalência-de-categorias-def}
\begin{titlemize}{Lista de dependências}
	\item \hyperref[funtor-categorias-def]{funtor-categorias-def};\\ %'dependencia1' é o label onde o conceito Dependência 1 aparece (--à arrumar um padrão para referencias e labels--) 
	\item \hyperref[transformação-natural-categorias-def]{transformação-natural-categorias-def};\\
  \item \hyperref[isomorfismo-em-categorias-def]{isomorfismo-em-categorias-def};\\
% quantas dependências forem necessárias.
\end{titlemize}
\begin{defi}[Equivalência de Categorias]
	Uma categoria $\mathcal{C}$ é equivalente a uma categoria $\mathcal{D}$ se, e somente se, existem funtores $F: \mathcal{C} \longrightarrow \mathcal{D}$ e $G: \mathcal{D} \longrightarrow \mathcal{C}$, onde se cumpre $F \circ G \cong \mathbf{1}_\mathcal{D}$ e $G \circ F \cong \mathbf{1}_\mathcal{C}$.
 Onde $\cong$ é o isomorfismo entre os objetos da categoria dos funtores $\mathbf{Fun(\mathcal{C}, \mathcal{D})}$, visto na seção \hyperref[transformação-natural-categorias-def]{transformação-natural-categorias-def}.
\end{defi}

Note a semelhança dessa definição com a noção de espaços homotópicamente equivalentes.

\begin{titlemize}{Lista de consequências}
	\item \hyperref[grupo-fundamental]{grupo-fundamental};\\ %'consequencia1' é o label onde o conceito Consequência 1 aparece
	\item \hyperref[]{}
\end{titlemize}

%[Bianca]: é mais fácil criar a lista de dependências do que a de consequências.

\subsection{Pushout de Categorias}
\label{pushout-de-categorias-def}
\begin{titlemize}{Lista de dependências}
	\item \hyperref[categorias-def]{categorias-def};\\ 
\end{titlemize}
\begin{defi}[Pushout de Categorias]
	Considere os morfismos $f:C \rightarrow A$ e $g:C \rightarrow B$. Definimos o Pushout de $f$ e $g$, caso exista, como sendo a tripla $\langle A +_C B, p, q\rangle$, onde $p: A \to A+_C B$ e $q: B \to A +_C B$ com $p \circ f = q \circ g$, e se $p': A \to P$ e $q':B \to P$ é tal que $p' \circ f = g
    q' \circ g$, então existe um único morfismo $k:A +_C B \to P$ que faz o seguinte diagrama comutar.

    % https://q.uiver.app/#q=WzAsNSxbMywzLCJDIl0sWzMsMSwiQSJdLFsxLDMsIkIiXSxbMSwxLCJBICtfQyBCIl0sWzAsMCwiUCJdLFswLDEsImYiLDJdLFswLDIsImciXSxbMSwzLCJwIiwyXSxbMiwzLCJxIl0sWzMsNCwiayIsMCx7InN0eWxlIjp7ImJvZHkiOnsibmFtZSI6ImRhc2hlZCJ9fX1dLFsxLDQsInAnIiwyLHsiY3VydmUiOjN9XSxbMiw0LCJxJyIsMCx7ImN1cnZlIjotMn1dXQ==
\[\begin{tikzcd}
	P \\
	& {A +_C B} && A \\
	\\
	& B && C
	\arrow["k", dashed, from=2-2, to=1-1]
	\arrow["{p'}"', curve={height=18pt}, from=2-4, to=1-1]
	\arrow["p"', from=2-4, to=2-2]
	\arrow["{q'}", curve={height=-12pt}, from=4-2, to=1-1]
	\arrow["q", from=4-2, to=2-2]
	\arrow["f"', from=4-4, to=2-4]
	\arrow["g", from=4-4, to=4-2]
\end{tikzcd}\]
\end{defi}

Na categoria dos conjuntos, o Pushout é obtido tomando a união disjunta de dois conjuntos $A \coprod B$ e quocientando pela menor relação de equivalência que contém o conjunto $\{(f(x), g(x)) \in A \times B; \ x \in C\}$.



%%% Local Variables:
%%% mode: LaTeX
%%% TeX-master: "../Alg.Top-Wiki"
%%% End:

\end{document}


%novos assuntos/secções devem ser adicionados através do comando \import{conteudo/}{assunto} para adicionar o arquivo conteudo/assunto.tex

%%% Local Variables:
%%% mode: LaTeX
%%% TeX-master: t
%%% End:
